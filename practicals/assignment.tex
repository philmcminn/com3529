\input{common/latex-teach-common/page-setup}
\input{common/latex-teach-common/packages}
\input{common/latex-teach-common/macros}

\renewcommand{\title}[2]{
    \documenttitle{Com3529}{Software Testing \& Analysis}{#1}{#2}
}

% URLs
\newcommand{\coderepourl}{\url{https://github.com/philmcminn/com3529-code}\xspace}

% Code snippets
\newcommand{\code}[1]{\mbox{\tt #1}\xspace}

% README file
\newcommand{\readmefile}{{\tt README.md}\xspace}

% Directories
\newcommand{\lecturesdirectory}{\code{lectures}\xspace}
\newcommand{\practicalsdirectory}{\code{practicals}\xspace}

% Packages
\newcommand{\lecturespackage}{\code{uk.ac.shef.com3529.lectures}}
\newcommand{\lecturesexecutionpackage}{\code{uk.ac.shef.com3529.lectures.execution}}
\newcommand{\practicalspackage}{\code{uk.ac.shef.com3529.practicals}}

% Classes
\newcommand{\calendarclass}{\code{Calendar}}
\newcommand{\randomlytesttriangleclass}{\code{RandomlyTestTriangle}}
\newcommand{\signutilsclass}{\code{SignUtils}}
\newcommand{\spacedefenderclass}{\code{SpaceDefender}}
\newcommand{\stringutilsbuggyoneclass}{\code{StringUtilsBuggy1}}
\newcommand{\triangleclass}{\code{Triangle}}
\newcommand{\weekoneclass}{\code{Week1}}
\newcommand{\teststringutilsbuggyoneclass}{\code{TestStringUtilsBuggy1}}
\newcommand{\testweekoneclass}{\code{TestWeek1}}


% Methods
\newcommand{\classifymethod}{\code{classify}}
\newcommand{\daysbetweentwodatesmethod}{\code{daysBetweenTwoDates}}
\newcommand{\instrumentedclassifymethod}{\code{instrumentedClassify}}
\newcommand{\randomlytestclassifymethod}{\code{randomlyTestClassify}}
\newcommand{\signmethod}{\code{sign}}

% for logic tables
\newcommand{\LTTrue}{T}
\newcommand{\LTFalse}{F}


\booltrue{shownotes}


\begin{document}

\title{Assignment, Spring 2021}{Automated Tool Support for Logic Coverage of Java Code}

\section{Overview}

The aim of this assignment is to develop automated tool support for logic
testing of Java methods. The aim of this tool, or framework, is to minimise the
amount of effort required by a human tester when faced with the task of manually
writing a test suite for some arbitrary Java method that they might be given.
%
The more automated support your tool/framework can provide, the higher the mark
you will get. 

A very basic submission that is worthy of a pass mark, for example, might
involve implementing Random Testing for Condition Coverage (i.e., a very basic
coverage criterion), along with a scheme for manually instrumenting conditions.
(This would be a simple extension of what we did in the Week~5 lecture for
randomly generating test cases for Branch Coverage of the \classifymethod~method
of the \triangleclass~class.)

A more advanced submission might seek to provide automated tool support for a
more advanced coverage criterion like MCDC. It might automatically parse Java
methods to obtain branch predicates and figure out what the test requirements
are that are needed by such a criterion. It might automatically instrument the
Java code, as opposed to relying on a tester manually having to insert logging
statements. It might apply a more advanced test generation method, for example
Search-Based Testing. A more advanced submission might even automatically write
out the Java code statements needed to produce a JUnit test suite. 

Section~\ref{sec:requirements} goes into further detail about the requirements
behind this assignment, and the different aspects that you are expected to
tackle. 

Before we go into further detail, there are two key things you should know about
this assignment:

\begin{itemize}
    \item {\bf You can use third-party Java libraries as part of your
    assignment.} For example, I wouldn't expect you to write a full-featured
    parser for Java, if you need one. (Alternatively, you may want to write a
    very basic parser yourself, that provides simple and limited support, it is
    up to you.)
    %
    If you can source a package to do certain key tasks (that might normally
    take months to program yourself), then you can use it. (This is a Software
    Engineering module, after all.)
    
    However, you must include any Java packages or tools ``as is'' (i.e.,
    without modifying them) --- for example as a {\tt .jar} file --- and declare
    it as a dependency of your tool. (This is quite easy to achieve in Maven,
    which we have been using throughout the course.) That is, you must not copy
    anybody's code and use it as part of your own tool, as though you had
    written it yourself. This would amount to plagiarism, of course.

    \item {\bf You can work as an individual or in teams of up to four people.}    
    Marks will be awarded on the basis of what you have achieved in proportion
    to the number of people in your team. The advantage of working in a team is
    that you will be able to tackle more aspects of the project, while also
    being free to discuss the different aspects of the problem it entails and
    give eachother help, without fear of falling foul of academic rules
    regarding assignment collusion. Unless you specify otherwise (e.g., because
    of a large discrepancy in the distribution of the work among team members),
    the overall mark you achieve for the assignment will be awarded to each
    member of the group.
\end{itemize}

%%%%%%

\section{Submission of Your Work}
\label{sec:submission}

Your submission should take the form of a GitHub repository. Your repository
should contain all the Java code needed to run and operate your tool, along with
example Java methods on which your tool with run, and for which it successfully
works with. (Note there is no choice of language here, this assignment must be
completed in Java.)

Your GitHub repository should contain a {\tt README.md} file in the root
directory, written in MarkDown. The {\tt README.md} should contain:

\begin{itemize}
    \item Details of your team, listing each person involved, their Sheffield
    email address, and the work they contributed to the project. 
    
    \item A section overviewing the support your tool provides and what
    automtation it delivers. This section should further include a detailed list
    of all of your tool's different features and how they would assist a human
    tester perform logic testing in practice.

    \item A section detailing how to install and run your tool (or how to run
    the different parts of your tool, if different stages are required). This
    section should include a list of libraries or utilities your tool is
    dependent on, and how to get hold of them. (A simple way to handle this
    requirement is to limit yourself to packages that can be automatically
    obtained via Maven (\url{https://maven.apache.org/}) or Gradle
    (\url{https://gradle.org/}) and use one of those build automation tools. 

    \item One or more worked examples that demonstrate how to use your tool in
    practice. 
\end{itemize}

The deadline for this assignment is {\bf Monday, 26 April 2020, 5pm (Week 9)}.

By this date, you should have emailed me, Phil McMinn ({\tt
p.mcminn@sheffield.ac.uk}), with the URL of your repository.

You must not make any further changes to your repository after this date. If you
do, your work will be subject to standard Department of Computer Science
lateness penalties, i.e. 5\% per working day following the submission date, up
to a maximum of five days --- after which you will score zero.

%%%%%%

\section{Help and Questions}

The lab session in Week~5 will be devoted any immediate questions about the
assignment. 

You may ask further questions about the assignment via the module discussion
board, where a member of the teaching team will respond to your query.

%%%%%%

\section{Detailed Requirements}
\label{sec:requirements}

There are many aspects to writing an automated test case generation
tool/framework. Your submission will consider each of the following aspects, but
may not need to address them completely. In other words, your submission may
only provide limited automation for some aspect, requiring a human tester to do
the rest of the work. The more automation you can provide ``out of the box'',
however, the higher the mark you will achieve. 

In this section, we will consider the \calculatemethod~method of the
\bmicalculatorclass~class, provided in the \practicalspackage~package of the
online code repository supporting this module (available at \coderepourl) and
listed here:

\begin{center} 
    \scriptsize
    \begin{tabular}{rl}
        \toprule
        1 & \verb$package uk.ac.shef.com3529.practicals;$\\
        2 & \verb$$\\
        3 & \verb$public class BMICalculator {$\\
        4 & \verb$$\\
        5 & \verb$    public enum Type {$\\
        6 & \verb$        UNDERWEIGHT,$\\
        7 & \verb$        NORMAL,$\\
        8 & \verb$        OVERWEIGHT,$\\
        9 & \verb$        OBESE;$\\
        10 & \verb$    }$\\
        11 & \verb$$\\
        12 & \verb$    public static Type calculate(double weightInPounds, int heightFeet, int heightInches) {$\\
        13 & \verb$        double weightInKilos = weightInPounds * 0.453592;$\\
        14 & \verb$        double heightInMeters = ((heightFeet * 12) + heightInches) * .0254;$\\
        15 & \verb$        double bmi = weightInKilos / Math.pow(heightInMeters, 2.0);$\\
        16 & \verb$$\\
        17 & \verb$        if (bmi < 18.5) {$\\
        18 & \verb$            return Type.UNDERWEIGHT;$\\
        19 & \verb$        } else if (bmi >= 17.5 && bmi < 25) {$\\
        20 & \verb$            return Type.NORMAL;$\\
        21 & \verb$        } else if (bmi >= 25 && bmi < 30) {$\\
        22 & \verb$            return Type.OVERWEIGHT;$\\
        23 & \verb$        } else {$\\
        24 & \verb$            return Type.OBESE;$\\
        25 & \verb$        }$\\
        26 & \verb$    }$\\
        27 & \verb$}$\\
        \bottomrule
    \end{tabular}
\end{center}

\subsection{Analysis of the Method Under Test}

To start at the beginning, your tool needs to know what conditions are present
in the method under test and their form. For example, the
\calculatemethod~method has branch predicates on lines 17, 19, and 21. The
predicate on line 17 only has one condition, while the predicates on lines 19
and 21 both consist of two conditions composed by the {\tt \&\&} logical
operators. Furthermore, the branch predicate on line 19 won't be encountered
unless the branch predicate on line 17 evaluates to false, while the branch
predicate on line 21 would be encountered unless both predicates on lines 17 and
19 evaluate to false.

At a basic level, the tool could simple expect to be inputted this information
manually by the tester. This could be provided via some additional file, or the
tester could use your framework as an API (i.e., a package of useful Java
methods for supporting automated testing), writing the required information as
parameters to methods provided by your framework. 
%
Of course, this wouldn't be providing much automated support at all, since the
burden of work is on the human. But it provides a basis by which to start ---
your framework will need some data structures in which to store all of this
information. 

Later on, you may choose to provide further automation in this area, for example
by parsing the Java method. You could write your own simple parser that is
capable of detecting {\it if}, {\it while}, and {\tt for} statements, and
dissecting the conditions within them.
%
Or, you may wish to enlist the support of a full-blown parsing tool, such as the
JavaParser library (\url{https://javaparser.org/}).

\subsection{Generating Test Requirements}

Once you've figured out the conditions in the method under test, you can figure
out the test requirements needed by various logic coverage criteria. This is
easy for a basic criterion like Condition Coverage, but more technical for a
coverage criterion like MCDC (in restricted or correlated form). 

For the branch predicate on line~17, there is only one condition, so this is
trivial. All logic coverage criteria for this predicate are the same as Branch
Coverage --- two test requirements, where the predicate/condition evaluates to
true in one requirement, and false in the other.

Things get more complicated for the branch predicates on lines~19 and~21,
however, which are composed of two conditions. Generating the test requirements
for Condition Coverage involves test requirements that executing each individual
condition as true and false. MCDC is more complicated, and you will need to
consider whether you will support either the restricted or correlated variant
(or both). Of course, you can support multiple coverage criteria for further
marks! 

%%%%

\subsection{Instrumentation}

One challenge your tool needs to solve is how to know which conditions have been
covered, i.e. whether they were executed or not, and if they were, whether they
evaluated to true or false. 

At the very least, you will need to provide a scheme so that a tester can insert
logging statements themselves that are needed by the tool in order to run. 

We did this for the \classifymethod~method of the \triangleclass~class in the
Week~5 lecture, for example. For considering individual conditions, you might
want to do something cleverer. For example, instrumentation of line~21 of the
\calculatemethod~method of the \bmicalculatorclass~class could take the form:

\begin{center}
{\tt \scriptsize ... if (logCondition(3, bmi >= 17.5) \&\& logCondition(4, bmi <
25) \{ ...} 
\end{center}

where {\tt logCondition} is a method provided by your framework that logs the ID
number of the condition as the first parameter, takes the condition as the
second parameter, and returns the boolean result of the evaluation of the
condition so that the program preserves its normal functionality. 

\begin{center} 
    \scriptsize
    \begin{tabular}{l}
        \toprule
        \verb$public void logCondition(int id, boolean condition) {$\\
        \verb$    boolean result = condition;$\\
        \verb$    // ... log the id somewhere, along with the result, $\\
        \verb$    // thereby storing whether the condition was executed$\\
        \verb$    // as true or false, for computing coverage later on...$\\
        \verb$    return result;$\\
        \verb$}$\\
        \bottomrule
    \end{tabular}
\end{center}

If you're planning on incorporating a Search-Based method into your tool (see
below), the instrumentation will likely need to log the values of variables so
that your tool can also compute fitness. For example:

\begin{center} 
    \scriptsize
    \begin{tabular}{l}
        \toprule
        \verb$public void logGreaterThanOrEqualsCondition(int id, double leftOperand, double rightOperand) {$\\
        \verb$    double fitness = leftOperand - rightOperand;$\\
        \verb$    boolean result = false;$\\
        \verb$    if (fitness <= 0) {$\\
        \verb$        result = true;$\\
        \verb$    } else {$\\
        \verb$        fitness += K;$\\
        \verb$    }$\\
        \verb$    // ... log the id somewhere, along with the result,$\\
        \verb$    // thereby storing whether the condition was executed$\\
        \verb$    // as true or false, for computing coverage later on...$\\
        \verb$$\\
        \verb$    // ... also log fitness ...$\\
        \verb$$\\
        \verb$    return result;$\\
        \verb$}$\\
        \bottomrule
    \end{tabular}
\end{center}

Or,
you may wish to have the tool insert those instrumentation statements itself.
This could be achieved in conjunction with the Java Parser, mentioned earlier.

%%%%

\subsection{Test Data Generation}

Your tool needs to provide some level of test data generation support. This
could be random, as per the example in the Week~5 lecture. However, you may wish
to provide a more advanced mechanism, using Search-Based techniques. You could
use the Alternating Variable Method (e.g., via the publicly available
open-source AVMf --- \url{https://avmframework.org} or by incorporating a simple
Evolutionary Algorithm of your own.

(Hint: The AVMf provides an example of test data generation that you could study
to see what ideas you could borrow for this assignment.)

%%%%

\subsection{Computing Coverage}

For a given test case, and given the logging information provided by your
instrumentation, you need to be able to figure out which test requirements were
covered by the test case and its coverage level. This would contribute to the
coverage level for a series of test cases that together maximise coverage,
provided as an automatically generated test suite for a tester to use.

%%%%

\subsection{Outputting Test Cases}

Your tool needs to output the test cases it generates in some form or other.
This could take the form of a very basic series of {\tt System.out.println}
statements that write information to the console. To provide comprehensive tool
support though, a tester would want a JUnit test suite. This may only be partial
--- containing calls to the method under test with test data only, requiring the
human tester to fill in the assertions. Or, it could have a go at providing
those assertions as well by observing the outputs of the method given the
inputs. For example, suppose the values 182, 6, and 0 were generated for the
\calculatemethod~method:

\begin{center}
{\tt \scriptsize assertTrue(BMICalculator.calculate(182, 6, 0) == Type.NORMAL);}
\end{center}

For methods that return integers or doubles, you could use {\tt assertEquals} instead.

\subsection{Examples}

You need to devise some examples on which to try your test tool/framework, and
present them as evidence of what your tool does as part of your submission. You
can use the \bmicalculatorclass as one of these examples. 

%%%%

\subsection{Documentation}

Finally, you should document what your tool does, and how to use it via a {\tt
README.md} file provided in the root directory of your GitHub repository. See
Section~\ref{sec:submission} for more details on what to do here.

%%%%

\section{Indicative Marking Scheme}

Your submission will be graded according to the following criteria. 

Each of the criteria will be equally weighted to produce a final mark,  taking
into account of the number of people in your team, or whether you are working
individually. 

\begin{tabular}{p{10em}p{15em}p{5em}}

    \toprule

    {\bf Automation Aspect} & {\bf Level of Achievement} & {\bf Indicative Grade} \\

    \midrule

    Analysis of the Method Under Test \\

    \midrule

    Test Requirement Generation \\

    \midrule

    Instrumentation \\

    \midrule

    Test Data Generation \\

    \midrule

    Computing Coverage \\

    \midrule

    Test Suite Output \\

    \midrule

    Quality of GitHub Repository and Documentation \\

    \bottomrule

\end{tabular}

\end{document}