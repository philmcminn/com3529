\input{common/latex-teach-common/page-setup}
\input{common/latex-teach-common/packages}
\input{common/latex-teach-common/macros}

\renewcommand{\title}[1]{
    \moduletitle{Com3529}{Software Testing \& Analysis}{#1}
}

% URLs
\newcommand{\coderepourl}{\url{https://github.com/philmcminn/com3529-code}\xspace}

% Code snippets
\newcommand{\code}[1]{\mbox{\tt #1}\xspace}

% README file
\newcommand{\readmefile}{{\tt README.md}\xspace}

% Directories
\newcommand{\lecturesdirectory}{\code{lectures}\xspace}
\newcommand{\practicalsdirectory}{\code{practicals}\xspace}

% Packages
\newcommand{\lecturespackage}{\code{uk.ac.shef.com3529.lectures}}
\newcommand{\practicalspackage}{\code{uk.ac.shef.com3529.practicals}}

% Classes
\newcommand{\calendarclass}{\code{Calendar}}
\newcommand{\signutilsclass}{\code{SignUtils}}
\newcommand{\spacedefenderclass}{\code{SpaceDefender}}
\newcommand{\stringutilsbuggyoneclass}{\code{StringUtilsBuggy1}}
\newcommand{\triangleclass}{\code{Triangle}}
\newcommand{\weekoneclass}{\code{Week1}}
\newcommand{\teststringutilsbuggyoneclass}{\code{TestStringUtilsBuggy1}}
\newcommand{\testweekoneclass}{\code{TestWeek1}}

% Methods
\newcommand{\classifymethod}{\code{classify}}
\newcommand{\daysbetweentwodatesmethod}{\code{daysBetweenTwoDates}}
\newcommand{\signmethod}{\code{sign}}

% for logic tables
\newcommand{\LTTrue}{T}
\newcommand{\LTFalse}{F}


\booltrue{shownotes}


\begin{document}

\title{Assignment 2021}{Automated Tool Support for Logic Coverage of Java Code}

\section{Overview}

The aim of this assignment is to develop automated tool support for logic
testing of Java methods.

The more automated support your tool can provide, the higher the mark you will
get. 

A very basic submission that is worthy of a pass mark, for example, might
involve implementing Random Testing for Condition Coverage (i.e., a very basic
coverage criterion), along with a scheme for manually instrumenting conditions.
(This is much like what we did in lectures for randomly generating test cases
for Branch Coverage of the \classifymethod~method of the \triangleclass~class in
lectures.)

A more advanced submission might seek to provide automated tool support for a
more advanced Coverage Criterion like MCDC. It might automatically parse Java
methods to obtain branch predicates and figure out what the test requirements
are that are needed by such a criterion. It might automatically instrument the
Java code, as opposed to relying on a tester manually having to insert logging
statements. It might apply a more advanced test generation method, for example
Search-Based Testing. A more advanced submission might even automatically write
out the Java code statements needed to produce a JUnit test suite. 

There are two key things you should know about this assignment, before we go
into further details about it:

\begin{itemize}
    \item {\bf You can use third-party Java libraries as part of your
    assignment.} For example, I wouldn't expect you to write a full-featured
    parser for Java, if you need one. (Alternatively, you may want to write a
    very basic parser yourself, that provides simple and limited support, it is
    up to you.)
    
    If you can source a package to do certain key tasks (that might take months
    to program yourself), then you can use it. (This is a Software Engineering
    module, after all!)
    
    However, you must include the package ``as is'' (i.e., without modifying
    it), for example as a {\tt .jar} file, and declare it as a dependency of
    your tool. (This is quite easy to achieve in Maven, which we have been using
    throughout the course.) That is, you must not copy anybody's code and use it
    as part of your own tool, as though you had written it yourself (this is
    plagiarism, of course).

    \item {\bf You can work as an individual or in teams of up to four people.}    
    Marks will be awarded on the basis of what you have achieved and the number
    of people in your team. The advantage of working in a team is that you will
    be able to achieve more and give eachother help, as opposed to working on
    your own. Unless you specify otherwise, the overall mark for the assignment
    will be given to each team member.
\end{itemize}

%%%%%%

\section{Submission of Your Work}

Your submission should take the form of a GitHub repository. Your repository
should contain all the code needed to run and operate your tool, along with
example code on which it will run and for which it works with.

Your GitHub repository should contain a {\tt README.md} file in the root
directory, written in MarkDown. The {\tt README.md} should contain:

\begin{itemize}
    \item Details of your team, listing each person involved, their Sheffield
    email address, and what they contributed to the project. 
    
    \item A section overviewing what your tool support does and what it achieve.
    This section should further include a detailed list of all of your tool's
    different features and how they would assist a tester in practice.

    \item A section detailing how to install and run your tool (or how to run
    the different parts of your tool, if different stages are required). This
    section should include a list of libraries or utilities your tool is
    dependent on, and how to get hold of them. (A simple way to handle this
    requirement is to limit yourself that can be obtained via Maven
    (\url{https://maven.apache.org/}) or Gradle (\url{https://gradle.org/}) and
    use one of those build automation tools. 

    \item A worked example that demonstrates how to use your tool with one of
    your examples. 
\end{itemize}

The deadline for this assignment is {\bf Monday, 26 April 2020, 5pm (Week 9)}.

By this date, you should have emailed me, Phil McMinn ({\tt
p.mcminn@sheffield.ac.uk}), the URL of your repository.

You must not make any further changes to your repository after this date. If you
do, you will incur a penalty of 5\% per day following the submission date, up to
a maximum of five days, after which you will score zero.

%%%%%%

\section{Help and Questions}

The lab session in Week 5 will be devoted any immediate questions about the
assignment. 

You may ask further questions about the assignment via the module discussion
board, where a member of the teaching team will respond to your query.


\section{Detailed Requirements}



\end{document}