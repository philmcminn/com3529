\input{common/latex-teach-common/page-setup}
\input{common/latex-teach-common/packages}
\input{common/latex-teach-common/macros}

\renewcommand{\title}[2]{
    \documenttitle{Com3529}{Software Testing \& Analysis}{#1}{#2}
}

% URLs
\newcommand{\coderepourl}{\url{https://github.com/philmcminn/com3529-code}\xspace}

% Code snippets
\newcommand{\code}[1]{\mbox{\tt #1}\xspace}

% README file
\newcommand{\readmefile}{{\tt README.md}\xspace}

% Directories
\newcommand{\lecturesdirectory}{\code{lectures}\xspace}
\newcommand{\practicalsdirectory}{\code{practicals}\xspace}

% Packages
\newcommand{\lecturespackage}{\code{uk.ac.shef.com3529.lectures}}
\newcommand{\lecturesexecutionpackage}{\code{uk.ac.shef.com3529.lectures.execution}}
\newcommand{\practicalspackage}{\code{uk.ac.shef.com3529.practicals}}

% Classes
\newcommand{\calendarclass}{\code{Calendar}}
\newcommand{\randomlytesttriangleclass}{\code{RandomlyTestTriangle}}
\newcommand{\signutilsclass}{\code{SignUtils}}
\newcommand{\spacedefenderclass}{\code{SpaceDefender}}
\newcommand{\stringutilsbuggyoneclass}{\code{StringUtilsBuggy1}}
\newcommand{\triangleclass}{\code{Triangle}}
\newcommand{\weekoneclass}{\code{Week1}}
\newcommand{\teststringutilsbuggyoneclass}{\code{TestStringUtilsBuggy1}}
\newcommand{\testweekoneclass}{\code{TestWeek1}}


% Methods
\newcommand{\classifymethod}{\code{classify}}
\newcommand{\daysbetweentwodatesmethod}{\code{daysBetweenTwoDates}}
\newcommand{\instrumentedclassifymethod}{\code{instrumentedClassify}}
\newcommand{\randomlytestclassifymethod}{\code{randomlyTestClassify}}
\newcommand{\signmethod}{\code{sign}}

% for logic tables
\newcommand{\LTTrue}{T}
\newcommand{\LTFalse}{F}


\booltrue{shownotes}


\begin{document}

\title{Week 2 Practical --- Coverage Part 1}

\begin{enumerate}

    \item Do the ``Who Wants To Be A Software Tester?'' Quiz for this week.
    If you're unclear about any of the answers then ask!

    \item Navigate to the \lecturespackage~package, and open the
    \triangleclass class. The class has a method, \classifymethod, that
    categorises the type of the triangle based on the lengths of its three sides,
    which correspond to the parameters to the method --- {\tt side1}, {\tt
    side2}, and {\tt side3}. The {\it equilateral} triangle is one in which all sides
    are the same length, an isosceles triangle is one in which at least two
    sides are the same length, a scalene triangle is any other (valid) triangle. 

        \begin{enumerate}

            \item Draw the CFG for the \classifymethod~method.
            
            \item Write a JUnit test suite for \triangleclass that gives 
            full Branch Coverage. 
            
            \item Enumerate all the paths through the CFG for the method. (There
            are no loops in the method, so this is possible.) Which paths are
            infeasible?
            
        \end{enumerate}

    \item The code segment that returns when a triangle is isosceles could be re-written as:

        \verb$if ((side1 + side2 > side3) &&$\\ 
        \verb$      (side1 == side2 || side2 == side3) &&$\\
        \verb$      (side1 != side2 || slide2 != side3)) {$\\
        \verb$    return Type.ISOCELES;$\\
        \verb$}$

        For the above branch predicate:

        \begin{enumerate}

            \item Enumerate each of the {\it conditions}.
            
            \item Determine the test requirements for coverage with Restricted
            MCDC. Format this as a table, which shows the truth values of the
            conditions and the branch predicate in columns of the table (as done
            in the lecture). Which test requirements are infeasible, if any?

            \item The same question as before, but this time with Correlated MCDC. 
            
            \item Take either (a) your Restricted MCDC test requirements or (b) your
            Correlated MCDC test requirements. Use these to construct a JUnit test
            suite but run against the original \triangleclass class. Note anything
            unusual about the results of your test suite.
            
        \end{enumerate}
        
\end{enumerate}

\end{document}