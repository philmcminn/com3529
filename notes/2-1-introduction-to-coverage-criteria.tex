\input{common/latex-teach-common/page-setup}
\input{common/latex-teach-common/packages}
\input{common/latex-teach-common/macros}

\renewcommand{\title}[2]{
    \documenttitle{Com3529}{Software Testing \& Analysis}{#1}{#2}
}

% URLs
\newcommand{\coderepourl}{\url{https://github.com/philmcminn/com3529-code}\xspace}

% Code snippets
\newcommand{\code}[1]{\mbox{\tt #1}\xspace}

% README file
\newcommand{\readmefile}{{\tt README.md}\xspace}

% Directories
\newcommand{\lecturesdirectory}{\code{lectures}\xspace}
\newcommand{\practicalsdirectory}{\code{practicals}\xspace}

% Packages
\newcommand{\lecturespackage}{\code{uk.ac.shef.com3529.lectures}}
\newcommand{\lecturesexecutionpackage}{\code{uk.ac.shef.com3529.lectures.execution}}
\newcommand{\practicalspackage}{\code{uk.ac.shef.com3529.practicals}}

% Classes
\newcommand{\calendarclass}{\code{Calendar}}
\newcommand{\randomlytesttriangleclass}{\code{RandomlyTestTriangle}}
\newcommand{\signutilsclass}{\code{SignUtils}}
\newcommand{\spacedefenderclass}{\code{SpaceDefender}}
\newcommand{\stringutilsbuggyoneclass}{\code{StringUtilsBuggy1}}
\newcommand{\triangleclass}{\code{Triangle}}
\newcommand{\weekoneclass}{\code{Week1}}
\newcommand{\teststringutilsbuggyoneclass}{\code{TestStringUtilsBuggy1}}
\newcommand{\testweekoneclass}{\code{TestWeek1}}


% Methods
\newcommand{\classifymethod}{\code{classify}}
\newcommand{\daysbetweentwodatesmethod}{\code{daysBetweenTwoDates}}
\newcommand{\instrumentedclassifymethod}{\code{instrumentedClassify}}
\newcommand{\randomlytestclassifymethod}{\code{randomlyTestClassify}}
\newcommand{\signmethod}{\code{sign}}

% for logic tables
\newcommand{\LTTrue}{T}
\newcommand{\LTFalse}{F}


\booltrue{shownotes}


\begin{document}

\title{2.1 An Introduction to Coverage Criteria for Testing}

\section{Recap --- Why Testing is a Hard Problem}

\note{Video}{Talking Head}

Last week, we discussed three key reasons why exhaustive testing of a problem is
generally impossible:

\note{Video}{First Slide}

\begin{enumerate}

    % Video suggestion! get a false beard and stroke it in a difficult
    % decision-making way

    \item {\bf Testing is Undecidable}. Testing is equivalent to the
    {\it The halting problem}. In general, we cannot know if the program will
    terminate with our test inputs.

    % Video suggestion! chuck a load of balls on me or throw marbles everywhere

    \item {\bf Exhaustive Testing is Intractable} for any reasonably-sized
    program, because the size of the input domain is too large or infinite to
    check every input.

    % Video suggestion! dress up as a fortune teller

    \item {\bf Testing involves the Oracle Problem}. As well as executing the
    software with an input, we need to check that the outputs are correct for
    each input. Doing this manually is error-prone in itself, constructing a
    reliable automated oracle is as hard as constructing the program in the
    first place.

\end{enumerate}

Since we cannot test with {\it all} inputs, we have to decide which ones we're
going to use... 

{\bf Coverage criteria} help us to decide how to do exactly that. 

\note{Video}{Title Slide}

\section{What is a Coverage Criterion?}

\note{Video}{Advance slide, and click once to fill the circle}

A coverage criterion takes some representation or aspect of a piece of software
and divides it up into a series of features. The coverage criterion produces a
series of test requirements based that the test suite should have, based on each
feature. 

By ensuring that our test suite fulfils each of these test requirements, we're
ensuring that it has a certain set of properties that we care about. 

\note{Video}{Click once to cover an element}

When a test case in the suite fulfils a given test requirement, that requirement
(or the software feature that the test requirement is associated with) is said
to be {\bf covered}. 

\note{Video}{Talking Head}

A simple, well-known coverage criterion that you may already have encountered is
{\bf Statement Coverage}. Statement Coverage divides a program up statement by
statement, and generates one test requirement for each statement of a program.
The requirement is that at least one test case of the test suite executes that
statement. 

Statement Coverage makes sense as a kind of testing ``baseline'', because it
makes sense to have exercised each line of code in our software at least once
before releasing it. 

As well as ensuring our test suites have a certain set of properties deemed to
be ``good'' for testing, coverage criteria give us rules {\bf as to when stop
testing}. Once we've attempted to cover each requirement of the criterion, we
have all the properties imbued by the criterion in our test suite. For this
reason, coverage criteria are also known as {\it test adequacy criteria}'',
because the criteria dictate when a test set is deemed to ``adequately'' test a
program. 

\note{Video}{Click through to get the measurement}

Typically, {\bf coverage is reported as a percentage} --- {\bf the percentage of
test requirements satisfied by the test suite}. So for Statement Coverage, the
percentage is the number of lines executed at least once by the test suite.

We will discuss, as part of lectures of this module, whether attaining 100\%
coverage is always a good idea. However, even if we wanted to get 100\%
coverage, it may not always actually be possible, due to {\bf infeasible test
requirements}. For example, with Statement Coverage, this could be due to dead
code, i.e., code that is unreachable and can never be executed. 

\note{Video}{Talking Head}

{\bf Does attaining 100\% coverage for a coverage criterion ensure our program
is bug-free? No.} As we said before, the purpose of coverage criteria is to
generate test requirements and measure how many of those requirements we've
actually managed to satisfy. But they still cannot provide guarantees that our
code is bug-free, even after testing. Take for example Statement Coverage. Even
if we have achieved 100\% coverage and have executed all statements, and thereby
{\it all faults too}, remember the RIPR model from the last lecture --- the
effects of those faults may not propagate to the output as observable failures.
To ensure propagation certain inputs may be needed that we cannot guarantee to
have chosen just by executing each program statement. And we'd still need to
design appropriate assertions to catch the failures.

As a result of this, testers have designed different kinds of coverage criteria,
some more advanced than others, in the hope of making it more likely that more
bugs will be caught. This can lead to a hierarchies of coverage criteria, {\bf
where some criterion ``includes'' all the test requirements of another. In this
instance, one coverage criterion is said to ``subsume'' the other}. 

The problem is that more advanced criteria tend to involve more test
requirements and hence more tests. So a balance has to be struck between
thorough testing and the amount of testing that can feasibly be achieved. 

\note{Video}{Back to slides, next slide. Click for each heading}

\section{Coverage: Important Definitions}

Before we move on to consider some of these more advanced criteria, let's define
some terms we've been using more thoroughly:

{\bf Test Requirement:} A test requirement is an aspect of a software artifact
that a test case must exercise or ``cover''. 

Note that a test requirement is {\it not} the same as a test case. A test
requirement is a goal that should be achieved by the final test suite. A test
case satisfies test requirements with concrete inputs to the software, along
with the expected outputs for those inputs. A test case may fulfil many test
requirements, and a test requirement may be fulfilled by more than one test
case. That is to say, there is not necessarily a one to one mapping between test
cases and test requirements.

% Example - statement coverage 
% test requirement: execute statement on line 8
% test case -- inputs to cover line 8
% test case also covers other lines
% other test cases may also cover line 8 to cover other lines.

{\bf Infeasible Test Requirement:} An infeasible test requirement is a test
requirement that is impossible to satisfy.

% Example 

{\bf Coverage Criterion:} A coverage criterion is rule or collection of rules
that produce a set of test requirements for a test suite.

{\bf Coverage Level:} The coverage level for a test suite is the percentage of
test requirements satisfied by that test suite for some coverage criterion.

% could use a real tool like JaCoCo

{\bf Coverage Criteria Subsumption:} A coverage criterion $C_1$ is said to {\it
subsume} some other coverage criterion $C_2$ if the test requirements of $C_1$
and $C_2$ are not identical, but in satisfying all the test requirements of
$C_1$, all the test requirements of $C_2$ will also be satisfied.

This may seem like more of an abstract comment at the moment, but you will see
why it is important later in the course. 

\note{Video}{Advance slide}

\section{``White-Box'', ``Black-Box'', and ``Grey-Box'' Coverage Criteria}

When talking about coverage criteria, testers often talk about {\bf white-box}
and {\bf black-box} testing to distinguish whether test requirements are derived
from the code or make no reference to the way that it has been programmed or its
structure. 

If we cannot ``see into'' the software's internals, we're treating it as a
``black box'', hence {\bf Black-Box Coverage Criteria}. Black-Box criteria are
directly based off requirements, or designs or abstract models of the software,
such as finite state machines. 

The opposite of treating the software as a black box is to peer inside it and
look at its internals --- that is, the code. Hence {\bf White-Box Coverage Criteria}
derive test requirements from the code itself. White-box coverage criteria are
often more simply known as {\bf Code Coverage Criteria}. Statement Coverage is
an example of a white-box (code) coverage criterion. 

\note{Video}{Click to reveal grey-box}

Furthermore, {\bf Grey-Box Coverage Criteria} are those that make use of a mix of
artefacts considered ``white box'' and ``black box''. 

In the remainder of this lecture, I will introduce the most common white-box
coverage criteria. Next week's lecture will look at testing based on Input
Domain Analysis, which makes use of both knowledge about the code and its
requirements, and is hence grey-box.

Later in the course, Rob will introduce some black-box methods, for example those
based on state machines.

\end{document}
