\input{common/latex-teach-common/page-setup}
\input{common/latex-teach-common/packages}
\input{common/latex-teach-common/macros}

\renewcommand{\title}[2]{
    \documenttitle{Com3529}{Software Testing \& Analysis}{#1}{#2}
}

% URLs
\newcommand{\coderepourl}{\url{https://github.com/philmcminn/com3529-code}\xspace}

% Code snippets
\newcommand{\code}[1]{\mbox{\tt #1}\xspace}

% README file
\newcommand{\readmefile}{{\tt README.md}\xspace}

% Directories
\newcommand{\lecturesdirectory}{\code{lectures}\xspace}
\newcommand{\practicalsdirectory}{\code{practicals}\xspace}

% Packages
\newcommand{\lecturespackage}{\code{uk.ac.shef.com3529.lectures}}
\newcommand{\lecturesexecutionpackage}{\code{uk.ac.shef.com3529.lectures.execution}}
\newcommand{\practicalspackage}{\code{uk.ac.shef.com3529.practicals}}

% Classes
\newcommand{\calendarclass}{\code{Calendar}}
\newcommand{\randomlytesttriangleclass}{\code{RandomlyTestTriangle}}
\newcommand{\signutilsclass}{\code{SignUtils}}
\newcommand{\spacedefenderclass}{\code{SpaceDefender}}
\newcommand{\stringutilsbuggyoneclass}{\code{StringUtilsBuggy1}}
\newcommand{\triangleclass}{\code{Triangle}}
\newcommand{\weekoneclass}{\code{Week1}}
\newcommand{\teststringutilsbuggyoneclass}{\code{TestStringUtilsBuggy1}}
\newcommand{\testweekoneclass}{\code{TestWeek1}}


% Methods
\newcommand{\classifymethod}{\code{classify}}
\newcommand{\daysbetweentwodatesmethod}{\code{daysBetweenTwoDates}}
\newcommand{\instrumentedclassifymethod}{\code{instrumentedClassify}}
\newcommand{\randomlytestclassifymethod}{\code{randomlyTestClassify}}
\newcommand{\signmethod}{\code{sign}}

% for logic tables
\newcommand{\LTTrue}{T}
\newcommand{\LTFalse}{F}


\booltrue{shownotes}


\begin{document}

\title{Automatic Test Case Generation}{4.2 Search-Based Testing}

\section{Introduction}

One disadvantage of Random Testing, or Fuzzing, is that if certain sets of
inputs needed for a test case, or to violate some property, occupy a very small
fraction of the overall input domain, they won't be found very quickly. And if
the overall input domain is very large, or even infinite, they may not be found
at all.

What if we could somehow {\it guide} the process to the right areas of the input
domain? Well, this is what Search-Based Testing aims to do. Search-Based Testing
treats the input domain of a program as a {\it search space}. Viewed through
this lens, some testing problems are effectively ``needle in a haystack'' search
problems. Without some hints as to where the needle is, we will never find it.

\section{Fitness Functions}
In order to provide guidance to the required test case, Search-Based Testing
needs a {\it fitness function}. The fitness function takes an input to the
program and returns a {fitness value} indicating ``how good'' the input is to
the one that is actually needed. If the fitness function is being {\it
minimised}, the close the value to zero, the better the input is. On the other
hand, if the fitness function is being maximised, the higher the value
(potentially with some ceiling), the better the input is.
%
Note that the fitness function doesn't need to know {\it what} the required
input is (that would be tantamount to having solved the problem already!), just
how exactly some candidate input measures up to the one that we want.

It's easier to explain with an example. Let's return to the classify triangle
problem again. Recall that Random Testing could not easily find inputs where the
three sides are equal, i.e. {\tt side1 == side2} and {\tt side2 == side3}.

Essentially with this particular test case generation problem, Random Testing
does not ``know'' if it's getting close to the input needed to trigger the
problematic branch or not. For example, it may get equal numbers for two of the
sides, but narrowly miss out on the third. This input would be judged in exactly
the same way as one where the three sides were very far apart in value.
Search-Based Testing, on the other hand, seeks to retain ``close'' inputs and
develop them further to try to seek to execute the goal of the test case. The
fitness function it would seek to minimise in this case would be:

\begin{center}
$\mathit{fitness} = |$ {\tt side1} $-$ {\tt side2} $| + |$ {\tt side2} $-$ {\tt
side3} $|$
\end{center}

That is, absolute difference of {\tt side1} compared with {\tt side2} added to
the same calculation with {\tt side2} and {\tt side3}.

Notice how the result of this equation is closer to zero the closer the values
of the three sides are to being the same as one another, as shown by the
examples in the following table:

\begin{center}
    \begin{tabular}{rrrr}
        \toprule 
        {\tt side1} & {\tt side2} & {\tt side3} & $\mathit{fintess}$ \\
        \midrule
        10 & 20 & 30 & 20 \\
        10 & 15 & 20 & 10 \\
        10 & 12 & 14 & 4  \\
        10 & 11 & 12 & 2  \\
        10 & 10 & 10 & 0  \\
        \bottomrule
    \end{tabular}
\end{center}

We can see this visually for making {\tt side1} and {\tt side2} equal (i.e., the
first addend of the equation) by plotting the fitness values:

We call the surface of fitness values like this the fitness ``landscape''. 

Compare this to what Random Testing ``sees'':

Random Testing is not guided by any information, so every input is treated the
same, unless it executes the branch in question. The plot clearly portrays the
``needle in a haystack'' nature of this individual test case generation problem.

\section{Search Techniques}

It's all very well and good having guidance, but we need a method of exploiting
it. That's where the ``search'' part of Search-Based Testing comes in.
Search-Based Testing uses the fitness function in combination with an
optimisation algorithm. 

One of the simplest optimisation algorithms is a search technique called {\it
Gradient Descent}. Gradient Descent aims to travel down the gradient of the
fitness landscape to the nearest minimum. It works as follows:

\begin{enumerate}
    \item Pick a point $p$ at random in the search space (the input domain), and
    evaluate its fitness, $\mathit{fitness}(p)$.

    \item Evaluate all the neighbouring points in the fitness landscape, $p_1
    \dots p_n$.

    \item If $\mathit{fitness}(p_i), 1 < i < n$ is less than than
    $\mathit{fitness}(p)$, set $p$ to $p_i$.

    \item Jump to step 2 and repeat until none of $p_1 \dots p_n$ offer an
    improved fitness over $p$.
\end{enumerate}

Gradient Descent (often called {\it Hill Climbing} if the fitness function is to
be maximised) is often described as a ``local'' search algorithm, because as
each point, it evaluates all of its neighbours for a position of improved
fitness in the landscape, until it cannot improve fitness any more. That is, it
has reached a local minima in the fitness landscape. If this local minima
represents a fitness value of zero, the test requirement will have been
satisfied (i.e., the problematic branch is executed with the input in our
triangle example). If not, the search cannot make any more progress from this
position, so it is {\it restarted} from a different random position in the
search space, in the hope of ending up in local minimum that does satisfy the
test requirement. This process continues until the test requirement is satisfied
or some resource limit (e.g., number of iterations or test data evaluations) is
exhausted. The search will always fail, for example, if the test requirement is
infeasible. 

\subsection{The Alternating Variable Method (AVM)}

The Alternating Variable Method (AVM) is an advanced Gradient Descent algorithm
designed for generating numerical inputs for test cases quickly. The AVM is
faster than normal Gradient Descent, because, when given the opportunity, it
accelerates down a gradient rather than inching down it, step-by-step. 

The AVM alternates between small ``exploratory'' moves in the search space to
establish the direction of the gradient, and then makes ``pattern'' moves of
increasing step size in the direction of fitness improvement. 

It does this by starting with a random set of inputs, as with Gradient Descent.
It then considers each input variable in turn, increasing it by a small amount
and decreasing it (exploratory moves) --- e.g., +1 and -1 for an input of
integer type. If one of the two moves leads to an improvement in fitness, a
larger increase/decrease (a pattern move) is made (e.g., +2 or -2). If the
fitness continues to improve, the step size is continually doubled until fitness
gets worse. At this point, further exploratory moves are made to re-establish a
new ``direction'' of improved fitness. This exploratory/pattern move cycle
continues until both exploratory moves result in a non-improvement in fitness.
The AVM then moves onto the next variable and repeats the same steps. This
continues until the AVM has made a complete sweep of the input variables and
there are no exploratory improvements --- or, inputs have been found that
satisfy the test requirement.

\subsection{Evolutionary Algorithms}

Local search only samples one point in the search space (input space) at a time.
Evolutionary Algorithms are so-called ``population''-based approaches that
sample multiple points in the search space at once. In Evolutionary
Algorithm-speak, each point in the search space (i.e., an input to the program,
in this problem domain) is an ``individual''. The idea is that the good parts of
one individual can be combined with another to reach a solution to the problem. 

For example, suppose we have a branch in our program that is executed when a
inputted string contains digits only. However, strings generated purely at
random may contain any character. Strings with one or two digits initially
receive a good fitness value compared to others in the population. Evolutionary
Algorithms emphasis a search operator called ``recombination'', which involves
combining the parts of one individual with another to make an ``offspring''
individual. So a string with digits early in the string may be recombined with a
string with digits later in the string, to produce a new string with digits at
the start and the end, with an even better fitness value. 

% diagram

Evolutionary Algorithms also employ another search operator called ``mutation''
that makes random changes to individuals. So for a test case, this may mean
changing one the inputs to a completely new value, or changing it by some delta
value. 


% A complete EA

\section{More on Fitness Functions}

For testing relational predicates beyond equality, researchers have developed
the following series of fitness functions. The term $K$ represents a small,
positive value.

% table

One problem with using fitness function based on the predicates in branches
only is that the branch needs to be actually {\it reached} in the code for the fitness
to be calculated. Otherwise, inputs will get the same, poor fitness value. If
the branch is hard to reach (e.g., because some other branch in the program
needs to be taken that is only executed by only a few inputs), the search does
not get any guidance and behaves more like Random Testing.

To get around this problem, the fitness function needs to care about all of the
branch predicates that need to be executed leading up to the ``target'' branch,
not just the target itself. To figure out which branches are important, we need
to go back to control flow graphs (CFGs). A branching node is a node of CFG with
two branches (one true and one false). An edge or node of a CFG is said to be
control-dependent on a branching node in the CFG if that branching node needs to
be executed in a certain way (i.e., either true or false) for the edge/node to
be reached in the program. Take, for instance, the following example:

In well-structured programs, control dependencies are obvious from its nesting
structure. However, when programs deploy language constructs like {\tt break},
{\tt goto} or multiple {\tt return} statements, control dependencies are not as
easy to figure out, and more complex algorithms are needed. This will be
covered in a later lecture, on the subject of {\it Program Slicing}. For now,
however, we just need to understand the general concept. 

Using the concept of control dependency, we can now assign a fitness score to
inputs based on how close they were to reaching the target branch. This fitness
score is called the {\it approach level}. For example:

% example

At each approach level, we can add the fitness for the predicate. We refer to
this part as the {\it branch distance}. The branch distance is normalised
(i.e., set to a value between 0 and 1) so that it never exceeds the approach
level component of the fitness function for that particular branching statement.


% fitness function and program 


\section{Other Applications of Search-Based Testing and Optimisation Algorithms
in Testing}

As well as being applied to automatically obtain structural coverage,
Search-Based Testing has been applied to other types of test requirement.

\subsection{Real-Time Systems}

One such application area is that of real-time systems, such as controller
software in embedded systems, such as those installed into cars. Often the
actions performed by these controllers are time-sensitive --- they have to
complete by a certain time period, else there are disastrous consequences. Take
for example an airbag controller. If the software fails to recognise the
conditions in which an airbag should be released, and do this in time, the
driver is placed at risk of a serious injury. 

When testing time limits, engineers are often concerned with finding its
worst-case execution time (WCET). That is, what conditions lead to the software
taking the {\it longest} period of time to execute. If the WCET exceeds an
acceptable threshold, then its back to the drawing board...

Yet, finding the WCET of a piece of software is not a straightforward task, when
you consider that it's not just about the steps programmed into the software
itself, but also the hardware it is run on, including the caching and pipelining
behaviour of the processor. 

Search-Based Testing is an ideal technique for this problem, however. Instead of
the fitness function rewarding inputs on the basis of how close they come to
executing branches, it instead rewards inputs on the basis of the time they make
the software run for. In doing this, the search can navigate a potentially large
input space to find inputs that produce some long execution times in scenarios
that engineers may not have thought of or conceived could happen.

\subsection{Functional Safety}

Search-Based Testing has also been applied to testing functional properties
about software (similar to property-based testing, but on a system level). It
has been used to simulate automated parking controllers. 

Some of these plots show simulated car positions and parking spaces, produced
from a simulated parking controller test at the car company Daimler. The fitness
function simply measured and rewarded test cases with the smallest distance to a
collision during as simulated ``park''. The first generation of the Evolutionary
Algorithm, the randomly generated parking scenarios were easy for the controller
to handle. However, as the Evolutionary Algorithm progressed, it was able to
develop trickier scenarios that eventually produced a collision, i.e., a test
failure.

% include pictures used in previous slides

\subsection{Beyond Test Case Generation}

Optimisation algorithms such as Gradient Descent and Evolutionary Algorithms
have been applied to more in software testing than just test case generation.
They have also been applied to two problems known as ``Test Suite Minimisation''
and ``Test Suite Prioritisation''. Both problems stem from very large test
suites, with too many test cases to run, because doing so would take an
inordinate amount of time. The question is, which test cases to actually
execute. Optimisation algorithms can be used to produce smaller test suites from
the original whole, or be used to produce a running order of test cases whereby
the most important are run first. These topics are the subject of a later
lecture in this module!

\end{document}