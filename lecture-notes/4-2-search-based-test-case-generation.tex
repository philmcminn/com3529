\input{common/latex-teach-common/page-setup}
\input{common/latex-teach-common/packages}
\input{common/latex-teach-common/macros}

\renewcommand{\title}[2]{
    \documenttitle{Com3529}{Software Testing \& Analysis}{#1}{#2}
}

% URLs
\newcommand{\coderepourl}{\url{https://github.com/philmcminn/com3529-code}\xspace}

% Code snippets
\newcommand{\code}[1]{\mbox{\tt #1}\xspace}

% README file
\newcommand{\readmefile}{{\tt README.md}\xspace}

% Directories
\newcommand{\lecturesdirectory}{\code{lectures}\xspace}
\newcommand{\practicalsdirectory}{\code{practicals}\xspace}

% Packages
\newcommand{\lecturespackage}{\code{uk.ac.shef.com3529.lectures}}
\newcommand{\lecturesexecutionpackage}{\code{uk.ac.shef.com3529.lectures.execution}}
\newcommand{\practicalspackage}{\code{uk.ac.shef.com3529.practicals}}

% Classes
\newcommand{\calendarclass}{\code{Calendar}}
\newcommand{\randomlytesttriangleclass}{\code{RandomlyTestTriangle}}
\newcommand{\signutilsclass}{\code{SignUtils}}
\newcommand{\spacedefenderclass}{\code{SpaceDefender}}
\newcommand{\stringutilsbuggyoneclass}{\code{StringUtilsBuggy1}}
\newcommand{\triangleclass}{\code{Triangle}}
\newcommand{\weekoneclass}{\code{Week1}}
\newcommand{\teststringutilsbuggyoneclass}{\code{TestStringUtilsBuggy1}}
\newcommand{\testweekoneclass}{\code{TestWeek1}}


% Methods
\newcommand{\classifymethod}{\code{classify}}
\newcommand{\daysbetweentwodatesmethod}{\code{daysBetweenTwoDates}}
\newcommand{\instrumentedclassifymethod}{\code{instrumentedClassify}}
\newcommand{\randomlytestclassifymethod}{\code{randomlyTestClassify}}
\newcommand{\signmethod}{\code{sign}}

% for logic tables
\newcommand{\LTTrue}{T}
\newcommand{\LTFalse}{F}


\booltrue{shownotes}


\begin{document}

\title{4.2 Search-Based Test Case Generation}

\section{Introduction}

One disadvantage of Random Testing, or Fuzzing, is that if certain sets of
inputs needed for a test case, or to violate some property, occupy a very small
fraction of the overall input domain, they won't be found very quickly. And if
the overall input domain is very large, or even infinite, they may not be found
at all.

What if we could somehow {\it guide} the process to the right areas of the input
domain? Well, this is what Search-Based Testing aims to do. Search-Based Testing
treats the input domain of a program as a {\it search space}. Viewed through
this lens, some testing problems are effectively ``needle in a haystack'' search
problems. Without some hints as to where the needle is, we will never find it.

\section{Fitness Functions}
In order to provide guidance to the required test case, Search-Based Testing
needs a {\it fitness function}. The fitness function takes an input to the
program and returns a {fitness value} indicating ``how good'' the input is to
the one that is actually needed. If the fitness function is being {\it
minimised}, the close the value to zero, the better the input is. On the other
hand, if the fitness function is being maximised, the higher the value
(potentially with some ceiling), the better the input is.
%
Note that the fitness function doesn't need to know {\it what} the required
input is (that would be tantamount to having solved the problem already!), just
how exactly some candidate input measures up to the one that we want.

It's easier to explain with an example. Let's return to the classify triangle
problem again. Recall that Random Testing could not easily find inputs where the
three sides are equal, i.e. {\tt side1 == side2} and {\tt side2 == side3}.

Essentially with this particular test case generation problem, Random Testing
does not ``know'' if it's getting close to the input needed to trigger the
problematic branch or not. For example, it may get equal numbers for two of the
sides, but narrowly miss out on the third. This input would be judged in exactly
the same way as one where the three sides were very far apart in value.
Search-Based Testing, on the other hand, seeks to retain ``close'' inputs and
develop them further to try to seek to execute the goal of the test case. 
The fitness function it would seek to minimise in this case would be:

\begin{center}
$\mathit{fitness} = |$ {\tt side1} $-$ {\tt side2} $| + |$ {\tt side2} $-$ {\tt side3} $|$
\end{center}

That is, absolute difference of {\tt side1} compared with {\tt side2} added to
the same calculation with {\tt side2} and {\tt side3}.

Notice how the result of this equation is closer to zero the closer the values
of the three sides are to being the same as one another, as shown by the
examples in the following table:

\begin{center}
    \begin{tabular}{rrrr}
        \toprule 
        {\tt side1} & {\tt side2} & {\tt side3} & $\mathit{fintess}$ \\
        \midrule
        10 & 20 & 30 & 20 \\
        10 & 15 & 20 & 10 \\
        10 & 12 & 14 & 4  \\
        10 & 11 & 12 & 2  \\
        10 & 10 & 10 & 0  \\
        \bottomrule
    \end{tabular}
\end{center}

We can see this visually for making {\tt side1} and {\tt side2} equal (i.e., the
first addend of the equation) by plotting the fitness values:

We call the surface of fitness values like this the fitness ``landscape''. 

Compare this to what Random Testing ``sees'':

Random Testing is not guided by any information, so every input is treated the
same, unless it executes the branch in question. The plot clearly portrays the 
``needle in a haystack'' nature of this individual test case generation problem.

\section{Search Techniques}

It's all very well and good having guidance, but we need a method of exploiting
it. That's where the ``search'' part of Search-Based Testing comes in.
Search-Based Testing uses the fitness function in combination with an
optimisation algorithm. 

One of the simplest optimisation algorithms is a search technique called {\it
Gradient Descent}. Gradient Descent aims to travel down the gradient of the
fitness landscape to the nearest minimum. It works as follows:

\begin{enumerate}

\item Pick a point $p$ at random in the search space (the input domain), and evaluate
its fitness, $\mathit{fitness}(p)$.
\item Evaluate all the neighbouring points in the fitness landscape, $p_1 \dots p_n$.
\item If $\mathit{fitness}(p_i), 1 < i < n$ is less than than
$\mathit{fitness}(p)$, set $p$ to $p_i$.
\item Jump to step 2 and repeat until none of $p_1 \dots p_n$ offer an improved
fitness over $p$.

\end{enumerate}

Gradient Descent (often called {\it Hill Climbing} if the fitness function is to
be maximised) is often described as a ``local'' search algorithm, because as
each point, it evaluates all of its neighbours for a position of improved
fitness in the landscape, until it cannot improve fitness any more. That is, it
has reached a local minima in the fitness landscape. If this local minima
represents a fitness value of zero, the test requirement will have been
satisfied (i.e., the problematic branch is executed with the input in our
triangle example). If not, the search cannot make any more progress from this
position, so it is {\it restarted} from a different random position in the
search space, in the hope of ending up in local minimum that does satisfy the
test requirement. This process continues until the test requirement is satisfied
or some resource limit (e.g., number of iterations or test data evaluations) is
exhausted. The search will always fail, for example, if the test requirement is
infeasible. 

\subsection{The Alternating Variable Method (AVM)}


\subsection{Evolutionary Algorithms}



\section{More on Fitness Functions}



\section{Other Applications of Search-Based Testing and Optimisation Algorithms in Testing}

% real time, functional safety

% test suite prioritisation and minimisation


\end{document}