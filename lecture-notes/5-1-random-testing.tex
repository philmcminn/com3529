\input{common/latex-teach-common/page-setup}
\input{common/latex-teach-common/packages}
\input{common/latex-teach-common/macros}

\renewcommand{\title}[2]{
    \documenttitle{Com3529}{Software Testing \& Analysis}{#1}{#2}
}

% URLs
\newcommand{\coderepourl}{\url{https://github.com/philmcminn/com3529-code}\xspace}

% Code snippets
\newcommand{\code}[1]{\mbox{\tt #1}\xspace}

% README file
\newcommand{\readmefile}{{\tt README.md}\xspace}

% Directories
\newcommand{\lecturesdirectory}{\code{lectures}\xspace}
\newcommand{\practicalsdirectory}{\code{practicals}\xspace}

% Packages
\newcommand{\lecturespackage}{\code{uk.ac.shef.com3529.lectures}}
\newcommand{\lecturesexecutionpackage}{\code{uk.ac.shef.com3529.lectures.execution}}
\newcommand{\practicalspackage}{\code{uk.ac.shef.com3529.practicals}}

% Classes
\newcommand{\calendarclass}{\code{Calendar}}
\newcommand{\randomlytesttriangleclass}{\code{RandomlyTestTriangle}}
\newcommand{\signutilsclass}{\code{SignUtils}}
\newcommand{\spacedefenderclass}{\code{SpaceDefender}}
\newcommand{\stringutilsbuggyoneclass}{\code{StringUtilsBuggy1}}
\newcommand{\triangleclass}{\code{Triangle}}
\newcommand{\weekoneclass}{\code{Week1}}
\newcommand{\teststringutilsbuggyoneclass}{\code{TestStringUtilsBuggy1}}
\newcommand{\testweekoneclass}{\code{TestWeek1}}


% Methods
\newcommand{\classifymethod}{\code{classify}}
\newcommand{\daysbetweentwodatesmethod}{\code{daysBetweenTwoDates}}
\newcommand{\instrumentedclassifymethod}{\code{instrumentedClassify}}
\newcommand{\randomlytestclassifymethod}{\code{randomlyTestClassify}}
\newcommand{\signmethod}{\code{sign}}

% for logic tables
\newcommand{\LTTrue}{T}
\newcommand{\LTFalse}{F}


\booltrue{shownotes}


\begin{document}

\title{Automatic Test Case Generation}{4.1 Random Testing}

\section{Introduction to Automatic Test Case Generation}

So far we've learnt a lot about how to design and write test cases, manually, using
various coverage criteria as a basis for doing. We also know about various tools
that will allow us to automatically run those tests and check the results of
those tests for us (like JUnit). 
% I'm wondering about a week 2 lecture on tools like JUnit, Cucumber etc??

But wouldn't it be great if we could automate the design and writing of those
test cases as well? Well, thanks to the efforts of researchers over the last
10--20 years it turns out that there are a lot of techniques and tools that will
allow us to actually do this. We're going to consider three main methods of
automated test case generation --- Random, Search-Based, and Symbolic Execution.
We'll start with arguably the simplest of these -- Random Test Case Generation.


\section{Random Testing}

Random Test Case Generation (or just {\it Random Testing} for short), is a very
simple and easy to automate technique that involves using random number
generators to generate inputs to programs. Those programs don't need to accept
numbers as inputs only --- given some rules, random numbers can be used to
generate strings and more complex inputs like XML files too.

You may think that this is too simple an idea to actually work, but actually
it's surprisingly effective. 

% Randomly Testing the Triangle Class

Let's return to the \classifymethod~method of the \triangleclass~class in the
\lecturespackage~package. We can hook this up to a method that randomly
generates the three integer inputs and track which branches are covered. 

I've done this in the \randomlytesttriangleclass~class in the
\lecturesexecutionpackage. The class contains a method
\instrumentedclassifymethod (which is the last method of the class). It's the
same as the original \classifymethod~method in terms of how it classifies
triangles based on their sides, but it also contains some additional code that
logs when different branches are taken in the code. This is so we can calculate
how much branch coverage we have obtained by executing the method. Each branch
has its own integer ID. When that branch is taken in the code, the ID is added
to a set, {\tt coveredBranches}. We can then investigate the content of the set
after executing the instrumented method a few times to see what Branch Coverage
we got. In practice, an automated test case generation tool would automate this
step for us.

The \randomlytestclassifymethod~method calls the instrumented method with three
randomly generated integers a certain number of times.

If we run the class to execute the \classifymethod~method 100 times (this is a
controllable parameter set bu changing the {\tt ITERATIONS} constant) and look
at the output, we can see that 10 of the 14 branches are covered (i.e.,
approximately 70\% of the branches) are covered. These branches are usually
covered in the first 10 executions. 

Even though the inputs are generated randomly, this pattern tends to hold most
of the time. The branches that are not covered are branches 9, 11, and 13. If we
inspect the code more closely, we can see that these branches are when two or
more of the side lengths are equal. Now, we're using the entire range of the
{\tt int} type in Java, so the chance of generating two numbers that are the
same are 1 in $2^{32}$, which is not a likely occurrence in 50 repeat
executions. We'd have to perform Random Testing for a long time to hit those
branches. The alternative is to narrow the domains of each of the three integers
representing each side. We can do this using the {\tt MIN\_INT} and {\tt
MAX\_INT} constants. If we set them to be -10 and 10, respectively, there is a
much greater chance of generating two numbers that are the same (1 in 21, to be
precise). When we run it now, more of the branches are covered, but some are
still stubborn --- branch 10 in particular. Branch 10 requires all three numbers
to be equal (with a 1 in 441 chance), so we either need to run it for more
iterations or reduce the input domains of the three integer inputs further. If
we increase the number of iterations to 10,000, branch 10 is covered and we have
obtained full branch coverage. 


\section{Automatic Test Case Generation and the Oracle Problem}

Note that although we can automatically generated test cases according to a
coverage criterion, so long that criterion is a well-defined one, like Branch
Coverage --- because applying Input Domain Analysis requires some modelling that is
difficult to automate! --- we still have the Oracle Problem to contend with.
Remember the Oracle Problem from week 1? Essentially, we still need to be able
to determine that the outputs are the right ones for the inputs. Essentially, we
need to select one of the automatically generated test cases for each branch,
and check whether the triangle type returned by the method is the right one
given the inputs. If our automated test case generation tool spits out a JUnit
test suite instead of just raw output, this means checking the JUnit test cases
generated are correct. As you can imagine, if the automated test case generator
generates a lot of test cases this could get quite tedious quite quickly.

\section{Randoop}

Some tools, therefore, generate test cases to violate specific properties of the
software. Randoop, for example, is a tool that has two modes ...

\section{Property-Based Testing}

Property-based testing leverages automatic input generation to test specific
properties of the software specified by the tester. Property-based testing is
not too dissimilar from writing test cases in JUnit, except instead of providing
inputs and assertions over outputs, the tester provides generic {\it properties}
instead, that should always hold for the software under test. For example, when
testing a method that adds a sequence of numbers, a possible property could be
that when each of the numbers is positive, the result is positive. Or, when
adding a positive number to a sequence of numbers, the sum should be greater
than the sum of the original sequence. And so on.

Property-based testing first appeared in a tool called {\it QuickCheck} for the
Haskell programming language. Since then, QuickCheck has been re-implemented for
many different languages. Java has a version --- one of the latest incarnations
is {\it jqwik} (\url{https://jqwik.net/}). For Python, a very effective version of
QuickCheck is available called {\it Hypothesis} (\url{hypothesis.readthedocs.io}). 

Property-based testing tools attempt to generate inputs randomly until they
break the specified properties, or they run out of iterations. When they break a
property, they often simplify the input for presentation to the tester, so that
the tester can easily understand the failure or the problem with their original
property (which itself may have been incorrectly specified).

\section{Fuzzing}

Random Testing goes by another name you may be familiar with, particularly in
the security community --- that is, {\it Fuzzing}. Fuzzing is like
Property-Based Random Testing but with more generic kinds of software properties
in mind --- such as crashing behaviour (thereby demonstrating instability
problems) or generic security issues such as buffer overflows.

According to security lore, Fuzzing was born in a ``dark and stormy night in the
Fall of 1988''. Sitting in his apartment in Wisconsin, Madison, professor Barton
Miller was connected to his university computer via a 1200 baud telephone line.
The thunderstorm caused noise on the line, and this noise in turn caused the
UNIX commands on either end to get bad inputs --- and crash. The frequent crashes
surprised him --- surely, programs should be more robust than that? As a
scientist, he wanted to investigate the extent of the problem and its causes. So
he crafted a programming exercise for his students at the University of
Wisconsin-Madison --- an exercise that would have his students create the first
fuzzers.

The assignment read as follows:

\begin{quote}
    {\it The goal of this project is to evaluate the robustness of various UNIX
    utility programs, given an unpredictable input stream. \dots First, you will
    build a fuzz generator. This is a program that will output a random character
    stream. Second, you will take the fuzz generator and use it to attack as many
    UNIX utilities as possible, with the goal of trying to break them.}
\end{quote}

This assignment captures the essence of Fuzzing. Generate random inputs, and see
if they break things by running it for a long enough period of time.

When Miller and his students ran their first fuzzers in 1989, they found an
alarming result: About a third of the UNIX utilities they fuzzed had issues ---
they crashed, hung, or otherwise failed when confronted with fuzzing input.

Considering that many of these UNIX utilities were used in scripts that would
also process network input, this was an alarming result. Programmers quickly
built and ran their own fuzzers, rushed to fix the reported errors, and learned
not to trust external inputs anymore.

What kind of problems did Miller's fuzzing experiment find?

% summarise Zeller

% cover Heartbleed?


\section{Complex Input Domains}

 %mutation fuzzing





\end{document}