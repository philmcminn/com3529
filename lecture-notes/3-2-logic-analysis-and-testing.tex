\input{common/latex-teach-common/page-setup}
\input{common/latex-teach-common/packages}
\input{common/latex-teach-common/macros}

\renewcommand{\title}[2]{
    \documenttitle{Com3529}{Software Testing \& Analysis}{#1}{#2}
}

% URLs
\newcommand{\coderepourl}{\url{https://github.com/philmcminn/com3529-code}\xspace}

% Code snippets
\newcommand{\code}[1]{\mbox{\tt #1}\xspace}

% README file
\newcommand{\readmefile}{{\tt README.md}\xspace}

% Directories
\newcommand{\lecturesdirectory}{\code{lectures}\xspace}
\newcommand{\practicalsdirectory}{\code{practicals}\xspace}

% Packages
\newcommand{\lecturespackage}{\code{uk.ac.shef.com3529.lectures}}
\newcommand{\lecturesexecutionpackage}{\code{uk.ac.shef.com3529.lectures.execution}}
\newcommand{\practicalspackage}{\code{uk.ac.shef.com3529.practicals}}

% Classes
\newcommand{\calendarclass}{\code{Calendar}}
\newcommand{\randomlytesttriangleclass}{\code{RandomlyTestTriangle}}
\newcommand{\signutilsclass}{\code{SignUtils}}
\newcommand{\spacedefenderclass}{\code{SpaceDefender}}
\newcommand{\stringutilsbuggyoneclass}{\code{StringUtilsBuggy1}}
\newcommand{\triangleclass}{\code{Triangle}}
\newcommand{\weekoneclass}{\code{Week1}}
\newcommand{\teststringutilsbuggyoneclass}{\code{TestStringUtilsBuggy1}}
\newcommand{\testweekoneclass}{\code{TestWeek1}}


% Methods
\newcommand{\classifymethod}{\code{classify}}
\newcommand{\daysbetweentwodatesmethod}{\code{daysBetweenTwoDates}}
\newcommand{\instrumentedclassifymethod}{\code{instrumentedClassify}}
\newcommand{\randomlytestclassifymethod}{\code{randomlyTestClassify}}
\newcommand{\signmethod}{\code{sign}}

% for logic tables
\newcommand{\LTTrue}{T}
\newcommand{\LTFalse}{F}


\booltrue{shownotes}


\begin{document}

\title{Coverage Criteria}{2.3 White-Box Coverage Criteria based on\\ Logic Analysis}

\section{The Motivation Behind Logic Coverage Criteria}

\note{Video}{Talking head}

Branch Coverage ensures that the predicates in each decision in a program are
exercised as true and false as part of a test suite. 

White-box {\bf logic coverage criteria} aim to test the logic in branch
predicates more finely.

\note{Video}{Title slide}

\note{Video}{Slide change}

Let's return to the \daysbetweentwodatesmethod~method from lecture in week 1.
The method takes six parameters representing two dates ({\tt year1},
{\tt month1}, {\tt day1} and {\tt year2}, {\tt month2}, {\tt day2},
respectively).

One of the initial tasks of the method is to ensure that the first date ({\tt
year1}, {\tt month1}, {\tt day1}) is before the second ({\tt year2}, {\tt
month2}, {\tt day2}) so that the later calculation of the days between the two
dates will function correctly. This takes place on lines 23--36 of the method,
shown here:

\begin{center} 
    \scalebox{0.8}{
    \begin{tabular}{rl} 
        \toprule
        Line \\
        \midrule
           & \dots \\
 5 & \verb$public static int daysBetweenTwoDates(int year1, int month1, int day1,$\\
 6 & \verb$                                      int year2, int month2, int day2) {$\\
   & \dots \\
24 & \verb$if ((year2 < year1) ||$ \\
25 & \verb$        (year2 == year1 && month2 < month1) ||$ \\
26 & \verb$        (year2 == year1 && month2 == month1 && day2 < day1)) {$ \\
27 & \verb$    int t = month2;$ \\
28 & \verb$    month2 = month1;$ \\
29 & \verb$    month1 = t;$ \\
30 & \verb$    t = day2;$ \\
31 & \verb$    day2 = day1;$ \\
32 & \verb$    day1 = t;$ \\
33 & \verb$    t = year2;$ \\
34 & \verb$    year2 = year1;$ \\
35 & \verb$    year1 = t;$ \\
36 & \verb$}$ \\
   & \dots \\
        \bottomrule
    \end{tabular}}
\end{center}    

In particular, let's focus in on the branch predicate, which is complex and
spans no fewer than three lines! Each line corresponds to a disjunct of the
overall predicate.
%
(Recall that a disjunct is a boolean expression that forms the operand of an
``or'' operator in a predicate, where the disjunct  cannot be further broken
down into simpler disjuncts. For example, $D_1$ and $D_2$ in $D_1 \vee D_2$.)

\note{Video}{Slide change}

In this particular branch predicate of \daysbetweentwodatesmethod,
each disjunct encodes a possible way in which the second supplied date could be
before the first: 

\begin{center}
\begin{tabular}{ll}
    \toprule
    \multicolumn{2}{l}{\bf Disjunct} \\
    \midrule
    1 & {\tt year2 < year1}                                         \\
    2 & {\tt year2 == year1 \&\& month2 < month1}                   \\
    3 & {\tt year2 == year1 \&\& month2 == month1 \&\& day2 < day1} \\
    \bottomrule
\end{tabular}
\end{center}

That is:

\begin{itemize}
    \item {\bf Disjunct 1.} On the first line of the predicate (line 24 of the
    code), if the year of the second date is before the first.

    \item {\bf Disjunct 2.} On the second line of the predicate (line 25 of the
    code), if the years are the same, but the month of the second date is before
    the first.

    \item {\bf Disjunct 3.} And finally, on the third line of the predicate
    (line 26 of the code), if the years and months of the two dates are the
    same, but the day of the second date is before the day of the first.
\end{itemize}

\note{Video}{Slide change}

Branch Coverage requires that the overall predicate is exercised as
true and false by the test suite, but we can do that by exercising 
Disjunct 1 ({\tt year2 < year 1}) as true and false only:

\begin{center}
\begin{tabular}{r|lll|lll|ccc|c}
    \toprule
    {\bf Test} & \multicolumn{6}{c|}{{\bf Example Input}} & \multicolumn{3}{c|}{{\bf Disjunct}} & {\bf Branch} \\
    {\bf Case} & {\tt year1} & {\tt month1} & {\tt day1} & {\tt year2} & {\tt month2} & {\tt day2} & {\bf 1} & {\bf 2} & {\bf 3} & {\bf Predicate} \\
    \midrule
    1 & 2019 & 12 & 13 & 2018 & 4  & 25 & \LTTrue & \LTFalse & \LTFalse & \LTTrue \\
    2 & 2018 & 4  & 25 & 2019 & 12 & 13 & \LTFalse & \LTFalse & \LTFalse & \LTFalse \\
    \bottomrule
\end{tabular}
\end{center}

It does not ensure that the other disjuncts are independently correct, by
testing them individually as true and false as well. (Suppose we slipped up in
the final clause, for example, and wrote {\tt day2 > day1}, or made an
equivalent mistake, {\tt day1 < day2}.)

\section{Analysing the Logic of a Predicate}

\note{Video}{Slide change}

{\bf Logic coverage criteria} aim to address this type of potential deficiency.
Logic criteria break a predicate down into {\bf conditions} according to their
{\bf logical operators}.

Typical {\bf logical operators} in programming languages are:

\begin{center}
\begin{tabular}{lll}
    \toprule
    {\bf Operator } & {\bf Mathematical} & {\bf Typical Programming} \\
    {\bf Name     } & {\bf Symbol      } & {\bf Language Symbol    }\\
    \midrule
    not & $\neg$   & {\tt !} \\
    and & $\wedge$ & {\tt \&\&} \\
    or  & $\vee$   & {\tt ||} \\
    \bottomrule
\end{tabular}
\end{center}

A {\bf condition} is a boolean expression that is a component of a more complex
predicate in that does not contain any logical operators.

\note{Video}{Advance slide}

How many conditions are there in the {\tt if statement} of
\daysbetweentwodatesmethod? 

\note{Video}{Advance slide}

There are six conditions --- one in the first disjunct, two in the second, and
three in the third. However, there are only five {\it unique} conditions, since
the condition {\tt year2 == year1} appears as part of the second {\it and} the
third disjunct:

\begin{center}
\begin{tabular}{llr}
    \toprule
    \multicolumn{2}{l}{\bf Condition}  & {\bf Disjunct} \\
    \midrule
    1  & {\tt year2 < year1}   & 1 \\
    2= & {\tt year2 == year1}  & 2 \\
    3  & {\tt month2 < month1} & 2 \\
    2= & {\tt year2 == year1}  & 3 \\
    4  & {\tt month2 == month1} & 3 \\
    5  & {\tt day2 < day1}      & 3 \\
    \bottomrule
\end{tabular}
\end{center}

\note{Video}{Advance slide}

\section{Logic Coverage Criteria}

Now that we've learnt how to decompose predicates into conditions, we can apply
various logic coverage criteria. As usual, these start simple and become
relatively more complex!

\subsection{Condition Coverage}

One possible coverage criterion would involve ensuring that each condition of
each predicate of a program is exercised as true and false. This is called {\bf
Condition Coverage}.

You would think that Condition Coverage subsumes Branch Coverage, because we're
testing each condition more thoroughly. But this isn't actually the case.
Consider \daysbetweentwodatesmethod and this test suite, and its
effect on each of the conditions and the branch:

\begin{center}
\begin{tabular}{r|ccccc|c||llllll}
    \toprule
    {\bf Test} & \multicolumn{5}{c|}{{\bf Condition}} & {\bf Branch} & \multicolumn{6}{c}{{\bf Example Input}} \\    
    {\bf Case} & {\bf 1} & {\bf 2} & {\bf 3} & {\bf 4} & {\bf 5} & {\bf Predicate} & {\tt year1} & {\tt month1} & {\tt day1} & {\tt year2} & {\tt month2} & {\tt day2} \\
    \midrule
    1 & \LTTrue  & \LTFalse & \LTFalse & \LTTrue  & \LTTrue  & \LTTrue & 2019 & 1 & 2 & 2018 & 1 & 1 \\
    2 & \LTFalse & \LTTrue  & \LTTrue  & \LTFalse & \LTFalse & \LTTrue & 2019 & 2 & 1 & 2019 & 1 & 1 \\
    \bottomrule
\end{tabular}
\end{center}

\note{Video}{Advance slide to highlight the T's}

Each condition is independently exercised as true and false by the the test
suite, but the branch is only ever exercised as true.

\note{Video}{Advance slide}

\subsection{Multiple Condition Coverage}

{\bf Multiple Condition Coverage} requires that each possible {\it combination}
of truth values for each condition are tested. As such, Multiple Condition
Coverage does subsume Branch Coverage, but at the cost of an exponential number
of test requirements. 
%
Given $n$ conditions in a predicate, Multiple Condition Coverage generates $2^n$
test requirements. 

\note{Video}{Advance slide --- reveal table}

How many test requirements would this mean for the branch of
\daysbetweentwodatesmethod that we've been studying?

\note{Video}{Advance slide --- reveal answer}

With five unique conditions, there would be $2^5 = 32$ combinations --- just for
one branch! 
%
We won't enumerate them all here!

However, note that some of the combinations would be infeasible. So the true
number of feasible test requirements is less than 32. For example, condition 1
({\tt year2 < year1}) cannot be true at the same time as condition 2 ({\tt year2
== year1}), likewise condition 3 ({\tt month2 < month1}) cannot be true while
condition 4 is true ({\tt month2 == month1}). 

\note{Video}{Advance slide}

\subsection{Condition Decision Coverage}

To counter the problem that Condition Coverage does not subsume Branch Coverage,
and that Multiple Condition Coverage results in an unwieldy number of test
requirements, {\bf Condition Decision Coverage}, combines Condition Coverage and
Branch Coverage together into one criterion. The word ``Decision'' in the name
comes from the fact that Branch Coverage is sometimes called Decision Coverage.

As you would expect from such a combination, Condition Decision Coverage
requires exercising each condition as true and false, as well as the overall
branch predicate (but without requiring testing each explicit combination of
truth values and the exponential number of resulting test cases).

Critics of Condition Decision Coverage, however, argue that this is not
sufficient or nearly thorough enough. It would be better, from a testing
perspective, to exercise the {\it effect} of each condition on the outcome of
the overall decision (i.e., the branch predicate). And that's what the next
coverage criterion, {\it Modified Condition Decision Coverage}, aims to achieve.

\note{Video}{Advance slide}

\subsection{Modified Condition Decision Coverage (MCDC)}

\note{Video}{Advance slide during this paragraph to reveal minor/major condition descriptive part of the slide}

{\bf Modified Condition Decision Coverage}, or MCDC for short, works by taking
each individual condition of the branch predicate and treating it as the {\it
major} condition. The other conditions (referred to as the {\it minor}
conditions at this point) are arranged such that if we were to flip the truth
value of the major condition, the overall predicate would also flip in truth
value so that the alternative branch would be taken in the program. In other
words, so that the major condition is responsible for {\it determining} the
predicate. 

There are two main ways to achieve this --- we will call these {\bf Restricted
MCDC} and {\bf Correlated MCDC}. 

Restricted MCDC requires the truth values of the minor conditions to be fixed as
the major condition and the predicate flip in truth value, while Correlated MCDC
relaxes this requirement. So long as the major condition and predicate both flip
together, we will have satisfied Correlated MCDC.

\note{Video}{Advance slide}

\subsubsection*{Restricted MCDC}

We'll start with Restricted MCDC first. Given a major condition, we then need to
choose truth values for the minor conditions so that flipping the outcome of the
major subcondition also flips (i.e., determines) the overall predicate.

So we begin by choosing condition 1 of the
\daysbetweentwodatesmethod example as the major condition. If we
set the other (minor) subconditions to false, then condition 1 now determines
the predicate. When we flip the truth value of condition 1, the branch predicate
also flips, as highlighted here:

\note{Video}{Advance slide to reveal highlight}

\begin{center}
\begin{tabular}{r|ccccc|c||llllll}
    \toprule
    {\bf Test} & \multicolumn{5}{c|}{{\bf Condition}} & {\bf Branch} & \multicolumn{6}{c}{{\bf Example Input}} \\    
    {\bf Case} & {\bf 1} & {\bf 2} & {\bf 3} & {\bf 4} & {\bf 5} & {\bf Predicate} & {\tt year1} & {\tt month1} & {\tt day1} & {\tt year2} & {\tt month2} & {\tt day2} \\
    \midrule
    1 & {\it \LTTrue}  & \LTFalse & \LTFalse & \LTFalse & \LTFalse & {\it \LTTrue}  & 2019 & 1 & 1 & 2018 & 2 & 1 \\
    2 & {\it \LTFalse} & \LTFalse & \LTFalse & \LTFalse & \LTFalse & {\it \LTFalse} & 2018 & 1 & 1 & 2019 & 2 & 1 \\
    \bottomrule
\end{tabular}
\end{center}

As we can see this generates two test requirements, which are fulfilled in the
table by two test cases. We now move onto condition 2 and apply a similar
process:

\note{Video}{Advance slide to reveal highlight}

\begin{center}
\begin{tabular}{r|ccccc|c||llllll}
    \toprule
    {\bf Test} & \multicolumn{5}{c|}{{\bf Condition}} & {\bf Branch} & \multicolumn{6}{c}{{\bf Example Input}} \\    
    {\bf Case} & {\bf 1} & {\bf 2} & {\bf 3} & {\bf 4} & {\bf 5} & {\bf Predicate} & {\tt year1} & {\tt month1} & {\tt day1} & {\tt year2} & {\tt month2} & {\tt day2} \\
    \midrule
    3 & \LTFalse & {\it \LTTrue}  & \LTTrue & \LTFalse & \LTFalse & {\it \LTTrue}  & 2019 & 2 & 1 & 2019 & 1 & 1 \\
    4 & \LTFalse & {\it \LTFalse} & \LTTrue & \LTFalse & \LTFalse & {\it \LTFalse} & 2018 & 2 & 1 & 2019 & 1 & 1 \\
    \bottomrule
\end{tabular}
\end{center}

This time we cannot set all of the other conditions to false --- we need
condition 3 to be true in order for condition 2 to determine the predicate.
%
This generates a further two test requirements. And we continue in this fashion.

One problem with Restricted MCDC is that we tend to encounter infeasibility
issues with it when there are variables that are shared across conditions that
mean that one condition cannot be varied independently of another. 

For this reason, Correlated MCDC, which we discuss in more detail next, is
sometimes preferred. 

\note{Video}{Advance slide}

\subsubsection*{Correlated MCDC}

Correlated MCDC does not care whether the minor conditions are fixed while the
major condition flips, just as long as there is a test case where the major
condition flips and the branch predicate also flips. 

As with Restricted MCDC, each condition therefore generates two test
requirements. However in practice, the set of truth values for each condition
may satisfy multiple test requirements, leading to fewer test cases overall.

In practice, we can arrange this by splitting our test suite into two. The first
subset of test cases is where the branch predicate evaluates to true, and the
second subset of test cases is where the branch predicate evaluates to false.
Now, we just need to ensure that each condition takes on one true/false outcome
in the first subset, and the opposite outcome in the second:

\begin{center}
\begin{tabular}{r|ccccc|c||llllll}
    \toprule
    {\bf Test} & \multicolumn{5}{c|}{{\bf Condition}} & {\bf Branch} & \multicolumn{6}{c}{{\bf Example Input}} \\    
    {\bf Case} & {\bf 1} & {\bf 2} & {\bf 3} & {\bf 4} & {\bf 5} & {\bf Predicate} & {\tt year1} & {\tt month1} & {\tt day1} & {\tt year2} & {\tt month2} & {\tt day2} \\
    \midrule
    1 & \LTTrue  & \LTFalse & \LTFalse & \LTTrue  & \LTTrue  & \LTTrue  & 2019 & 1 & 2 & 2018 & 1 & 1 \\
    2 & \LTFalse & \LTTrue  & \LTTrue  & \LTFalse & \LTFalse & \LTTrue  & 2019 & 2 & 1 & 2019 & 1 & 1 \\
    \hdashline
    3 & \LTFalse & \LTFalse & \LTFalse & \LTFalse & \LTFalse & \LTFalse & 2018 & 1 & 1 & 2019 & 2 & 1 \\
    \bottomrule
\end{tabular}
\end{center}

Note that for test case 3, all conditions are false, and the branch
predicate is also false. For test cases 1 and 2 the branch predicate is
true, and each condition is true in at least one test requirement. 

\note{Video}{Advance slide, advance again on ``condition 2 for ...''}

That is, condition 1 can be regarded as the major condition in test cases 1 and
3; and condition 2 for test cases 2 and 3. (Note that these aren't the only
conditions that could qualify as major in these instances.)

There is less scope to re-use truth values across test requirements and test
cases with Restricted MCDC because the truth values are more tightly controlled.
And so Correlated MCDC tends to involve fewer test cases, and therefore smaller test
suites. However, Restricted MCDC is better at test thoroughness: notice how the
the test suite for Correlated MCDC does not independently test Disjuncts 1 and
3. In fact, due to short circuit evaluation in Java, Disjunct 3 is never even
evaluated by Test Case 1, since the outcome of the entire branch predicate is
known after Disjunct 1.

\note{Video}{Advance slide}

\subsection{Subsumption Hierarchy}

This leads to the following subsumption hierarchy of Logic Coverage Criteria
(including Branch Coverage for comparison purposes).

\slide{2-3-subsumption-hierarchy}

The diagram reflects how Branch Coverage is not subsumed by Condition Coverage,
yet both are subsumed by Condition Decision Coverage, which combines the two.
Condition Decision is then subsumed by MCDC up to Multiple Condition Coverage.

As with Structural Criteria, Logic Criteria go by different names. In this
lecture I've tried to use the most common names --- the diagram lists others
alongside the common one. 

The only place where I've broken with the common name is with the two variants
of MCDC, since there is no common established term that incorporates the acronym
``MCDC'', and yet the term MCDC is incorporated into many software standards, as
we will discuss next.
%
There has historically been a lot of confusion around MCDC. Sometimes MCDC is
referred to mean one of the restricted or correlated variants, but neither one of
them, specifically. This is probably because many testers are not even aware
there is a subtle difference. 
%
Therefore, Restricted and Correlated MCDC can just go by the name MCDC if the
person using the term is not familiar with the different strategies for handling
minor conditions. 

Some alternative names only differ because the individuals who named them chose
to use the word ``clause'' instead of the word ``condition''. The word
``condition' is potentially confusing often used to mean the same thing as a
complete predicate, and not specifically a component of a predicate. However,
I've tried to stick with the most commonly used names for coverage criteria
throughout this course.

\note{Video}{Advance slide}

\subsection{When to Use Logic Coverage Criteria}

So which logic coverage criterion, should we use, and when? 

\note{Video}{Click through the slide on each point to reveal it as a bullet}

In terms of all of the logic coverage criteria, the MCDC variants offer the best
trade-off in terms of number of test requirements and test thoroughness, with
Correlated MCDC resulting in smaller test suites in general. Multiple Condition
Coverage results in too many test requirements, and therefore test cases, while
plain Condition Coverage does not subsume Branch Coverage.

So when should we use MCDC? MCDC is good at testing complicated predicates in
your program code. However, many good programmers will have refactored code
like that which we have encountered in \daysbetweentwodatesmethod,
so that it is easier to understand when reading a program, for example:

\note{Video}{Advance slide}

\begin{center} 
    \scalebox{0.8}{
    \begin{tabular}{l} 
        \toprule
        \dots\\
\verb$boolean yearSame = year2 == year1;$\\
\verb$boolean yearAndMonthSame = yearSame && month2 == month1;$\\
\verb$$\\
\verb$boolean secondDateBeforeFirstByYear = year2 < year1;$\\
\verb$boolean secondDateBeforeFirstByMonth = yearSame && month2 < month1;$\\
\verb$boolean secondDateBeforeFirstByDay = yearAndMonthSame && day2 < day1;$\\
\verb$$\\
\verb$boolean secondDateBefore = secondDateBeforeFirstByYear ||$\\ 
\verb$            secondDateBeforeFirstByMonth || secondDateBeforeFirstByDay;$\\
\verb$$\\
\verb$if (secondDateBeforeFirst) { $\\
\verb$    $ \dots\\
\verb$}$\\
\dots\\
        \bottomrule
    \end{tabular}}
\end{center}  

Here, the original condition has been refactored into several statements that
make assignments to Boolean variables that contribute to a Boolean variable used
in the eventual branch predicate.

Such refactorings make it less immediately obvious as to how to apply MCDC. In
these cases, we would need to do the reverse of the refactoring to reconstruct
the predicate that leads to establishing the {\tt secondDateBeforeFirst} boolean
value.

You may therefore think that complex predicates are not that frequent in
practice. One area where they tend to crop up a lot is in {\it control systems}
or {\it cyber-physical systems} --- i.e., software that controls a physical
device, for example a motor or an engine.

It's for this reason that different flavours of logic coverage criteria (usually
MCDC) are mandated by many standards in order for safety-critical software
systems to be certified --- including road, rail and avionic software. Some of
these standards include, for example:

\note{Video}{Advance slide}

% source:
% https://www.froglogic.com/coco/modified-condition-decision-coverage-mcdc/

\begin{itemize}

    \item ISO 26262, ``Road vehicles --- Functional safety''.\\
    See: \url{https://en.wikipedia.org/wiki/ISO_26262}

    \item EN 50128, a functional safety standard used in the rail industry.\\
    See: \url{https://www.adacore.com/industries/rail/en50128}

    \item DO-178B and DO-178C, ``Software Considerations in Airborne Systems and Equipment
    Certification''. (DO-178C succeeded DO-178B in 2012.)\\
    See: \url{https://en.wikipedia.org/wiki/DO-178B}, \url{https://en.wikipedia.org/wiki/DO-178C}

    \item IEC 61508, a basic functional safety standard applicable to all
    industries.\\
    See: \url{https://en.wikipedia.org/wiki/IEC_61508}
    
\end{itemize}

These standard involve the attainment of different ``levels'' referred to as
``SILs'' (Safety Integrity Levels) or ``ASILs'' (Automotive Safety Integrity
Levels).

\end{document}