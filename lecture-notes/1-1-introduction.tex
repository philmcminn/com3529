\input{common/latex-teach-common/page-setup}
\input{common/latex-teach-common/packages}
\input{common/latex-teach-common/macros}

\renewcommand{\title}[2]{
    \documenttitle{Com3529}{Software Testing \& Analysis}{#1}{#2}
}

% URLs
\newcommand{\coderepourl}{\url{https://github.com/philmcminn/com3529-code}\xspace}

% Code snippets
\newcommand{\code}[1]{\mbox{\tt #1}\xspace}

% README file
\newcommand{\readmefile}{{\tt README.md}\xspace}

% Directories
\newcommand{\lecturesdirectory}{\code{lectures}\xspace}
\newcommand{\practicalsdirectory}{\code{practicals}\xspace}

% Packages
\newcommand{\lecturespackage}{\code{uk.ac.shef.com3529.lectures}}
\newcommand{\lecturesexecutionpackage}{\code{uk.ac.shef.com3529.lectures.execution}}
\newcommand{\practicalspackage}{\code{uk.ac.shef.com3529.practicals}}

% Classes
\newcommand{\calendarclass}{\code{Calendar}}
\newcommand{\randomlytesttriangleclass}{\code{RandomlyTestTriangle}}
\newcommand{\signutilsclass}{\code{SignUtils}}
\newcommand{\spacedefenderclass}{\code{SpaceDefender}}
\newcommand{\stringutilsbuggyoneclass}{\code{StringUtilsBuggy1}}
\newcommand{\triangleclass}{\code{Triangle}}
\newcommand{\weekoneclass}{\code{Week1}}
\newcommand{\teststringutilsbuggyoneclass}{\code{TestStringUtilsBuggy1}}
\newcommand{\testweekoneclass}{\code{TestWeek1}}


% Methods
\newcommand{\classifymethod}{\code{classify}}
\newcommand{\daysbetweentwodatesmethod}{\code{daysBetweenTwoDates}}
\newcommand{\instrumentedclassifymethod}{\code{instrumentedClassify}}
\newcommand{\randomlytestclassifymethod}{\code{randomlyTestClassify}}
\newcommand{\signmethod}{\code{sign}}

% for logic tables
\newcommand{\LTTrue}{T}
\newcommand{\LTFalse}{F}


\booltrue{shownotes}


\begin{document}

\title{Introduction}{1.1 Introduction to Software Testing}

\note{Video}{Change slide on each example}

\section{Examples of High Profile Software Failures}

{\bf Self-Driving Cars.}
% Sources: https://en.wikipedia.org/wiki/Death_of_Elaine_Herzberg
% https://arstechnica.com/tech-policy/2018/05/report-software-bug-led-to-death-in-ubers-self-driving-crash/#:~:text=The%20fatal%20crash%20that%20killed,Amir%20Efrati%20reported%20on%20Monday.&text=Unfortunately%2C%20the%20software%20classified%20her,need%20to%20stop%20for%20her.
In the late evening of March 18, 2018, Elaine Herzberg was pushing her bicycle
laden with shopping bags across a highway in Arizona, USA. At approximately
9:58pm she was struck by a prototype Uber self-driving car. The vehicle had been
in self driving mode for approximately 20 minutes. The car's safety human backup
driver, was unable to intervene in time to stop the collision. It was later
reported that the crash occurred because of a bug in Uber's self-driving car
technology. According to reports, Uber's sensors {\it did} detect Herzberg as
she crossed the street with her bicycle. However, the software classified her as
a {\it false positive} and decided it didn't need to stop for her. Ms Herzberg
later died in hospital.

{\bf Heartbleed --- Open Source Encryption Libraries.}
% Source:
% https://www.computerworld.com/article/3412197/top-software-failures-in-recent-history.html
In April 2014, a member of Google's security team found a flaw in the encryption
library Open SSL that secures communications over computer networks. Although it
was quickly patched by most IT firms, the sheer scale of the services affected
means it is likely that there are still servers out there that remain vulnerable
to attack. Named ``Heartbleed'', this bug was so notorious it was the first to
get its own logo!

{\bf Prison Systems.}
% Source:
% https://www.computerworld.com/article/3412197/top-software-failures-in-recent-history.html
In December 2015, a software glitch caused more than 3,200 US prisoners to be
released early. The software calculated a prisoner's sentence depending on
good/bad behaviour and was introduced in 2002. According to reports, the problem
had persisted for 13 years until a new IT boss was appointed and informed
the governor's office. It is estimated that on average, prisoners were released
49 days early.

{\bf Divorce Settlements.}
% Source:
% https://www.computerworld.com/article/3412197/top-software-failures-in-recent-history.html
% https://www.theguardian.com/law/2015/dec/17/revealed-divorce-software-error-to-hit-thousands-of-settlements
Also in December 2015, the UK government's online calculator for calculating
spouse's financial worth was found to have a fault in ``Form E'' on its website,
meaning that those calculations were wrong for couples that used the online form
to process their divorce. According to reports this error has been inflating
spouse's finances since April 2014, but was only discovered 20 months
later. According to the Office of National Statistics there were 100,000
divorces in England and Wales in 2015. However, the true cost of this bug remains
unknown.

{\bf Meltdown and Spectre --- CPUs.}
% Source:
% https://www.computerworld.com/article/3412197/top-software-failures-in-recent-history.html
At the start of 2018, Google researchers revealed two CPU hardware
vulnerabilities, known as Meltdown and Spectre, which had affected almost all
computers on the market. Meltdown primarily affects Intel processors, while
Spectre affects Intel, AMD and ARM processors. Daniel Gruss, one of the
researchers that discovered the flaw at Graz University of Technology described
Meltdown as ``one of the worst CPU bugs ever found''. 
%
Although these are both primarily hardware vulnerabilities, they involve
communication with the operating system to access locations in its memory space.
%
Both bugs break the isolation between user applications and potentially the
operating system, allowing programs to access the memory of other programs, and
trick them into revealing potentially sensitive information.

%%%%%%%%%%%%%%%%%%%%%%%%%%%%%%%%%%%%%%%%%%%%%%%%%%%%%%%%%%%%%%%%%%%%%%%%%%%%%%%%%%%%%%

These are just five, high-profile software failures from the last five years.
There have been many more that I could have chosen! 

\newpage

\note{Video}{Talking head}

\section{Challenging Your Current View of Software Testing}

Let me ask you some questions:

\begin{itemize}
    \item {\it Do you think software testing is important?}
    %
    Even if you did not think so previously, one would hope that those few
    examples would have convinced you, that yes, software testing is important.
    
    \item {\it Do you think software testing is hard?}
    %
    Even if you've never had any serious bugs in the programs you've developed,
    likely you're now thinking that maybe yes, it is hard, otherwise these
    faults would have been found before the software was released and not
    wreaked the havoc that they did.
\end{itemize}

Do you think that:

\begin{itemize}
    \item {\it Testing is just a form of debugging?}

    \item {\it The purpose of software testing is to show that software works?}

    \item {\it Or, that the purpose of software testing is to show that software doesn't work?}

    \item {\it Or, that the purpose of software testing is not to show anything
    in particular, but just to reduce the risk of using software?}

    \item {\it Or, do you believe that testing is a mental discipline that helps
    all IT professionals develop higher quality software?}
\end{itemize}

%%%%%%%%%%%%%%%%%%%%%%%%%%%%%%%%%%%%%%%%%%%%%%%%%%%%%%%%%%%%%%%%%%%%%%%%%%%%%%%%%%

\note{Video}{Title Slide}

\section{Levels of Mature Thinking to Software Testing}

\note{Video}{Advance to Beizer's Maturity Model Slide}

Those last five questions were used by a prominent American Software Engineer
called Boris Beizer to characterise the maturity of an organisation's approach
to testing.

But they can also help us too think about what testing is, and our own view of
testing and our approach to it.

\note{Video}{These appear on the slide bullet-by-bullet – click to proceed to each level}

\begin{itemize}
    \item[0.] There's no difference between testing and debugging.

    \item[1.] The purpose of software testing is to show that software works.

    \item[2.] The purpose of software testing is to show that the software
    doesn't work.

    \item[3.] The purpose of software testing is not show anything in
    particular, but to reduce the risk of using software.

    \item[4.] Testing is a mental discipline that helps all IT professionals
    develop higher quality software.
\end{itemize}

The basic, {\it least} mature view of testing is that of {\bf Level~0} --- {\it
testing is the same as debugging}. At Level~0 thinking, programmers get their
programs to compile, then debug the programs with a few arbitrary inputs. This
view does not distinguish between a program's incorrect behaviour and a mistake
within the program. It also does very little to help develop software that is
reliable or safe.

In the more advanced {\bf Level~1} view of testing, {\it the purpose of software
testing is to show that software works}. While a significant step up from the
naive Level~0, this has the unfortunate problem that in any but the most trivial
of programs, correctness is virtually impossible to either achieve or
demonstrate. We'll come back to this point in more detail. 

But just for now, suppose we run a collection of tests and find no failures.
What do we know? Should we assume that we have good software or just bad tests?
Since the goal of correctness is impossible, test engineers usually have no
strict goal, real stopping rule, or formal test technique. If a development
manager asks how much testing remains to be done, the test manager has no way to
answer the question. In fact, test managers are in a powerless position because
they have no way to quantitatively express or evaluate their work.

In a {\bf Level~2} view of testing, {\it the purpose is to show failures}.
Although looking for failures is certainly a valid goal, it is also a negative
goal. Testers may enjoy finding the problem, but the developers never want to
find problems --- they want the software to work. 
%
\note{Video}{Next slide}
%
The US academic Bill McKeeman characterised this situation quite well when he
said {\it ``excellent testing can make you unpopular with almost everyone''}!

\note{Video}{Next slide to go back to the list, click to proceed to level 3 and 4}

Level~2 testing puts testers and developers into an adversarial relationship,
which can be bad for team morale. Beyond that, when our primary goal is to look
for failures, we are still left wondering what to do if no failures are found.
Is our work done? Is our software very good, or is the testing weak? Having
confidence in when testing is complete is an important goal for all testers.

{\bf Level~3} thinking lets us accept the fact that whenever we use software, we
incur some risk. The risk may be small, and the consequences unimportant, or the
risk may be great and the consequences catastrophic, but risk is always there.

This allows us to realise that the entire development team wants the same thing
--- to reduce the risk of using the software. In Level~3 testing, both testers
and developers work together to reduce risk.

Once the testers and developers are on the same ``team'', an organisation can
progress to {\bf Level~4} testing. Level~4 testing defines {\it testing as a
mental discipline that increases quality}. Various ways exist to increase
quality, of which creating tests that cause the software to fail is only one.

Adopting this mindset, test engineers can become the technical leaders of the
project (as is common in many other engineering disciplines). They have the
primary responsibility of measuring and improving software quality, and their
expertise should help the developers.

Beizer used the analogy of a spell checker. We often think that the purpose of a
spell checker is to find misspelled words. But in fact, the best purpose of a
spell checker is to improve our ability to spell.

Every time the spell checker finds an incorrectly spelled word, we have the
opportunity to learn how to spell the word correctly.  The spell checker is the
``expert'' on spelling quality. In the same way, Level~4 testing means that the
purpose of testing is to improve the ability of the developers to produce high
quality software.

Better software testing makes for better software development.

%%%%%%%%%%%%%%%%%%%%%%%%%%%%%%%%%%%%%%%%%%%%%%%%%%%%%%%%%%%%%%%%%%%%%%%%%%%%%%%%%%

\note{Video}{Advance Why Software Testing is Hard}

\section{Why Software Testing is Hard}

\note{Video}{Talking Head}

The relentless march of computing power is a fundamental force in modern
society.
%
Every year, computer hardware gets better and faster. Every year, the software
algorithms running on this hardware get better and faster. So it's natural to
wonder whether there are any limits to this progress, is there anything
computers can't do?

More specifically, is there anything that computers will {\it never} be able to
do, no matter how fast the hardware or how smart the algorithms?

Of course, you should know from {\it COM2109, Automata, Computation and
Complexity}, the answer to this question is {\it yes} --- there {\it are}
certain tasks that computers will never be able to perform. More than that,
Computer Scientists have an elegant way of classifying computational problems
according to whether they can be solved effectively, ineffectively, or not at
all. 

\note{Video}{Advance to table slide}

This table summarises these three categories of computational problems using the
terminology {\it tractable} for problems that can be solved efficiently, {\it
intractable} for problems whose only methods of solution are hopelessly
time-consuming, and {\it uncomputable} for problems that cannot be solved by any
computer program. 

\begin{center}
\begin{tabular}{@{}cccc@{}}
\toprule
 & {\bf Tractable}      & {\bf Intractable}    & {\bf Uncomputable} \\
 & {\bf Problems} ($N$) & {\bf Problems} ($P$) & {\bf Problems} \\
\midrule

{\bf Description} & 
Can be solved &
Method for solving exists, &
Cannot be solved by \\
&
efficiently &
but is hopelessly time &
any computer program \\
&
& 
consuming \\

{\bf Computable in theory?} & \cmark & \cmark & \xmark \\

{\bf Computable in practice?} & \cmark & ? & \xmark \\

{\bf Example} & 
Find the shortest &
Decryption &
{\bf Finding all bugs}
\\
&
route on a map &
&
{\bf computer programs} \\

\bottomrule   
\end{tabular}
\end{center}

{\bf Tractable problems} take polynomial time to solve and belong to the
complexity class $P$. Several instances of tractable problems exist, and
the table lists one that you will have come across in {\it COM1009, Introduction
to Algorithms and Data Structures}, namely {\it the shortest path problem}.
Given the details of a road network, find the shortest route between any two
points. Computers can quickly find the optimal solution to this problem even if
the input consists of every road on earth.

{\bf Intractable problems}, on the other hand, are problems where there is a
program that can compute the answer, but the program is too slow to be useful
--- more precisely, it takes too long to solve the problem with ``large''
inputs. These tend to be problems that take super polynomial or exponential time
and belong to the complexity class $\mathit{NP}$. Hence, these problems can be
solved in theory, but currently not in practice (except for small inputs) ---
and potentially not ever, unless the classic Computer Science conundrum of
whether $P = NP$ is solved. 

Intractable problems include Decryption, as you will come across in {\it COM3501
Computer Security and Forensics}, if you are signed up to that module. Given a
document encrypted with a modern encryption scheme, and without knowledge of the
decryption key, decrypt the document. Here, the difficulty of decryption depends
on the size of the decryption key, which is generally thousands of bits long in
modern implementations. Of course, an encryption scheme would be useless if the
decryption problem were tractable, so it should be no surprise that decryption
is believed to be intractable for typical key sizes. For the schemes in common
use today, it would require at least billions of years, even using the best
known algorithms on the fastest existing supercomputer, to crack an encryption
performed with a 4000-bit key. 

Finally, the last column of the table is devoted to problems that are
uncomputable. These are problems that cannot be solved by any computer program.
They cannot be solved in practice, and they cannot be solved in theory either.

Finding all bugs in a program, or ``complete software testing'', if you will ---
and therefore, program correctness --- is in general, uncomputable. 

Think about it --- finding all bugs in a program means that we must have tests
that execute each input to the program at least once. (In reality, this may {\it
still} not be sufficient for some types of programs, because there may be other
things that influence their behaviour\dots but for now let's leave that issue
for another time!)

Ok, so we try to execute a program with all inputs. If the program seems not
respond, how do we know if it's entered an infinite loop or not? Or, to put it
another way, whether it will halt? We cannot be sure that we're going to get an
``answer'' to our test. Essentially, we've hit on an instance of the halting
problem, a problem that is known to be {\it undecidable}.

Suppose by some analysis involving the internals of the program, we were able to
conclude that the program did always terminate. The next problem we encounter is
that exhaustive testing is likely to be {\it intractable} for a program of any
reasonable complexity. 

\note{Video}{Cut to demo at this point}

Take the Java method \daysbetweentwodatesmethod Java method that can be found in
the \calendarclass class of the \lecturespackage package of the code examples
repository for this module (available at \coderepourl). It takes two dates,
and computes the number of days between them. The dates are represented by three
integers each. An {\tt int} ranges from -2,147,483,648 to 2,147,483,647 in Java,
i.e. has $2^{32}$ possible values. With six {\tt int} inputs that's
${(2^{32})}^6$, or $2^{192}$, which is approximately $6 \times {10}^{57}$. Even
for a relatively simple method, that's a large number of inputs to execute.
Suppose each possible input takes on average one nanosecond to execute. It would
take ${10}^{41}$ years to try them all!

I don't suggest we do try it, because at some point in that time the sun will
have exploded and the universe will have ended. Assuming that we survived both
of those events, this will have been the longest lecture in history, and not one
that would have been fun for anybody.

Ok, but most of those values are not valid, you may argue. Well, let's hold the
fact that we need to check for both valid and invalid inputs for a second, and
concentrate on all valid dates in a reasonable time range, say 1/1/1600 to
31/12/2099 --- a timespan of 500 years, or 182,622 days (i.e., possible inputs
for each of the two dates). The dates can appear in any order (earliest or
largest first), so that's an input domain of ${182,622}^2$ in size, or
$33,350,794,884$ ($3 \times {10}^{10}$).

\note{Video}{Cut to meme, and let it hang for a second}

\note{Video}{Cut to Dijkstra slide when mentioned}

Ok, you continue to argue, but even still, we don't need every one of those
inputs to ensure the program is working? Well, this is the essence of the
software testing problem --- choosing a set of inputs that will reveal as much
information about the quality of the software as possible. But we don't {\it
know} that we've selected {\it all} the inputs that will reveal {\it all} of the
bugs. As the famous Computer Scientist Edsger Dijkstra's commented {\it
``Software testing can only show the presence, not the absence of bugs''}.
To have a hope of showing all failures, we'd really need to try all inputs. So
we've come full circle. 

And this is with the {\tt daysBetweenDates} method, a situation where we {\it
can}, in theory, enumerate all of the inputs. Yet the number of potential inputs
for most programs is so large that they are effectively infinite. Consider a
Java compiler --- the number of potential inputs to the compiler is not just all
Java programs or even all almost syntactically-correct Java programs, but all
possible strings. The only limitation is the size of the file that can be read
by the parse. Therefore, the number of potential inputs is effectively infinite,
and cannot be explicitly enumerated.

\note{Video}{Advance to Oracle slide, click to cross out company logo when mentioned}

Suppose we {\it could} execute the program with all these inputs. We then hit
upon another problem, which would be how to check that they're all correct! This
would represent significant manual activity. Ideally, we'd have an automated
checking process --- an ``{\bf oracle}'', as it's known in testing. (And in case
you were wondering, the name is nothing to do with the technology company behind
the Java language.) Constructing a 100\% reliable oracle is almost as difficult
as writing the program in the first place. This is known as the {\bf oracle
problem} in software testing.

%%%%%%%%%%%%%%%%%%%%%%%%%%%%%%%%%%%%%%%%%%%%%%%%%%%%%%%%%%%%%%%%%%%%%%%%%%%%%%%%%%

\note{Video}{Click to advance slide}

\section{The Beginnings of Software Testing as a Subject in Computer
\mbox{Science}}

{\bf The First Bug.}
%
We do not know when the first defect in a program was introduced. We know
though, when the first actual bug was found. 
%
On September 9, 1947, researchers at Harvard University in Cambridge,
Massachusetts, found that their computer, the Mark II, was delivering consistent
errors. When they opened the computer's hardware, they found ... a moth.
%
Maybe it was in search of plant food, or a warm place to lay its eggs, or both.
It wandered around in a warm, humming machine that constantly clicked and
rattled. But suddenly, it got stuck between the metal contacts of a relay ---
actually, one of 13,000 high performance relays commissioned for this particular
machine. The current killed it instantly, and its remains caused the machine to
fail.

The technician taped the moth into the logbook with the comment ``15:45 Relay
\#70 Panel F (moth) in relay. First actual case of bug being found''. The moth
thus became the living proof that computer problems could indeed be caused by
actual bugs.

\slide{1-1-first-bug}

\note{Video}{Click to advance slide}

{\bf The First Research Paper on Software Testing.}
%
The problem of software testing is as old as the computer itself. The first
research paper to make mention of software testing may have been by Alan Turing
in 1949 published in the ``{\it Report of a Conference on High Speed Automatic
Calculating Machines}'', an inaugural conference in Cambridge.

In the paper, Turing opens with the question ``How can one check a routine in
the sense of making sure it is right?''. In it, he proposes that a programmer
should make a number of assertions ``in order that the man who checks may not
have too difficult a task''. This early proposal to use assertions as part of
checking a program is working correctly is bourne out today in many test
frameworks, such as JUnit.

\slide{1-1-turing}

%%%%%%%%%%%%%%%%%%%%%%%%%%%%%%%%%%%%%%%%%%%%%%%%%%%%%%%%%%%%%%%%%%%%%%%%%%%%%%%%%%

\note{Video}{Click to advance slide}

\section{What You Will Get From This Module}

Hopefully you'll now have an appreciation as to why Software Testing is a hard
and important problem, and why it requires study.

% It'd be great to be able to say more about Rob's half of the course here

In the first half of this module, you'll gain:

\begin{enumerate}

    \item A deeper understanding of how failures happen, and the difference
    between faults and failures (Week~1).

    \item An understanding of systematic approaches to Software Testing,
    underpinned by coverage criteria (Weeks~2--4).

    \item Insights into some of the latest automated techniques for test case
    generation and bug finding, with hands-on practical experience (Week~5).

\end{enumerate}

Each week will be accompanied by a lab sheet that will involve pencil and paper
problems, as well as practical experience using Software Testing tools.

There will also be a weekly video quiz, called ``Who Wants To Be a Software
Tester?''. This isn't marked, it's just for fun! Watch it and answer the
questions to check your understanding of the weekly material.

\note{Video}{Click to advance slide}

Full code examples that appear in the lectures and practicals are available in a
GitHub repository at:

\begin{center}\coderepourl\end{center}

\end{document}