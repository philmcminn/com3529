\input{common/latex-teach-common/page-setup}
\input{common/latex-teach-common/packages}
\input{common/latex-teach-common/macros}

\renewcommand{\title}[1]{
    \moduletitle{Com3529}{Software Testing \& Analysis}{#1}
}

% URLs
\newcommand{\coderepourl}{\url{https://github.com/philmcminn/com3529-code}\xspace}

% Code snippets
\newcommand{\code}[1]{\mbox{\tt #1}\xspace}

% README file
\newcommand{\readmefile}{{\tt README.md}\xspace}

% Directories
\newcommand{\lecturesdirectory}{\code{lectures}\xspace}
\newcommand{\practicalsdirectory}{\code{practicals}\xspace}

% Packages
\newcommand{\lecturespackage}{\code{uk.ac.shef.com3529.lectures}}
\newcommand{\practicalspackage}{\code{uk.ac.shef.com3529.practicals}}

% Classes
\newcommand{\calendarclass}{\code{Calendar}}
\newcommand{\signutilsclass}{\code{SignUtils}}
\newcommand{\spacedefenderclass}{\code{SpaceDefender}}
\newcommand{\stringutilsbuggyoneclass}{\code{StringUtilsBuggy1}}
\newcommand{\triangleclass}{\code{Triangle}}
\newcommand{\weekoneclass}{\code{Week1}}
\newcommand{\teststringutilsbuggyoneclass}{\code{TestStringUtilsBuggy1}}
\newcommand{\testweekoneclass}{\code{TestWeek1}}

% Methods
\newcommand{\classifymethod}{\code{classify}}
\newcommand{\daysbetweentwodatesmethod}{\code{daysBetweenTwoDates}}
\newcommand{\signmethod}{\code{sign}}

% for logic tables
\newcommand{\LTTrue}{T}
\newcommand{\LTFalse}{F}


\booltrue{shownotes}


\begin{document}

\title{1.2 A Bug's Life}

\section{A Buggy Example}

\note{Video}{Talking Head}

As increasing users of software, the general population has, to an extent,
learned to live with software bugs. As computer scientists and software
engineers though, we know that bugs do not come from nowhere. There are many
phases to the software engineering lifecycle, and many opportunities for
mistakes to be made, be they in the initial requirements for the software, to
its implementation.

Let's look at the following Java method, by means of an example.

\note{Video}{Screen capture. Hover the mouse over the appropriate points.}

It's a method that finds all the duplicate letters in a string. I give it a
string input, and it returns a set of all the duplicate letters that it finds.

You can find and try out this example by cloning the Git code repository for
this module, which you can find at \coderepourl. The example method is in
the class~\stringutilsbuggyoneclass in the \lecturespackage~package. 

\begin{center} 
    \scalebox{0.8}{
    \begin{tabular}{rl} 
        \toprule
        Line \\
        \midrule
        1 & \verb|public static Set<Character> duplicateLetters(String s) {|\\
2 & \verb|    // lower case the string and remove all characters that are not letters|\\
3 & \verb|    s = s.toLowerCase().replaceAll("[^a-z.]", "");|\\
4 & \verb||\\
5 & \verb|    // initialise the result set|\\
6 & \verb|    Set<Character> duplicates = new TreeSet<>();|\\
7 & \verb||\\
8 & \verb|    // iterate through the string|\\
9 & \verb|    for (int i = 0; i < s.length(); i++) {|\\
10 & \verb|        char si = s.charAt(i);|\\
11 & \verb||\\
12 & \verb|        // iterate through the rest of the string, checking for the same letter|\\
13 & \verb|        for (int j = i; j < s.length(); j++) {|\\
14 & \verb|            char sj = s.charAt(j);|\\
15 & \verb||\\
16 & \verb|            if (si == sj) {|\\
17 & \verb|                // a match has been found, add it to the result set|\\
18 & \verb|                duplicates.add(si);|\\
19 & \verb|            }|\\
20 & \verb|        }|\\
21 & \verb|    }|\\
22 & \verb||\\
23 & \verb|    return duplicates;|\\
24 & \verb|}|\\
        \bottomrule
    \end{tabular}}
\end{center}    
~\\

Sometimes it works: If I give it the string {\tt "software testing"} and check
for duplicates of the letter {\tt "t"} the following JUnit test passes:

\begin{center} 
    \scalebox{0.8}{
    \begin{tabular}{l} 
        \midrule
        \verb$@Test$\\
\verb$public void testRepeatedChar() {$\\
\verb$    Set<Character> resultSet = StringUtilsBuggy.duplicateLetters("software testing");$\\
\verb$    assertTrue(resultSet.contains('t'));$\\
\verb$}$\\

        \midrule
    \end{tabular}}
\end{center} 

\newpage
And sometimes it doesn't: If I give it the string {\tt "software debugging"} and
check that there are not duplicates of the letter {\tt "t"}, the following JUnit
test fails:

\begin{center} 
    \scalebox{0.8}{
    \begin{tabular}{l} 
        \midrule
        \verb$@Test$\\
\verb$public void testNonRepeatedChar() {$\\
\verb$    Set<Character> resultSet = StringUtilsBuggy.duplicateLetters("software debugging");$\\
\verb$    assertFalse(resultSet.contains('t'));$\\
\verb$}$\\
        \midrule
    \end{tabular}}
\end{center} 

The tests referred to are in the \teststringutilsbuggyoneclass~class of the
\lecturespackage~package.

From the {\bf failure} of the method, as ascertained by the failing test, we can
deduce that the program has a bug, or more precisely, a {\bf fault}.

\note{Video}{Freeze on the code at this point}

Where is the fault in the example?

How did it cause the failure?

\note{Video}{Don't say ``line 13'' as it does not match up with the real code line number}

The fault is on the line containing the second, nested {\tt for} statement
(line~13). The method iterates through the characters of the string, but starts
the search for repeats of a particular letter at the {\it } index ({\tt i}),
rather than the {\it next} one ({\tt i + 1}). This means that {\it every} letter
in the string will appear in the returned set, regardless of whether they are
repeated or not, causing the method to fail. 

In this way, our ``positive'' test passes --- checking the letter {\tt t} {\it
is} repeated in {\tt "software testing"}, but ``negative'' test fails ---
checking the letter {\tt t} {\it is not} repeated in {\tt "software debugging"}
fails.

To correct the fault so that the failure no longer occurs, we need to change the
initialisation expression of the second {\tt for} loop from ``{\tt int j = i}''
to ``{\tt int j = i + 1}''. If we do this, both of our tests now pass.

\note{Video}{Title Slide}

\note{Video}{Advance slide}

\section{From Faults to Failures}

In general, a failure, such as that in our example program, comes about when the
following three conditions are met, one after the other:

\note{Video}{Advance slide}

\subsection*{1. The Program Location Containing the Fault is {\it Reached}
During Execution}

For the failure to occur the section of the code involving the fault also needs
to be reached by the execution path through the program. 

\note{Video}{Flick back to code}

If this buggy version of {\tt duplicateLetters} is executed with an empty string,
the nested {\tt for} loop would never be reached, and no failure would occur
(the method returns an empty set). 

The inner {\tt for} loop, containing the fault, must be executed for the failure
to occur. That is, there must be letters in the inputted string, as with our
failing test case involving the string ``{\tt software debugging}''.

\note{Video}{Advance slide}

\subsection*{2. The Fault {\it Infects} the State of the Program}

\note{Video}{Flick back to code}

When the fault is reached, in the initialisation step of the inner {\tt for}
loop, it causes an {\it infection}. An infected is when a fault affects the
program state such that it differs from that which was intended. In other words,
the program moves from an otherwise ``correct'' state into an ``incorrect''
state. In practice, this means that the program variables take on the wrong
values, and/or the wrong lines of subsequent program code are executed.

In this buggy version of {\tt duplicateLetters}, an infection always occurs as
soon as the fault is executed. The method thinks that it has found a duplicate
letter (when actually it's the {\it same} letter), and immediately places it
into the result set {\tt duplicates} that is returned by the method.

This is what happens with our failing test case involving the string {\tt
"software debugging"}. The letter {\tt t} is placed into the {\tt duplicates}
set as soon as it is encountered in the inputted string. This later causes the
assertion in our test to fail, because the returned set of characters contains
the letter {\tt t} when it shouldn't.

However, it's important to note that merely reaching a fault does not {\it
always} result in an infection. 
%
Suppose, for example, the fault was in the outer loop instead. Suppose it
started iterating from the character in the string at position 1 rather than 0
(i.e., it the loop initialisation step was ``{\tt int i = 1}'' instead of ``{\tt
int i = 0}''). For empty strings, the faulty initialisation step in the outer
loop would always be executed, but there would not be an infection --- the loop
condition would be false and the loop body would not be executed. Only strings
with letters in them would potentially infect the state.

\note{Video}{Advance slide}

\subsection*{3. The Infection {\it Propagates} to the Output, Causing a Failure}

\note{Video}{Flick back to code}

As the program executes, the infection may spread into later program states,
eventually spreading to the output of the program. The infected state is said to
{\it propagate} to the output, causing a failure. A failure is an externally
observable error in the program behaviour.

This is what happens with our failing test case involving the string {\tt
"software debugging"}. The set of duplicate characters is returned by the
method, containing characters that are not duplicates, including the letter {\tt
t}, as specifically checked for by our test's assertion.

However, an infection may not {\it always} propagate to the output. It may be
overwritten, masked, or corrected by some later program action.

For example, suppose we supply the string {\tt "ss"} to our original version of
the method with the fault in the inner {\tt for} loop. The letter {\tt s} is
entered in the {\tt duplicates} set as soon as it is encountered in the first
position in the string. The state is infected. However, the infection ``stops''
as soon as the second {\tt s} is encountered. Since {\tt s} is already in the
set, it is not added a second time. In effect, the infected state has been
``corrected'' in this particular circumstance --- the method returns the right
result, but only by the coincidence of an actual duplicate character later in
the string.

\note{Video}{Advance slide}

\section{Inducing Failures --- The RIP model}

The three conditions necessary for inducing failures are therefore {\bf
reachability (R)}, {\bf infection (I)}, and {\bf propagation (P)}:

\begin{enumerate}
    \item The fault must be {\bf reached (R)} in the code.
    
    \item The fault must then {\bf infect (I)} the state. 
    
    \item The infection must then {\bf propagate (P)} to the output, causing a
    failure.
\end{enumerate}

\note{Video}{Advance slide. Note Mnemonic is pronounced ``nemonic'' (silent ``m'' at start)}

You can remember these conditions through the mnemonic ``RIP''. 

The three conditions for revealing failures are referred to as the {\bf ``RIP
model''}.

\note{Video}{Advance slide}

\section{Being Precise about our Testing Terminology}

So note then that the word ``bug'' is rather an imprecise word, that could refer
to faults, failures, {\it or} infections. In the realm of software testing, it's
important then to be specific what we're talking about!

A {\bf fault} is a defect in a piece of code. 

An {\bf infection} is what happens when the defect is executed, and the program
state is affected. When a program's state is infected it starts to work
incorrectly. But at this point, it has not affected the output of the program
(and the fault, so far, has had no observable effect).

A {\bf failure} is when the infection propagates to the output of the program,
that is, the program observably behaves incorrectly.

\note{Video}{Advance slide}

The definitions of fault and failure allow us to distinguish testing from
debugging. 

{\bf Testing} is evaluating software by observing its execution.

A {\bf test failure} is an execution that results in a failure during testing.

{\bf Debugging} is the process of finding a fault, given some failure.

\section{Observing Failures and Inducing Test Failures --- The RIPR Model}

Note that there is a subtle difference between a ``test failure'' and a
``failure''. A failure is when a program returns an incorrect result or produces
observably incorrect behaviour. A test failure is when a particular execution
{\it reveals} a failure in the program while testing it. 

\note{Video}{Advance slide --- first RIPR slide}

In order to induce failures, a test needs to fulfil each of the three RIP
conditions --- {\it reachability}, {\it infection} and {\it propagation} ---
introduced earlier. To {\it reveal} the failure, however, a test needs to
ascertain that the output or behaviour of the software was incorrect. That is,
it needs to fulfil a further, fourth condition, {\it revealability}, to cause
the {\it test} to fail, thereby inducing a test failure. In practice,
revealability is fulfilled by the judicious use of assertions in a testing
framework like JUnit. 

\note{Video}{Back to the code}

To demonstrate revealability we return to the example at the beginning of this
lecture, when we executed {\tt duplicateLetters} with two JUnit tests --- one
that checked for {\tt t} in the set returned by the method in the case of
firstly the string {\tt "software testing"} and secondly the string {\tt
"software debugging"}.

Note that, in actual fact, {\it both} tests induce failures, thereby satisfying
the required conditions of reachability, infection and propagation. But, only
the latter test satisfies the {\it reachability} condition and actually {\it
observes} the failure, triggering an assertion violation that results in a test
failure. 

Why is that the case, and how does that happen? Well, when the method is
executed with the string {\tt "software testing"}, the set of letters returned
by the faulty method is: 

\begin{center}
    \{ {\tt a}, {\tt e}, {\tt f}, {\tt g}, {\tt i}, {\tt n}, {\tt o}, {\tt r},
    {\tt s}, {\tt t}, {\tt w} \}
\end{center}

This of course, is the set of {\it all} letters in the string, not just the
duplicates. The set of duplicates (i.e., the set of letters returned by the
corrected method) is just:

\begin{center}
    \{ {\tt e}, {\tt s}, {\tt t} \}
\end{center}

So while this particular execution {\it did} reach the fault, the fault {\it
did} infect the state, and the infection {\it did} propagate to the method's
output, the test's assertion {\it didn't} specifically capture and check the
right part of the output that {\it revealed} the failure to the test. It checked
for the letter {\tt t}, which is present both when the method fails and also
when it doesn't. If the assertion had checked for any other non-duplicated
letter, or checked the length of the set, it would have revealed the failure and
the test would have failed. (But the purpose of the test case was to check that
a duplicated characters did appear in the returned result. So in that sense, the
test achieved what it set out to do.)

The assertion checking the output of the method with the string {\tt "software
debugging"} on the other hand {\it does} reveal the fault of course, because in
this particular case, {\tt t} does not appear as part of both the correct and
incorrect output for this particular fault. 

\note{Video}{Back to slide, then advance to RIPR highlighted acronym slide at the appropriate point}

To conclude this part of the lecture then, for a successful test (i.e., one that
reveals software failures), we do not just need to have the conditions to induce
a failure (reachability, infection, and propagation --- RIP), the test needs to
{\it reveal} the failure too, meaning a further fourth condition is necessary
--- revealability. The four conditions together are known as {\bf RIPR} for
short, or the {\bf RIPR} model.

\section{Test Cases and Test Suites}

A {\bf test case}, then, is an execution of a program with the intention of
revealing failures caused by any faults that it may or may not have. 

A test case consists of four main ingredients:

\begin{enumerate}

    \item {\bf The inputs necessary to put the software into the appropriate
    state for testing.} 
    %
    In the context of unit testing, this may involve constructing the required
    objects and invocating certain methods on them to move them into some state
    required so that we can do some specific test. In system testing, this may
    involve launching a GUI and proceeding through some part of it with example
    inputs to get to the part of the system we want to test. 
    %
    The {\tt duplicateLetters} method is a simple static, stateless, method, so
    no special inputs were needed to put it into a state so that we could
    properly test it.

    \item {\bf The actual test case values.} 
    In the {\tt duplicateLetters} example, these were the actual strings we
    tested the method with, i.e., {\tt "software testing"}, {\tt "software
    debugging"}, etc.

    \item {\bf The expected results of the tests.}
    It's not enough just to execute the test, we need to know if the outputs are
    the correct ones. This is established by assertion statements in many
    testing frameworks. For example, we expect {\tt t} to be a duplicated
    character in {\tt "software testing"} but not in {\tt "software debugging"}.

    \item {\bf Finally, a reset of the system state.} 
    As a last step, the test case may need to do some housekeeping like
    resetting any system state it has interfered with, or by releasing any
    resources it has acquired (e.g. network connections, etc). This to clear
    things down in preparation for the ``next'' test, so that the starting state
    for that test is always reliably the same. Not doing so may interfere with
    the results of the next test that is executed, and whether any test failures
    are triggered as a result. 

\end{enumerate}    

So note that ingredients~1 and~4 may not always be needed in practice. Omission
of ingredient~4 where it is required, or inadequately performing it, may cause
subsequent tests to behave non-deterministically --- that is, they may pass and
fail without any changes to the code under test, because of differing or
unreliable starting states. Such tests are known as {\it flaky tests}.

Test values for ingredients 1 and 2 may be determined automatically. This is the
subject of a later lecture!

\note{Video}{Advance to the last slide and let it ``play''}

Finally, a {\bf test set}, or a {\bf test suite} is simply a set of test cases.

\end{document}