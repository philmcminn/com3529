\input{common/latex-teach-common/page-setup}
\input{common/latex-teach-common/packages}
\input{common/latex-teach-common/macros}

\renewcommand{\title}[1]{
    \moduletitle{Com3529}{Software Testing \& Analysis}{#1}
}

% URLs
\newcommand{\coderepourl}{\url{https://github.com/philmcminn/com3529-code}\xspace}

% Code snippets
\newcommand{\code}[1]{\mbox{\tt #1}\xspace}

% README file
\newcommand{\readmefile}{{\tt README.md}\xspace}

% Directories
\newcommand{\lecturesdirectory}{\code{lectures}\xspace}
\newcommand{\practicalsdirectory}{\code{practicals}\xspace}

% Packages
\newcommand{\lecturespackage}{\code{uk.ac.shef.com3529.lectures}}
\newcommand{\practicalspackage}{\code{uk.ac.shef.com3529.practicals}}

% Classes
\newcommand{\calendarclass}{\code{Calendar}}
\newcommand{\signutilsclass}{\code{SignUtils}}
\newcommand{\spacedefenderclass}{\code{SpaceDefender}}
\newcommand{\stringutilsbuggyoneclass}{\code{StringUtilsBuggy1}}
\newcommand{\triangleclass}{\code{Triangle}}
\newcommand{\weekoneclass}{\code{Week1}}
\newcommand{\teststringutilsbuggyoneclass}{\code{TestStringUtilsBuggy1}}
\newcommand{\testweekoneclass}{\code{TestWeek1}}

% Methods
\newcommand{\classifymethod}{\code{classify}}
\newcommand{\daysbetweentwodatesmethod}{\code{daysBetweenTwoDates}}
\newcommand{\signmethod}{\code{sign}}

% for logic tables
\newcommand{\LTTrue}{T}
\newcommand{\LTFalse}{F}


\booltrue{shownotes}


\begin{document}

\title{3.3 Coverage Criteria --- Interim Conclusions}

\section{Introduction}

We're at the end of this block of the course on coverage, and we've encountered
a number of criteria already! How do we know which to use, and when? What level
of coverage should we aim to attain anyway? 

We're now going take a look at the pros and cons of coverage criteria, and how
to best apply them in practice.

\section{Industry and Coverage Criteria}

It's fair to say that there's a lot of scepticism about testing from some
quarters of industry. Some of this is down to the belief that coverage means
code coverage only. Even in the software industry though, most people agree that
code coverage is a useful metric.

The view changes somewhat for high integrity or safety critical systems where
coverage criteria are very important, so much so that they form the basis of
many software standards, as we saw with MCDC.

\section{Pros and Cons of Coverage}

Some common criticisms of coverage criteria can be summarised as follows:

\begin{enumerate}
    \item They can generate too many test requirements.
    \item Not all the test requirements are useful.
    \item Higher coverage does not necessarily guarantee that more bugs will be
    found.
\end{enumerate}

And these are all true. But there are several pros as well, let's go back to
some of the original motivations for coverage criteria and re-evaluate them in
the light of what we've learnt:

\begin{enumerate}

    \item {\bf Coverage Criteria ensure our test suite have certain sets of 
    properties.}
       
    Professional testers are often following testing strategies similar to those
    provided by coverage criteria when designing tests, although they may not
    realise it. In particular, good testers excel at thinking about rare
    situations, putting together combinations of events, or attempting
    ``what-if'' scenarios that are not too dissimilar to the steps performed as
    part of Input Domain Analysis.
    
    Coverage Criteria help formalise the process for creating tests, on the
    basis of test requirements. Although we've discussed them at a mostly
    informal level, we could write them out as algorithms, if we wanted.
    This systematic derivation is useful, because as well as telling us about
    what's covered --- i.e., what we {\it did} --- we also know what's {\it not}
    covered, i.e., what we {\it didn't do}. Perhaps more usefully, compared to a
    human-driven process, they can tell us what we {\it forgot}. It's hard to
    always remember things without a system for reminding you!

    \item {\bf Coverage Criteria tell us how much testing has been done, 
    and provide an indication as to when to stop testing.}
    
    Coverage Levels provide a useful metric as to how much testing we've done.
    Although it says nothing about the number of bugs we've not found, most
    coverage criteria provide confidence that the more testing you've done, the
    less {\it likely} it is that unfound serious bugs remain. Coverage Levels
    give developers, testers, and software managers concrete metrics to talk
    about. We can decide that ``enough'' of the software has been tested once we
    reach a certain coverage level (more on that later).

\end{enumerate}

\section{Software Testing Research and Coverage Criteria}

Coverage Criteria continue to be important to Software Testing research and
researchers (of which Rob and I are just two!).

As we've seen Software Testing involves many difficult problems, and Coverage
Criteria play into some of the insights behind some of those issues. One day,
the metrics that coverage criteria provide may be able to inform practical
testing about useful things like:

\begin{itemize}

    \item The relationship between actual faults, failures and types of coverage
    and coverage levels.

    \item Where to concentrate testing in order to find failures.

    \item When time is limited, which tests to run first, and in what order.

    \item What the ideal test suite size is.
    
    \item How to check that a fault has actually been repaired, and has not
    introduced further faults. 
    
    \item Whether a fault can be automatically repaired.

    \item ... 

\end{itemize}    

The list goes on! We are not there with providing solid answers to these
interesting research topics yet. If they interest you, maybe you'd considered
doing a PhD in this area!

\end{document}
