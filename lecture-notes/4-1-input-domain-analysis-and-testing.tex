\input{common/latex-teach-common/page-setup}
\input{common/latex-teach-common/packages}
\input{common/latex-teach-common/macros}

\renewcommand{\title}[2]{
    \documenttitle{Com3529}{Software Testing \& Analysis}{#1}{#2}
}

% URLs
\newcommand{\coderepourl}{\url{https://github.com/philmcminn/com3529-code}\xspace}

% Code snippets
\newcommand{\code}[1]{\mbox{\tt #1}\xspace}

% README file
\newcommand{\readmefile}{{\tt README.md}\xspace}

% Directories
\newcommand{\lecturesdirectory}{\code{lectures}\xspace}
\newcommand{\practicalsdirectory}{\code{practicals}\xspace}

% Packages
\newcommand{\lecturespackage}{\code{uk.ac.shef.com3529.lectures}}
\newcommand{\lecturesexecutionpackage}{\code{uk.ac.shef.com3529.lectures.execution}}
\newcommand{\practicalspackage}{\code{uk.ac.shef.com3529.practicals}}

% Classes
\newcommand{\calendarclass}{\code{Calendar}}
\newcommand{\randomlytesttriangleclass}{\code{RandomlyTestTriangle}}
\newcommand{\signutilsclass}{\code{SignUtils}}
\newcommand{\spacedefenderclass}{\code{SpaceDefender}}
\newcommand{\stringutilsbuggyoneclass}{\code{StringUtilsBuggy1}}
\newcommand{\triangleclass}{\code{Triangle}}
\newcommand{\weekoneclass}{\code{Week1}}
\newcommand{\teststringutilsbuggyoneclass}{\code{TestStringUtilsBuggy1}}
\newcommand{\testweekoneclass}{\code{TestWeek1}}


% Methods
\newcommand{\classifymethod}{\code{classify}}
\newcommand{\daysbetweentwodatesmethod}{\code{daysBetweenTwoDates}}
\newcommand{\instrumentedclassifymethod}{\code{instrumentedClassify}}
\newcommand{\randomlytestclassifymethod}{\code{randomlyTestClassify}}
\newcommand{\signmethod}{\code{sign}}

% for logic tables
\newcommand{\LTTrue}{T}
\newcommand{\LTFalse}{F}


\booltrue{shownotes}


\begin{document}

\title{Coverage Criteria}{3.2 Grey-Box Coverage Criteria based On \\ Input Domain Analysis}

\section{Test Equivalence}
%
\note{Video}{Show bag of Haribo.}

Suppose I want to try all the sweets in a bag of Haribo, but I don't want to eat
the whole pack --- because that would be greedy, and I'd put on weight.

\note{Video}{Show an empty packet, but me really fat with a cushion up my top}

If I was to take an organised and systematic approach to this problem, I could
empty the sweet packet out into a big bowl, and sort them into smaller bowls,
by type -- one bowl for the cola bottles, one for the fried eggs, and so on.
Once they were sorted into bowels, I could try one sweet from each bowel, on the
assumption that one of each type of sweet will taste the same as all the others
of that type. 

\note{Video}{Show me taking a sweet from each bowel. Mmmm!}

\note{Lecture}{This could be interactive. You could ask the audience how to
solve the ``Haribo problem''. While they may say to try one of each type, ask
how would I know I've got all the types... Could ask a student to test a couple
of sweets in the whole bowel.}

In a very fundamental way, all testing is about choosing elements from the input
space of the software being tested. 

We can view all coverage criteria as defining ways to divide the input space
according to its test requirements. 

Given a set of inputs that can satisfy a particular test requirement, the
assumption is that any input is ``just as good'' as every other input in the set
for revealing faults. 

The assumption is that {\it any} value or collection of values that satisfies a
test requirement will be ``just as good'' as any other in the set. 

It's also a view that {\it input domain analysis} takes, in a very direct way.


\section{An Overview of Testing Using Input Domain Analysis}

The {\it input domain} of a program is the set of possible values that its input
parameters can have. At a system level, inputs can be user entered data, or
points and clicks of a mouse. At a unit level, input parameters may correspond
directly to method parameters, or global variables, or objects representing the
current system state. 

Input domain analysis involves partitioning an input domain into so-called {\it
equivalence classes} that are assumed to contain equally useful values from a
testing perspective. The tester is then free to select any value belonging to
each equivalence class for actual testing. 

Suppose we have a Java method with the signature {\tt public static boolean
isPositive(int n)}. The input domain is the set of values belonging to the type
{\tt int}, comprising all integers from -2,147,483,648 to 2,147,483,647. It
would make sense to test this method with at least two test cases --- one
involving a positive value, and one involving a negative value. 
%
The set of positive values and the set of negative values are examples of
equivalence classes --- we're basically saying that the program should behave
the same with any input chosen from each of those respective sets.

This way of testing has several advantages. The tester does not need to
understand the implementation --- although that can help (and hence why we class
it as a {\it grey-box} technique). But yet, it is fairly easy to get started,
because it can be applied with no automation, and very little training.
Everything we need to know for testing is based on a description of the inputs.
This method also has the advantage that it is easy to ``tune'' to generate more
or fewer tests, with the classic testing trade-off of cost (reduced by creating
fewer tests) vs fault-finding effectiveness (which is generally increased by
creating more tests).

An input domain {\bf partition} must satisfy two properties:

\begin{enumerate}
    \item The partition must cover the entire domain (i.e., it is {\bf complete}) 
    \item The equivalence classes must not overlap (i.e., they are {\bf disjoint})
\end{enumerate}

While our our partitioning of the input domain for {\tt isPositive} is indeed
disjoint, note that it is {\it not} complete: We did not consider zero, which is
neither positive nor negative, and not a value that {\tt isPositive} should
return {\tt true} for. This is an example of how analysing and partitioning the
input domain can clarify our thinking and throw up interesting cases that we may
not have considered otherwise. 
%
On the one hand, we could put zero in the same equivalence class as the negative
numbers, because it is not positive, and therefore {\tt isPositive} should
return {\tt false}, as for all the other numbers in that equivalence class. On
the other hand, zero could be viewed as a special case, so we might want to put
it in an equivalence class by itself. This would give us three equivalence
classes overall, $e_1$, $e_2$, $e_3$:

\slide{3-2-input-domain-for-ispositive}

We can now {\it cover} the partition by selecting a value from each equivalence class,
based on the aforementioned principle that any value from an equivalence class
is as good as any other for testing. This would give us three test cases, e.g.
-100, 0, and 100, for which {\tt isPositive} should return {\tt false}, {\tt
false}, and {\tt true}, respectively:

\begin{center}
\begin{tabular}{r|crr}
    {\bf Test Case} & {\bf Equivalence Class} & {\bf Example Input} & {\bf Expected Outcome} \\
    \midrule
    1 & Less than zero    & -100 & {\tt false} \\
    2 & Zero              &    0 & {\tt false} \\
    3 & Greater than zero &  100 & {\tt true}  \\
\end{tabular}
\end{center}

We said that we can tune this method to give us more or fewer tests. We could
have put zero in with the negative numbers class, for example, to give us two
tests rather than three. We can increase the number of tests by subdividing our
so-called equivalence classes further, to ensure more fine-grained testing.
%
For example, we know that programmers are prone to making mistakes around
boundaries. This is one reason why we might want to keep the zero case.
Programmers easily make off-by-one errors when writing conditions, e.g. by
writing ``{\tt if (n > 1) return true}'' instead of ``{\tt if (n >= 1) return
true}''. We may decide that not all the values in a particular equivalence class
should be considered equivalent after all. We can then subdivide our partition
even further to ensure special consideration to so-called {\it boundary values},
i.e., 1 in this example. Putting the value 1 in its own equivalence class gives
us a further equivalence class, and an additional test.
%
The consideration of using test values around boundaries is called {\bf boundary
value analysis}.

Later in this lecture, we'll see how we can consider several partitions together
as part of an advanced testing strategy --- one which if not done carefully, can
lead to a combinatorial explosion of test cases! 


\subsection{Using the Concept of ``Characteristics'' to Perform Partitioning}

Rarely are we faced with testing an example as simple as {\tt isPositive}, with
only one input parameter with a domain (that of the {\tt int} type in Java) that
is finite and is easy to work with in terms of precisely defining where the
boundaries of the partition fall.

Take for example a program that takes a collection of elements {\tt elements} to sort,
where there is no limit on the length of the collection. How might we partition
this example, with a multi-dimensional, unbounded domain?

In general, input domain partitions are based on a {\it characteristic} of the
program's inputs. Possible characteristics for {\tt elements} might be:

\begin{itemize}
    \item Whether {\tt elements} is {\tt null} (true, false)
    \item The {\it size} of {\tt elements} (0, 1, greater than 1)
    \item The {\it initial order} of {\tt elements} (ascending, descending, arbitrary)   
\end{itemize}

Each characteristic allows the tester to define a partition, each of which must
be complete and disjoint (as previously defined). 

Consider the characteristic ``initial order of {\tt elements}''.
This could be used to create the following (but defective) partitioning: 

\begin{itemize}
    \item Initial order of {\tt elements}:
    \begin{itemize}
        \item $e_1 =$ {\tt elements} sorted in ascending order
        \item $e_2 =$ {\tt elements} sorted in descending order
        \item $e_3 =$ {\tt elements} is in an arbitrary order
    \end{itemize}
\end{itemize}

However, this is {\bf not} a valid partitioning. Specifically, if the {\tt
elements} collection is of length 0 or 1, then it will belong in all three
equivalence classes. That is, the equivalence classes are {\it not} disjoint.
The easiest strategy to address this problem is to make sure that each
characteristic addresses only one property. The problem above is that the
notions of being sorted into ascending order and being sorted into descending
order are part of the same characteristic. Splitting into two characteristics,
namely {\it sorted ascending} and {\it sorted descending}, solves the problem.
The result is the following (valid) partitioning of two characteristics:

\begin{itemize}
    \item {\tt elements} sorted in ascending order:
    \begin{itemize}
        \item $e_1 =$ true
        \item $e_2 =$ false
    \end{itemize}
\end{itemize}

\begin{itemize}
    \item {\tt elements} sorted in descending order:
    \begin{itemize}
        \item $e_1 =$ true
        \item $e_2 =$ false
    \end{itemize}
\end{itemize}

With these equivalence classes, collections of size 0 or 1 are in the {\it true}
class for both characteristics.

The completeness and disjointness properties are formalised for pragmatic
reasons, not to make the method deliberately more complicated!
%
Partitions that are not complete or disjoint probably reflect a lack of clarity
in the rationale for the partition. 
%
In particular, if a partition actually encodes two or three rationales, the
partition is likely to be quite messy, and is also likely to violate either the
completeness or the disjointness property (or both!). Identifying and correcting
completeness or disjointness errors typically results in aesthetically more
pleasing partitions. 
%
% Further, formally objectionable ``partitions'' cause
% unnecessary problems when generating tests, as discussed below. 
%
% The rest of this lecture assumes complete and disjoint partitions. 

\section{Applying Input Domain Analysis}

Ok, having given an overview of how input domain analysis works, we're now
going to discuss how to apply it in practice. 

\subsection*{Step 1: Identify the Functions to Test}

The first step of applying input domain analysis is identifying the different
{\it functions} of the software that we want to apply it to, and what the input
domain is for those functions. 

The word ``function'' conjures up procedures, methods, or units of a programming
language. While a function in input domain analysis may map directly to a
particular method of a program, it is actually a more general concept. A
function is a particular {\it operation} of the system does, or a particular
aspect of its {\it behaviour} that we want to focus our tests on. 

For example, if we're testing a mobile app, it might be a particular screen that
we're testing. For a web app, it might be a form, and how the application
responds to different field entries. If it's a computer game, it might be the
behaviour of an individual game character, for example, testing how they react
given certain stimuli or what they can do in certain situations. If it's a
console app, we might want to consider the entire app if it relatively simple
and has one specific purpose (e.g., a system function like {\tt mkdir}). But if
it's a complex application with different modes and functionalities, like a
compiler, we would need to break its behaviour into smaller distinct operations.

In summary, how we carve up a complete system into functions is not a formal
process and relies on the ingenuity of the tester and their knowledge of the
system.  

\subsection*{Step 2: Identify the Input Domain}

Once we have a set of functions, we need to identify the input domain for each.
This depends on the function, of course. In some cases, this may seem obvious
(the input to {\tt isPositive}, for example). In others, it's less so, and may
include aspects of the operating environment or the system state. 

\subsection*{Step 3: Identify Characteristics}

Once we have identified the input domain, there are two schools of thought as to
how to model it in terms of determining its characteristics. In practice, we
probably would want to use aspects of both.  

\subsubsection*{Interface-Based Input Domain Modelling}

{\it Interface-based input domain modelling} involves simply identifying the
parameters to the interface of the function more or less as with the examples we
just described --- i.e., method parameters, form fields etc. It's an easy,
almost mechanical way to go about things and makes the characteristics and
partitions in the next steps easy to apply. 

A downside of this approach is that parts of the functionality may depend
on combinations of specific values of several interface parameters, that is,
certain domain knowledge about what the function does is not taken into account.

Consider the \triangleclass~class from last week's lab sheet. The
\classifymethod~method, which classifies the type of a triangle based on three
integer inputs representing the lengths of each of its sides. In an interface
input domain analysis, {\tt side1} will have a number of characteristics, as
will side {\tt side2} and {\tt side3}. Since the three variables are all
integers, the interface-based characteristics for each will likely be identical
and which will probably all focus on legal and illegal inputs (e.g., zero and
negative numbers).

\subsubsection*{Functionality-Based Input Domain Modelling}

The idea of {\it functionality-based input domain modelling} is to describe the
input domain in terms of the intended functionality of the system, allowing the
tester to incorporate some {\it semantics} and {\it domain knowledge} into the
analysis. 

Let's return to the \classifymethod~method of \triangleclass again. While the
interface-based approach would identify the input as three integers, the
functionality-based approach would instead say the input is a triangle, and
would partition the input into different types of triangle (isosceles,
equilateral, etc.). 

We'll return to the \triangleclass~class as a basis for a practical case study
later in the lecture. 

Developing a functionality-based input domain model is more challenging than the
interface-based approach, but the resulting tests are potentially more effective.

{\it Pre-conditions} are good sources of information for functionality-based
characteristics. Preconditions are predicates that should hold upon entry into a
function, and are stated informally in several forms, for example in documentation.
Preconditions separate ``normal'' behaviour from abnormal, exceptional
behaviour. For example, if a method {\tt choose()} is supposed to select a
value, it has a precondition that a value must be available to select. A
characteristic might be whether the value is available or not.

{\it Post-conditions} are also good sources for characteristics. Postconditions
are predicates that should hold upon exit from a function. In the case of the
Triangle Classification program, the different kinds of triangles are based on
the postconditions of the method --- i.e., if the lengths of the sides are valid
and all equal, the triangle is equilateral. 

We also need to look for other relationships between variables. These may be
explicit or implicit. For example, a curious test engineer given a method {\tt
m()} with two object parameters {\tt x} and {\tt y} might wonder what happens if
{\tt x} and {\tt y} reference the same object, or to two logically equal
objects.
~\\

Generally, it is preferable to use specifications or documentation instead of
program code to develop characteristics. The idea is that the tester should
apply input domain partitioning based on domain knowledge about the problem, not
the implementation. Overall, teh more semantic information that can be
incorporated into characteristics, the better the resulting test set is likely
to be.

Finally, it is usually better to have many characteristics with few blocks than
the reverse. It is also true that characteristics with small number of blocks
are more likely to satisfy the completeness and disjointness properties. 

\subsection*{Step 4: Identify Equivalence Classes}

After choosing characteristics, we must partition them into sets of values, or
equivalence classes. This is another creative step that allows the tester to
tune the test process: More classes will result in more tests, which will
require more resources but which increases the likelihood of finding faults.
Fewer classes will result into fewer tests, saving resources but potentially at
the expense of finding fewer faults. 

General strategies for identifying equivalent sets of values for equivalence
classes include:

\begin{itemize}

    \item {\bf Valid sets of values} --- what sets of values represent legal
    inputs that we need to test the function with?
    
    \item {\bf Invalid sets of values} --- what sets of values represent illegal
    inputs that the function needs to handle (i.e., by rejecting with a message
    to the user, requesting re-entry of the value(s), or throwing an exception)?

    \item {\bf Normal sets of values} --- what sets of inputs would be
    considered ``usual'' or typical for the function? For example, the number of
    days in a month includes the numbers 1--28 for all months.

    \item {\bf Abnormal sets of values} --- what sets of inputs would be
    considered ``unusual'' or atypical? For example, all months except February
    also have days 29 and 30, and some months have 31. At the extreme end of
    atypical is February in a leap year, where it goes from having 28 days to 29
    days. 

    \item {\bf Boundary values} --- as we've already noted in this lecture,
    software engineers (i.e., humans!) typically make mistakes around
    boundaries. Programmers are prone to using {\tt >} and {\tt <} instead of
    {\tt >=} and {\tt <=} respectively in an {\tt if} statement or looping
    condition, as well as making off-by-one errors in calculations. Choosing
    values close to boundaries are sometimes good at revealing faults.

    \item {\bf Special values} --- for example, {\tt null} is typically a
    special case for references or pointers in languages such as Java that need
    to be treated differently from non-{\tt null} values.

\end{itemize}    

And, of course, we must check for the following to ensure the partitioning is
valid:

\begin{itemize}

    \item {\bf Missing sets of values} --- check that the union of all classes
    of a characteristic completely covers the input domain of that
    characteristic, to ensure the partition is {\it complete}.
    
    \item {\bf Overlapping sets of values} --- check that no value belongs to
    more than one equivalence class, to ensure the partitioning is {\it
    disjoint}.
    
\end{itemize}

\section{Case Study: The Triangle Classification Program}

Armed with all the information we need to apply Input Domain Analysis, let's now
make a more complete attempt at using it with the \classifymethod~method of the
\triangleclass~class. 

One common partitioning for an integer variable considers the relation of the
variable's value to some special value in the testable function's domain, such
as zero.

This table shows three partitions for three different characteristics using an
interface-based input domain model:

\begin{center}
\begin{tabular}{rllll}
    \toprule
    & & \multicolumn{3}{c}{\bf Equivalence Classes} \\
    \cline{3-5}
    & {\bf Characteristic} & $e_1$ & $e_2$ & $e_3$ \\
    \midrule
    1 & Relation of {\tt side1} to 0 & greater than 0 & equal to 0 & less than 0 \\
    2 & Relation of {\tt side2} to 0 & greater than 0 & equal to 0 & less than 0 \\
    3 & Relation of {\tt side3} to 0 & greater than 0 & equal to 0 & less than 0 \\
    \bottomrule    
\end{tabular}
\end{center}

Consider Partition~1 for {\tt side1}. If one value is chosen from each
equivalence class, the result is three tests. For example, we might choose {\tt
side1} to have the value 7 in test 1, 0 in test 2, and -3 in test 3. Of course,
we also need values for {\tt side2} and {\tt side3} of the triangle to complete
the test case values. Notice that some of the equivalence classes represent
valid triangles and some represent invalid triangles. For example, no valid
triangle can have a side of negative length. 

It is easy to refine this categorisation to get more fine-grained testing if the
budget allows. For example, more equivalence classes can be created by
separating inputs value 1. This decision would lead to partitions with four
blocks:

\begin{center}
\begin{tabular}{rlllll}
    \toprule
    & & \multicolumn{4}{c}{\bf Equivalence Classes} \\
    \cline{3-6}
    & {\bf Characteristic} & $e_1$ & $e_2$ & $e_3$ & $e_4$ \\
    \midrule
    1 & Length of {\tt side1} & greater than 1 & equal to 1 & equal to 0 & less than 0 \\
    2 & Length of {\tt side2} & greater than 1 & equal to 1 & equal to 0 & less than 0 \\
    3 & Length of {\tt side3} & greater than 1 & equal to 1 & equal to 0 & less than 0 \\
    \bottomrule    
\end{tabular}
\end{center}

Notice that if the type of the inputs for each fo the sides were floating point
(e.g., {\tt float} or {\tt double} in Java), this partitioning would {\it not}
yield valid partitions. None of the equivalence classes would include values
between 0 and 1, so they would {\it not} be complete and cover the entire input
domain. However, since the domain of each variable is integers, the partitions
are valid.

Our partitions, so far, are inteface-based and only use white-box information
about the \classifymethod~method (it has three integer input parameters). A
functionality-based approach can use the semantic information of the traditional
geometric classification of triangles, for example:

\begin{center}
\begin{tabular}{rlllll}
    \toprule
    & & \multicolumn{4}{c}{\bf Equivalence Classes} \\
    \cline{3-6}
    {\bf Partition} & {\bf Characteristic} & $e_1$ & $e_2$ & $e_3$ & $e_4$ \\
    \midrule
    1 & Geometric Classification & scalene & isosceles & equilateral & invalid \\
    \bottomrule    
\end{tabular}
\end{center}

See the practical from last week to refresh yourself on what constitutes an
equilateral, isosceles, and a scalene triangle.

Note that this does not form valid partitioning. An equilateral triangle is also
isosceles, so the partition is not disjoint ($e_3$ overlaps with $e_2$). 

As such we must correct the partitions:

\begin{center}
\begin{tabular}{rlllll}
    \toprule
    & & \multicolumn{4}{c}{\bf Equivalence Classes} \\
    \cline{3-6}
    & {\bf Characteristic} & $e_1$ & $e_2$ & $e_3$ & $e_4$ \\
    \midrule
    1 & Geometric Classification & scalene & isosceles, & equilateral & invalid \\
    & & & not equilateral & & \\
    \bottomrule    
\end{tabular}
\end{center}

While partitioning, it often useful for the tester to identify candidate values
for each block to be used in testing. The reason to identify values now is that
choosing specific values can help the test engineer think more concretely about
the predicates that describe each block. While these values may not prove
sufficient when refining test requirements to test cases, they do form a good
starting point:

\begin{center}
    \begin{tabular}{lllll}
        \toprule
        {\bf Param} & $e_1$ & $e_2$ & $e_3$ & $e_4$ \\
        \midrule
        Triangle & (4, 5, 6) & (3, 3, 4) & (3, 3, 3) & (3, 4, 8) \\
        \bottomrule    
    \end{tabular}
    \end{center}

A different approach to the ``isosceles but not equilateral'' problem is to
break the characteristic ``Geometric Classification'' into four separate
characteristics, one for each type:

\begin{center}
\begin{tabular}{rlll}
    \toprule
    & & \multicolumn{2}{c}{\bf Equivalence} \\
    & & \multicolumn{2}{c}{\bf Classes} \\
    \cline{3-4}
    & {\bf Characteristic} & $e_1$ & $e_2$ \\
    \midrule
    1 & Triangle is Scalene & true & false \\
    2 & Triangle is Isosceles & true & false \\
    3 & Triangle is Equilateral & true & false \\
    4 & Triangle is Invalid & true & false \\
    \bottomrule    
\end{tabular}
\end{center}

This kind of approach is highly recommended and it invariably satisfies the
disjointness and completeness properties. 

We can combine the equivalence classes together to produce a test case where the
triangle is equilateral but not isosceles, as we'll discuss next. The impossible
combinations, for example where the triangle is equilateral {\it and} invalid
are handled by introducing constraints that control the combinations. 

\section{Combination Coverage Criteria}

So far, we have ignored an important question -- ``How should we consider
multiple partitions at the same time?'' This is the same as asking ``What
combinations of equivalence classes should we choose values from?'' For example,
we might wish to require a test case that satisfies equivalence class 1 from
characteristic $c_1$ with equivalence class 3 from characteristic $c_2$. 

The most obvious choice is to choose all combinations:

{\bf All Combinations Coverage} --- all combinations of equivalence classes from
all characteristics must be used. 

As we've seen before, simply combining everything with everything else quickly
results in an explosion of test cases that need to considered. Things can get
out of hand pretty quickly!

At the opposite extreme is the following criterion:

{\bf Each Choice Coverage} --- one value from each equivalence class for each
characteristic should be used in at least one test case. 

This criterion leaves a lot of flexibility in terms of how to combine
equivalence classes, but doesn't {\it require} that any classes be combined at
all. 

For this reason, criteria exist to try to limit the effect of combinatorial
explosion, while still allowing for a varying amount of coupling of equivalence
classes. The first one we will look at is {\it Pair-Wise Coverage}:

{\bf Pair-Wise Coverage} --- a value from each equivalence class for each
characteristic must be combined with a value from every equivalence class from
every equivalence class for each other characteristic.

A natural extension to this is to require $t$ values, instead of just pairs:

{\bf T-Wise Coverage} --- a value from each equivalence class for each group of
$t$ characteristics must be combined.

T-wise coverage is a generalisation of the last two criteria: If the value for
$t$ is equal to the number of partitions, then it is equivalent to all
combinations. If $t=2$, then t-wise coverage is equivalent to pair-wise
coverage.

Despite attempting to limiting combinatorial explosion by only requiring $t$
characteristics to be considered together, T-wise coverage can still result in
very many test cases, and experience suggests that going beyond pair-wise (i.e.,
$t=2$) does not help much.

While pair-wise and t-wise combine equivalence classes ``blindly'', without
regard for which classes are being combined, the next criterion considers
combining classes by incorporating a small but crucial piece of domain knowledge
into the equation --- namely, by deciding which is the most ``important'' class
for each characteristic. This class is called the {\it base choice}.

{\bf Base Choice Coverage} --- A base choice equivalence class is chosen for
each characteristic, and a base test is formed by using the base choice for each
characteristic. Subsequent tests are chosen by holding all but one base choice
constant and suing each non-base choice in each other characteristic.

Sometimes it's difficult to choose a single base choice, and we may decide that
multiple base choice are needed. The following criterion allows for multiple
base choices to be made for each characteristic:

{\bf Multiple Base Choice Coverage} --- at least one, and possibly more base
choice blocks are chosen for each characteristic, and base tests are formed by
using each base choice for each characteristic at least once. Subsequent tests
are chosen by holding all but one base choice constraint for each base test and
using each non-base choice in each other characteristic.

% Subsumption criterion -- slide

\subsection{Constraints Among Partitions}

One thing to note about applying the combination coverage criteria above is that
some combinations of equivalence classes will be infeasible. 

{\bf Constraints} are relations between equivalence classes from different
characteristics. There are two kinds of constraints:

\begin{enumerate}

    \item {\bf Cannot Be Combined.} Two equivalence classes cannot be combined
    together. 
   
    \item {\bf Must Be Combined.} Two equivalence classes must be combined
    together.

\end{enumerate}

We came across an example of this in our final partitioning of the
\classifymethod~method of the \triangleclass: ``Triangle is Equilateral'' as
``true'' {\it cannot be combined} with ``Triangle is Invalid'' as also ``true''.
Instead ``Triangle is Equilateral'' as ``true'' {\it must be combined} with
``Triangle is Invalid'' as ``false''.

How constraints are handled depends on the coverage criterion chosen. For all
combinations, pair-wise and t-wise coverage, the only reasonable option is to
drop the infeasible pairs from consideration. 

However, the situation is different for base-choice coverage. If a particular
class conflicts with teh base case, then the obvious thing to do is to change
the offending choice for the base case so that the variation is feasible. 

\section{Final Thoughts}

\subsection{Using More than One Input Domain Model}

% TODO: what is an input domain model -- define. Is it one of the tables above??

For a complex program it might be better to have several small input domain
models than one large one. This approach allows for a divide-and-conquer
strategy when modelling characteristics and equivalence classes. Another
advantage with multiple input domain models for the same software is that it
allows for varying levels of coverage.

For instance, one input domain model may contain only valid values and another
input domain model may contain invalid values to focus on error handling. The
valid value input domain model may be covered using a higher level of coverage.
The invalid value input domain model may use a lower level of coverage. 

Multiple input domain models may be overlapping as long as the test cases
generated make sense. However, overlapping input domain models are likely to
have more constraints.

\subsection{Checking the Input Domain Model}

It is important to check the input domain model. In terms of characteristics,
the test engineer should ask whether there is any information about how the
function behaves that is not incorporated in some characteristics. This is
necessarily an informal process.

The tester should also explicitly check each characteristic for the completeness
and disjointness properties. The purpose of this check is to make sure that, for
each characteristic, not only do the equivalence classes cover the complete
input space, but selecting a particular equivalence class implies excluding all
the other equivalence classes in that characteristic. 

If multiple input domain models are used, completeness should be relative to the
portion of the input domain that is modelled in each input domain model. When
the tester is satisfied with the characteristics and their equivalence classes,
it is time to choose with combinations of values to test with, and identify
constraints among the equivalence classes.



\end{document}