\input{common/latex-teach-common/page-setup}
\input{common/latex-teach-common/packages}
\input{common/latex-teach-common/macros}

\renewcommand{\title}[2]{
    \documenttitle{Com3529}{Software Testing \& Analysis}{#1}{#2}
}

% URLs
\newcommand{\coderepourl}{\url{https://github.com/philmcminn/com3529-code}\xspace}

% Code snippets
\newcommand{\code}[1]{\mbox{\tt #1}\xspace}

% README file
\newcommand{\readmefile}{{\tt README.md}\xspace}

% Directories
\newcommand{\lecturesdirectory}{\code{lectures}\xspace}
\newcommand{\practicalsdirectory}{\code{practicals}\xspace}

% Packages
\newcommand{\lecturespackage}{\code{uk.ac.shef.com3529.lectures}}
\newcommand{\lecturesexecutionpackage}{\code{uk.ac.shef.com3529.lectures.execution}}
\newcommand{\practicalspackage}{\code{uk.ac.shef.com3529.practicals}}

% Classes
\newcommand{\calendarclass}{\code{Calendar}}
\newcommand{\randomlytesttriangleclass}{\code{RandomlyTestTriangle}}
\newcommand{\signutilsclass}{\code{SignUtils}}
\newcommand{\spacedefenderclass}{\code{SpaceDefender}}
\newcommand{\stringutilsbuggyoneclass}{\code{StringUtilsBuggy1}}
\newcommand{\triangleclass}{\code{Triangle}}
\newcommand{\weekoneclass}{\code{Week1}}
\newcommand{\teststringutilsbuggyoneclass}{\code{TestStringUtilsBuggy1}}
\newcommand{\testweekoneclass}{\code{TestWeek1}}


% Methods
\newcommand{\classifymethod}{\code{classify}}
\newcommand{\daysbetweentwodatesmethod}{\code{daysBetweenTwoDates}}
\newcommand{\instrumentedclassifymethod}{\code{instrumentedClassify}}
\newcommand{\randomlytestclassifymethod}{\code{randomlyTestClassify}}
\newcommand{\signmethod}{\code{sign}}

% for logic tables
\newcommand{\LTTrue}{T}
\newcommand{\LTFalse}{F}


\booltrue{shownotes}


\begin{document}

\title{Introduction}{1.3 Types of Testing}

\section{Types of Fault}

\note{Video}{Talking Head}

We tend to think of faults as mistakes in particular lines of program code, as
with our example. But the fault could also be to do with missing code --- i.e.,
code that is not there but should be. 

For this reason, it is helpful to think about faults in terms of whether they
are {\it faults of omission} or {\it faults of commission}. 

\note{Video}{First slide}

{\bf Faults of omission} are when the code {\it doesn't} do something it is
supposed to. That is, the code does not implement some functionality that it
should. For example, a requirement that was left unspecified, or which the
developer simply forgot about. 

{\bf Faults of commission} are when the code does something {\it more} that it
is supposed to do. For example, a C program iterating beyond the end of an
array, causing a buffer overflow. 

Hence, faults do not necessarily correspond to an existing code location ---
both faults of omission {\it and} commission may be caused by code that is
missing from the implementation. 

You'll see later in the course that different types of testing are better for at
detecting faults of omission, while others are better at detecting faults of
commission. 

\note{Video}{Next slide hidden in the presentation, not needed for video}

\section{Who's At Fault for The Fault?}

\note{Video}{Talking Head}

Because failures result from faults in the code, and because code is comes from
programmers, does that mean the programmer is to blame? Not necessarily.
Consider the following:

\begin{itemize}
    \item The original requirements did not forsee future changes. Think about
    the year 2000 problem for example. In the 1960s, programmers did not forsee
    their code would still be used in 40 or 50 years time, and only used two
    digits to store years.

    \item A program behaviour may only become classified as a ``failure'' when
    the user sees it for the first time. In other words, the programmer did not
    know the behaviour they were programming was actually incorrect.

    \item In a modular program, a failure may occur because of incompatible
    interfaces of two modules.

    \item In a distributed program, a failure may be the result of some
    unpredictable interaction of components. 

\end{itemize}

This points to the need for different types of testing that target different
aspects of the software lifecycle, something formalised by the ``V-Model''.

\note{Video}{Cut to introduction slide}

\section{The V-Model}

\note{Video}{Next slide}

Tests can be derived from requirements and specifications, design artifacts, and
the source code. A different level of testing accompanies each distinct software
development activity. 

\note{Video}{Click to reveal each one in the slide}

\begin{itemize}

    \item {\bf Acceptance Testing} --- assess software with respect to requirements
    
    \item {\bf System Testing} --- assess software with respect to its
    architectural design 

    \item {\bf Integration Testing} --- assess software with respect to
    subsystem design

    \item {\bf Module Testing} --- assess software with respect to detailed design
    
    \item {\bf Unit Testing} --- assess software with respect to its implementation

\end{itemize}

\note{Video}{Next slide}

These types of testing can be organised into the ``V-Model'', which illustrates
a typical scenario for testing levels and how they related to software
development activities by isolating each step. Information for each test level
is typically derived from the associated development activity. Indeed, standard
advice is to design the tests concurrently with each development activity, even
though the software will not be in an executable form until the implementation
phase. The reason for this advice is that the mere process of explicitly
articulating tests can identify defects in design decisions that otherwise
appear reasonable. Early identification of defects is by far the best means of
reducing their ultimate cost. 

\note{Video}{Click to reveal the caution!}

Note that the V-Model is {\bf not intended to mirror the waterfall methodology
or imply a waterfall-type process to engineering software}. The synthesis and
analysis activities generically apply to any software development process.

\note{Video}{Click to highlight each of these:}

The {\bf requirements analysis} phase of software development captures the
customer's needs. {\bf Acceptance Testing} is designed to determine whether the
completed software in fact meets these needs. In other words, acceptance testing
probes whether the software does what the users want. Acceptance testing must
involve users or other individuals who have strong domain knowledge.

The {\bf architectural design} phase of software development chooses components
and connectors that together realise a system whose specification is intended to
meet the previously identified requirements. {\bf System Testing} is designed to
determine whether the assembled system meets its specifications. It assumes that
the pieces work individually, and asks if the system works as a whole. This
level of testing looks design and specification problems. It is a very expensive
place to find lower-level faults and is usually not done by the programmers, but
a separate testing team.

The {\bf subsystem design} phase of software development specifies the structure
and behaviour of subsystems, each of which is intended to satisfy some function
in the overall architecture. Often the subsystems are adaptations of previously
developed software. {\bf Integration Testing} is designed to to assess whether
the interfaces between modules in a given subsystem have consistent assumptions
and communicate correctly. Integration testing must assume that modules work
correctly. 

The {\bf detailed design} phase of software development determines the structure
and behaviour of individual modules. A program {\it unit}, or procedure, is one
of more contiguous program statements, with a name that other parts of the
software use to call it. Units are called {\it methods} in Java.
A {\it module} is a collection of related units that are assembled in a file,
package, or class. {\bf Module Testing} is designed to assess individual modules
in isolation, including how the component units interact with each other and
their associated data structures. Most software development organisations make
module testing the responsibility of the programmer. 

{\bf Implementation} is the phase of software development that actually produces
code. {\bf Unit Testing} is designed to assess the units produced by the
implementation phase and is the ``lowest'' level of testing. In some cases, such
as when building general-purpose library modules, unit testing is done without
knowledge of the encapsulating software application. As with module testing,
most software development organisations make unit testing the responsibility of
the programmer. 
%
It is straightforward to package unit tests together with the corresponding
code through the use of tools such as {\it JUnit} for Java classes.

\note{Video}{Click to highlight regression testing:}

Not shown by the V-Model is {\bf Regression Testing}, a standard part of the
maintenance phase of the software development. Regression Testing is testing
that is done after changes are made to the software's implementation, and its
purpose is to help ensure that the updated software still possesses the
functionality it had before the updates. 

\note{Video}{Advance to next slide to reveal vanilla V-Model}

Mistakes in requirements and high-level designs wind up being implemented as
faults in the program; and as such testing can reveal them. Unfortunately, the
software faults that come from requirements and design mistakes are visible only
through testing months or years after the original mistake. The effects of the
mistake tend to be dispersed throughout multiple software components (unlike our
original example!), hence such faults are usually difficult to pin down and
expensive to correct. 

\section{Closing Examples}

\note{Video}{Advance slide, and show video}

% Original video source
% https://www.youtube.com/watch?v=kYUrqdUyEpI&ab_channel=TopQuark

\note{Video}{Talking head}

Some faults can only be found at the system level. One dramatic example was the
launch failure of the first Ariane 5 rocket, which exploded 37 seconds after
liftoff on June 4, 1996. The low-level cause was an unhandled floating-point
conversion exception in an inertial guidance system function. It turned out that
the guidance system could never encounter the unhandled exception when used on
the Ariane 4 rocket. In other words, the guidance system function was correct
for Ariane 4. The developers of Ariane 5 quite reasonably wanted to re-use the
successful inertial guidance system from the Ariane 4, but no one re-analysed
the software in the light of the substantially different flight trajectory of
Ariane 5. Furthermore, the system tests that would have found the problem were
technically difficult to execute, and so were not performed. The result was
spectacular, and expensive.

\note{Video}{Notes only:}

For more information about the Ariane~5 failure, see 
\url{https://www.youtube.com/watch?v=5tJPXYA0Nec&ab_channel=TopQuark}.

\note{Video}{Advance slide, zooming Mars Orbiter}

Another public failure was the Mars Climate Orbiter of September 1999, which
crashed due to a misunderstanding in the units of measure used by two modules
created by two separate software groups. One module computed thruster data in
non-SI units of pound-force seconds, and forwarded the data to a module that 
expected it in the SI units of newton-seconds. This is a typical {\it
integration fault}. However this one was enormously expensive, both in terms of
money and prestige.

\note{Video}{Advance slide, zooming Pentium Chip}

Finally, one of the best examples of the difference between unit testing and
system testing can be illustrated in the context of the infamous Pentium bug. In
1994, Intel introduced its Pentium microprocessor, and a few months later,
Thomas Nicely, a mathematician at Lynchburg College in Virginia, found that the
chip gave incorrect answers to certain floating-point division calculations. 

The chip was slightly inaccurate for a few pairs of numbers; Intel claimed
(probably correctly) that only one in nine billion division operations would
exhibit reduced precision. The fault was the omission of five entries in a table
of 1,066 values (part of the chip's circuitry) used by a division algorithm. The
five entries should have contained the constant +2, but the entries were not
initialised and contained zero instead. The MIT mathematician Edelman claimed
that ``the bug in the Pentium was an east=y mistake to make, and a difficult one
to catch'', an analysis that misses one of the essential points. This was a
difficult mistake to find during system testing, and indeed, Intel claimed to
have run millions of tests using this table. But the table entries were left
empty because a loop termination condition was incorrect; that is, the loop
stopped storing numbers before it was finished. This turns out to be a very
simple fault to find during unit testing; indeed, analysis showed that almost
any unit level {\it coverage criterion} would have found this multimillion dollar
mistake. 

\note{Video}{Talking head}

We'll be discussing software test coverage criteria in detail in the next
lecture.
%
But that concludes the third part of our introduction to software testing, so
be sure to tune in next week!

\end{document}