\input{common/latex-teach-common/page-setup}
\input{common/latex-teach-common/packages}
\input{common/latex-teach-common/macros}

\renewcommand{\title}[1]{
    \moduletitle{Com3529}{Software Testing \& Analysis}{#1}
}

% URLs
\newcommand{\coderepourl}{\url{https://github.com/philmcminn/com3529-code}\xspace}

% Code snippets
\newcommand{\code}[1]{\mbox{\tt #1}\xspace}

% README file
\newcommand{\readmefile}{{\tt README.md}\xspace}

% Directories
\newcommand{\lecturesdirectory}{\code{lectures}\xspace}
\newcommand{\practicalsdirectory}{\code{practicals}\xspace}

% Packages
\newcommand{\lecturespackage}{\code{uk.ac.shef.com3529.lectures}}
\newcommand{\practicalspackage}{\code{uk.ac.shef.com3529.practicals}}

% Classes
\newcommand{\calendarclass}{\code{Calendar}}
\newcommand{\signutilsclass}{\code{SignUtils}}
\newcommand{\spacedefenderclass}{\code{SpaceDefender}}
\newcommand{\stringutilsbuggyoneclass}{\code{StringUtilsBuggy1}}
\newcommand{\triangleclass}{\code{Triangle}}
\newcommand{\weekoneclass}{\code{Week1}}
\newcommand{\teststringutilsbuggyoneclass}{\code{TestStringUtilsBuggy1}}
\newcommand{\testweekoneclass}{\code{TestWeek1}}

% Methods
\newcommand{\classifymethod}{\code{classify}}
\newcommand{\daysbetweentwodatesmethod}{\code{daysBetweenTwoDates}}
\newcommand{\signmethod}{\code{sign}}

% for logic tables
\newcommand{\LTTrue}{T}
\newcommand{\LTFalse}{F}


\booltrue{shownotes}


\begin{document}

\title{2.2 White-Box Coverage Criteria based On\\ Control Flow Analysis}

\note{Video}{Title slide}

\section{Introduction}
%
So far, we've met probably the simplest and most well-known coverage criterion
of all, {\bf Statement Coverage}. Statement Coverage is where each test
requirement entails that each program statement is executed (covered) by a test
suite. 

Statement Coverage is a good baseline coverage criterion often used in industry
because it requires that each line of the code has at been exercised at least
once in the course of testing before its release.

Statement Coverage belongs to a family of coverage criteria based on the
control-flow of a program. Coverage criteria in this family are defined in terms
of the program's {\bf Control Flow Graph (CFG)}, and require that test suites
execute elements of that graph or particular paths through it. 

These types of coverage criteria are therefore {\bf white-box}, since CFGs are
derived from program code). Because they ultimately involve executing certain
types of structures in a piece of code, they are often more commonly referred to
as {\bf structural coverage criteria} or {\bf code coverage criteria}.

\note{Video}{First slide}

\section{Control Flow Graphs (CFGs)}

\note{Video}{Click to start animation}

Computers execute programs statement by statement, moving to statements
appearing out of series in different parts of the program, depending on the
evaluation of different decisions depending on what inputs it was supplied with.

The sequence of statements executed by a program given a particular input is
referred to as the ``path'' taken through the program. The movement of program
execution from one particular statement to another is called the ``flow of
control'' (or simply control flow).

\note{Video}{Next slide}

A Control Flow Graph (CFG) represents all the possibilities for the flow of
control, i.e., the ways in which a program can move from the execution of each
statement in the program to another, in the form of an easy-to-visualise graph. 

\note{Video}{Next slide}

In a CFG, nodes represent program statements, and edges represent transfers of
control from one statement to another. 

\note{Video}{Click to make map appear on slide}

CFGs are a bit like a geographical map, but of a program: the places are program
statements, and the roads between them are possible ways in which control can
flow from one statement to another. The possible ``routes'' with the map are
potential execution paths through the program.

\note{Video}{Next slide}

\subsection{Linear Sequences of Statements}

For linear sequences of statements, CFGs are quite straightforward. Each node
has a single edge that goes to the next node representing the next statement in
the program. 

\note{Video}{Next slide}

Linear sequences of statements are so uninteresting they often get merged into
single nodes that are sometimes referred to as ``basic blocks'', or just
``blocks''. Things get more complicated when we consider ``if'' and looping
statements.

\note{Video}{Next slide}

\subsection{``If'' Constructs}

The CFG node for an {\tt if} statement (call this ``$A$'') has two possible
successor nodes, the node corresponding to the statement when $A$ evaluates to
true when the program is executed (let's call this ``$B$'') and the node
corresponding to the statement when $A$ evaluates to false (let's call this
``C''). Therefore $A$ would have an outgoing edge to $B$ and an outgoing edge to
$C$ in the graph:

\slide{2-2-if-construct}

The edges are labelled with ``T'' and ``F'' to explicitly show the edges from
the {\tt if} statement corresponding to the true and false outcomes
respectively.

We call edges labelled as ``T'' and ``F'' {\bf branches}. We call the edge
labelled with a ``T'' the {\it true branch} and the edge labelled with an ``F''
the {\it false branch}. The predicate controlling which branch is taken (the
condition between the parenthesis in the the {\tt if} statement) is known as the
{\bf branch predicate}.

\note{Video}{Next slide}

Here's that again, this time with some actual code. 

\slide{2-2-if-construct-code}

This time we've labelled the nodes according to the line numbers that each
statement appears on. The branches are the edges leading out of node 2, with the
true branch going from to node 3, while the false branch goes to node 5. The
branch predicate is the call to the method {\tt isMorning()}, which returns {\tt
true} or {\tt false}, depending on the time of day.

Note that line 4 doesn't correspond to a node in the graph, because it's not
really a program statement, it's a delimiter.

However, we {\it did} label line 5 as a node, even though there's only a
delimiter on that line too --- the end of the method! A complete CFG has special
entry (``start'') and exit (``end'') nodes to represent the start and end of
execution of the program, respectively. 

\note{Video}{Next slide}

These nodes are labelled ``s'' and ``e''. If we treat the method as the program,
we can label it with specific start and end nodes. We can say that the end node
corresponds to line 5. If there was a specific ``{\tt return}'' statement, that
instead would represent the end node. Likewise, we can say that the start node
corresponds to line 1 (the method header).

\note{Video}{Next slide}

Here's a more complicated {\tt if} statement involving {\tt else} clauses, and
six branches altogether:

\slide{2-2-if-else}

Note that the node that is the end of the {\tt if} construct, which is actually
the end node for the whole of the method, e, is the successor node for several
prior nodes, including the following nodes executed when the {\tt if} or a
particular {\tt else if} is {\tt true} and also {\tt false} for the final {\tt
else if}.



\note{Video}{Next slide}

\subsection{``While'' Constructs}

The CFG node for a {\tt while} statement (call this $A$) has two successor nodes
--- one where the {\tt while} condition is true (let's call this $B$), and one
where the {\tt while} condition is false (call this $C$). $B$ represents the
first statement of the loop body, while $C$ is the first statement following the
loop. 

The edge to $B$ is the true branch from $A$, while the edge to $C$ is the false
branch. This time, the branching predicate is the {\tt while} condition.

\slide{2-2-while-construct}

There is an edge leading from the last statement of the loop body back to $A$.
Depending on whether the condition at $A$ is evaluated as true or false or not
depends on whether control is transferred back to $B$ for a second iteration of
the loop, or whether the loop has terminated, and control should pass to $C$.
Note that we do not add edges for each iteration of the loop. (For one, we may
not know how many iterations the loop has.) Control simply transfers to the
first node of the loop body, in the same way that control passes from statement
to statement in the program.

\note{Video}{Next slide}

\subsection{``For'' Constructs}

The subgraphs of the CFG corresponding to {\tt for} constructs is much more
complicated, not least because the statement involving the {\tt for} keyword is
consists of three parts and is actually three statements in one. The first part
(call this $A$) is the loop initialisation statement, where variables
responsible for iteration are initialised. 

\slide{2-2-for-construct}

The second part is the loop iteration statement ($B$), which controls whether
the loop body is entered. Finally $C$ is the loop update. So the flow of control
is from $A$ to $B$. If $B$ is true, the loop body $D$ is entered. Following $D$,
the loop update statement $C$ is executed, then control passes back to $B$ to
determine if the loop should iterate a further time. If $B$ is false, the loop
terminates, and control passes to the first statement after the loop, $E$. 

Since $B$ controls loop iteration, it is also the branch predicate as far as
{\tt for} loops are concerned. The true branch to $D$ initiates execution of the
loop body, while the false branch to $E$ terminates the loop.

\note{Video}{Next slide}

\subsection{CFGs for Programs with Multiple Constructs}

Of course, programs in general do not consist of a single {\tt if}, {\tt for},
or {\tt while}. So far, we've just shown the building blocks. The CFG for the
{\tt countZeros} method shows an example of this. The {\tt countZeros} program
takes an array of {\tt int}s and returns a count of the number of elements that
are equal to zero.

\slide{2-2-count-zeros}

The program has a {\tt for} loop with an {\tt if} statement forming its loop
body. As such the main form of the CFG for the method is similar to the template
of a {\tt for} loop, except the body is expanded to account for the {\tt if}
statement. Notice how we've combined the initial block of statements, from the
entry node, $s$, to the loop initialisation statement as one node. Note that we
{\it cannot} include 3b as part of this block, because we need it to directly
proceed the loop update statement, 3c. 3c does not go back to the start of the
method, re-executing 2a and 3a {\it before} the loop condition statement 3b.

We'll use the CFG of {\tt countZeros} as the basis of examples of the code
coverage criteria based on CFGs, i.e. Structural Coverage Criteria.

\note{Video}{Next slide}

\section{Structural Coverage Criteria}

Structural code coverage criteria involve executing different elements or paths
through the control flow graph.

\subsection{Statement Coverage} 

If our test suite traverses each node of the CFG (i.e., at some point during the
execution of the tests, each statement in the program being tested has been
reached), it will have achieved Statement Coverage. 

\note{Video}{Next slide}

For {\tt countZeros}, a test case with the input {\tt x = [0]} would execute all
nodes. This input takes the path s $\rightarrow$ 2 $\rightarrow$ 3a
$\rightarrow$ 3b $\rightarrow$ 4 $\rightarrow$ 5 $\rightarrow$ 3c $\rightarrow$
3b $\rightarrow$ e. In practice, for larger programs, it may not be as easy as
this! Obtaining full statement coverage may require several test cases. 
    
\note{Video}{Next slide}

Like many other subjects in Computer Science, Software Testing is an evolving
topic with a heavy practical application in the software industry. As such, many
of the same or roughly equivalent concepts in Software Testing have different
names. Software Testing feeds from Maths and other areas of Computer Science,
for example Graph Theory and Program Analysis --- to name just two --- each of
which have their own sets of terms.

Although it's not critically important to know all of these different terms,
this background knowledge is sometimes useful to know when somebody is referring
to something that is the more or less or exactly the same thing as something,
and not be thrown off by the difference in name. 

We've encountered a few of these already --- for example, ``Structural
Coverage'' and ``Code Coverage'' basically refer to the same thing. Likewise,
Statement Coverage exists in many guises:

\note{Video}{These appear one by one in the slide:}

\begin{itemize}
    \item {\bf Line Coverage.} Obtaining 100\% Statement Coverage ultimately
    involves executing every line of code (except blank lines and comments,
    which are not usually intended to affect the behaviour of compiled code).
    {\it Line Coverage} is commonly a term used in industry to mean Statement
    Coverage.

    \item {\bf Node Coverage}. Statement Coverage is sometimes loosely referred
    to as ``Node Coverage'', but usually only in academic circles, and
    importantly only in the context of the CFG, whose nodes the criterion is
    interested in. There are many other types of graphs in Software Engineering,
    so it's important to know the nodes of which type of graph we're talking
    about!
    
    \item {\bf Basic Block Coverage.} Because it involves executing each node of
    the CFG, and because the nodes of a CFG represent blocks of statements that
    follow each other sequentially, covering each node of the CFG is also
    referred to as ``Basic Block Coverage'' or just ``Block Coverage'' for
    short. In theory, covering an individual statement is not the same as
    covering the block of sequential statements to which it belongs. In
    practice, however, individual statements cannot be executed without
    executing the entire block as well. The term ``basic block'' is mostly due
    to the compilers community. 
\end{itemize}

\note{Video}{Next slide}

Note that our test case for obtaining full statement coverage seemed to cover
every part of the CFG, that is, except the false edge from node 4. 

\note{Video}{Next slide}

Statement Coverage requires we execute every statement, but not necessarily
every branch in the program (i.e., every true/false decision the program makes).
This is taken care of by a stronger coverage criterion, Branch Coverage.

\note{Video}{Next slide}

\subsection{Branch Coverage} 

Branch Coverage requires every branch (i.e., every true and false edge in the
CFG) is executed at least once by the test suite. In executing every true/false
edge is executed, we will also execute actually all unlabelled edges in the CFG
too.

The test case {\tt x = [0]} ensures the branch predicate at node 4 is true, and
so the true branch will be taken. Since the test case involves an array with an
element in it, the loop is both entered and exited, so that the true and false
branches from the {\tt for} loop statement (node 3b) are also taken. To execute
the false branch from node 4, we need a test case that has an element that is
non-zero, so that the branch predicate at node 4 evaluates to false. We can
arrange this by updating our existing test case to {\tt x = [0, 1]} or by having
an additional test case, {\tt x = [1]}.

\note{Video}{Next slide}

Other names for Branch Coverage (or its equivalent) include:

\note{Video}{These appear one by one in the slide:}

\begin{itemize}

    \item {\bf Decision Coverage.} Branches represent the outcome of each {\it
    decision} made by a program. Hence, Branch Coverage is sometimes also called
    ``Decision Coverage''.

    \item {\bf Predicate Coverage.} Because taking each branch involves the
    evaluation of the branch predicates of the program's as true and false in
    the course of the test suite's execution, Branch Coverage is also referred
    to as ``Predicate Coverage''.

    \item {\bf Edge Coverage.} Branch Coverage is also known as ``Edge
    Coverage'' in the context of the CFG. In terms of the CFG, the two are not
    strictly equivalent, as Branch Coverage is really just concerned with the
    edges we've labelled ``T'' and ``F'' in our examples. Yet, if these edges
    are covered, all linear sequences of edges will also be covered. So in
    practice, Full Edge Coverage and Full Branch Coverage are equivalent to one
    another.

\end{itemize}

\note{Video}{Next slide}

\subsection{Path Coverage} 

Covering each path through the CFG is the same as executing each path in the
program, and is known as Path Coverage. In CFG terms, a path is a sequence of
nodes through the CFG from the start node $s$, to the end node $e$. 

\note{Video}{Next slide}

However, in the presence of loops, there may be an infinite number of paths, so
obtaining full path coverage for all programs in general is not possible. For
{\tt countZeros}, the number of possible paths depends on the number of possible
elements in array --- and array can have up to $2^{32}$ elements!

Instead, a tester may opt to test the so-called ``simple paths'' only --- paths
that do not execute a particular node in the CFG more than once. Even this may
result in a large number of paths, however. As such, Path Coverage or Simple
Path Coverage are rarely applied beyond the individual method or function level.

\note{Video}{Next slide}

\subsection{Infeasible Test Requirements Derived From Control Flow Analysis} 

There is a problem lurking in this example for Branch Coverage. Can you see what
it is?

\note{Video}{Click to get CFG}

\note{Say}{Maybe it's easier to spot if I show you the CFG:}

\slide{2-2-infeasible-branch}

The false branch from line 11 of the example can never be executed. The variable
{\tt y} is {\it always} greater than zero by the time it gets to line 11. It
starts off at zero on line 3. But no matter whatever the value of {\tt x} in the
decision on line 5, the value of {\tt y} is increased --- on either line 6 or 8.
If there is no way we can execute it, there will be no test case that can cover
it.

We say that such branches are {\it infeasible}. 

\note{Video}{Next slide}

\note{Video}{Click to reveal statement info}

This can happen with Statement Coverage too. Because infeasible statements can
never be executed, they tend to point to dead code. Such statements could/should
be removed. It may be because they are only executed because of an infeasible
decision in the code, i.e., an infeasible branch.

\note{Video}{Click to reveal branches info}

Infeasible branches point to redundant decisions in the code. Removing these
should also be considered. However, sometimes there is a good reason for their
existence, and a good reason as to why they are infeasible: they could be due to
some environmental condition that cannot be satisfied during the course of
testing. For example, the branch condition may test for an out of memory
condition --- a scenario that is difficult to reproduce during testing, or the
decision may be only be true when a date-time constraint is satisfied. For
example, the {\tt isMorning()} branch predicate considered earlier will only
return true in the morning! To get around these problems, testers sometimes
write and use {\it mock objects} in unit tests that can stand in for real
objects. Mock objects simulate the behaviour of an original object while faking
the environmental conditions that are difficult to reproduce in testing. 

\note{Video}{Click to reveal paths info}

Infeasible paths are not necessarily the result of redundancy. Not all the paths
through the CFG are actually possible to take in actual program code.  It may
not be possible to take certain paths in the code, for example, because of
earlier decisions made in the path that rules out different possibilities later
down the line. This is another significant problem for Path Coverage. Detecting
infeasible paths is as hard as finding all bugs --- in general it is equivalent
to the Halting Problem. At best, it is an intractable problem.

All in all, infeasible paths are a more common occurrence than infeasible
branches, which turn are more common than infeasible statements --- or in other
words, the more complex the coverage criterion. 

\note{Video}{Next slide}

\subsection{Subsumption of Coverage Criteria Derived from the Control Flow Graph} 

\note{Video}{Click to reveal hierarchy}

Note that achieving full Branch Coverage means we've covered all the statements
in the program too --- i.e., we get Full Statement Coverage as well. Likewise,
if we achieve full Path Coverage, we get full Branch Coverage, which in turn
assures us of full Statement Coverage. In other words, these criteria form a
subsumption hierarchy. Path Coverage subsumes Branch Coverage, which in turn
subsumes Statement Coverage.

\note{Video}{Next slide}

\subsection{When to Use Structural Coverage Criteria} 

So, to conclude this part, when should we use Structural Coverage Criteria in
testing?

\note{Video}{Click to reveal}

The truth is that structural coverage tends to be used more as a metric to
indicate test suite quality than a method for {\it manually} developing test
cases and test suites. (We will come back to this point when we consider {\it
automated} test case generators in Week 4.)

That is, the level of structural coverage obtained by a test suite is useful for
helping us understand how {\it much} of our code has been executed by an
existing test suite, rather than writing tests, per se.

\note{Video}{Click to reveal}

The rationale for this is fairly simple: you wouldn't necessarily want to
release parts of your software if it hadn't been exercised by at least one test.
However, slavishly adding more and more tests with the aim of specifically
achieving 100\% Structural Coverage is seldom a worthwhile endeavour either. It
is better to go back to the design of your tests and consider why these cases
were missed in the first place, and whether the cost of writing them and adding
them to your test suite justifies the effort to do so.

\note{Video}{Click to reveal}

As such, the simplest form of structural coverage, Statement (or Line) Coverage
is frequently used metric in industry, and there are many tools available that
support its measurement. 

\note{Video}{Click to reveal}

Yet, Branch Coverage is a stronger criterion, and is obtainable without much
more additional effort. 

\note{Video}{Click to reveal}

Path Coverage, as the strongest criterion, however, is not seldom used and is
often intractable.

\end{document}