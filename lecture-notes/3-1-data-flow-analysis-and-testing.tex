\input{common/latex-teach-common/page-setup}
\input{common/latex-teach-common/packages}
\input{common/latex-teach-common/macros}

\renewcommand{\title}[2]{
    \documenttitle{Com3529}{Software Testing \& Analysis}{#1}{#2}
}

% URLs
\newcommand{\coderepourl}{\url{https://github.com/philmcminn/com3529-code}\xspace}

% Code snippets
\newcommand{\code}[1]{\mbox{\tt #1}\xspace}

% README file
\newcommand{\readmefile}{{\tt README.md}\xspace}

% Directories
\newcommand{\lecturesdirectory}{\code{lectures}\xspace}
\newcommand{\practicalsdirectory}{\code{practicals}\xspace}

% Packages
\newcommand{\lecturespackage}{\code{uk.ac.shef.com3529.lectures}}
\newcommand{\lecturesexecutionpackage}{\code{uk.ac.shef.com3529.lectures.execution}}
\newcommand{\practicalspackage}{\code{uk.ac.shef.com3529.practicals}}

% Classes
\newcommand{\calendarclass}{\code{Calendar}}
\newcommand{\randomlytesttriangleclass}{\code{RandomlyTestTriangle}}
\newcommand{\signutilsclass}{\code{SignUtils}}
\newcommand{\spacedefenderclass}{\code{SpaceDefender}}
\newcommand{\stringutilsbuggyoneclass}{\code{StringUtilsBuggy1}}
\newcommand{\triangleclass}{\code{Triangle}}
\newcommand{\weekoneclass}{\code{Week1}}
\newcommand{\teststringutilsbuggyoneclass}{\code{TestStringUtilsBuggy1}}
\newcommand{\testweekoneclass}{\code{TestWeek1}}


% Methods
\newcommand{\classifymethod}{\code{classify}}
\newcommand{\daysbetweentwodatesmethod}{\code{daysBetweenTwoDates}}
\newcommand{\instrumentedclassifymethod}{\code{instrumentedClassify}}
\newcommand{\randomlytestclassifymethod}{\code{randomlyTestClassify}}
\newcommand{\signmethod}{\code{sign}}

% for logic tables
\newcommand{\LTTrue}{T}
\newcommand{\LTFalse}{F}


\booltrue{shownotes}


\begin{document}

\title{Coverage Criteria}{3.1 White-Box Coverage Criteria based On \\ Data Flow Analysis}


\note{Video}{Talking head}

\section{Introduction}

The set of coverage criteria we're going to look at in this lecture consider
that in order to adequately test a program we should focus on the ``flow'' of
data values. Specifically, we should try to ensure that data values in the
program are created and used correctly. Identifying where data values are
created, used, and the paths that run between points in the program where they
are created and used is called {\it data flow analysis}. 

\note{Video}{First slide}

\section{An Introduction to Data Flow Analysis}

\subsection{Definitions and Uses}

A {\bf definition} is a location in a program where a value for a variable is
stored into memory. Examples include assignment statements (the variable being
``defined'' is the variable being assigned to) or the beginning of a method called
with parameters (the variables being defined are the parameter variables):

\begin{center} 
    \scalebox{0.8}{
    \begin{tabular}{l} 
        \verb$int x = y; // "x" is defined$\\
\verb$$\\
\verb$public int method(int a) { // "a" is defined with whatever value "method" is called with$\\
\verb$    // ...$\\
\verb$}$\\
    \end{tabular}}
\end{center}   

\note{Video}{Advance slide}

Here {\tt x} is defined when it is assigned the value of {\tt y}. Also, the
input variable {\tt a} to the method called ``{\tt method}'' seen here is
defined when it is assigned the value that is passed into the method as the
first parameter.

\note{Video}{Advance slide}

A {\bf use} is a program location where a variable's value is accessed, for
example, as part of a predicate, when a method on it is invocated (if it is an
object), it is output (e.g., via a \code{System.out.println} statement), or
returned from a method via a {\tt return} statement:

\begin{center} 
    \scalebox{0.8}{
    \begin{tabular}{l} 
        \verb$int x = y; // "y" is being used$\\
\verb$$\\
\verb$System.out.println("Hello " + name); // "name" is being used$\\
\verb$$\\
\verb$o.update(); // "o" is being used$\\
\verb$$\\
\verb$if (a > b) { // "a" and "b" are being used$\\
\verb$    // ...$\\
\verb$}$\\
\verb$$\\
\verb$return result; // "result" is being used$\\
    \end{tabular}}
\end{center}  

\note{Video}{Advance slide}

Here are some examples of uses.

The variable {\tt y} is used to define the value of {\tt x}. Similarly, the variable
{\tt name} is used as part of the string passed into the {\tt
System.out.println} method. 

The object {\tt o} is used when the method {\tt update} is called on it. The
variables {\tt a} and {\tt b} are used in the branch predicate of the {\tt if}
statement. Finally, the variable {\tt result} is used when it is passed out of
a method.

\note{Video}{Advance slide}

Note that some statements are definitions {\it and} uses, as in the ``{\tt int x = y}''
example above --- {\tt x} is defined, while {\tt y} is used. 

\note{Video}{Advance slide}
\note{Video}{Advance slide to let other part of slide appear}
\note{Video}{Advance slide to let other part of slide appear}
The {\it same} variable may be defined and used in the same statement, e.g. {\tt
x} in ``{x = x + 1}'' or the equivalent statements ``{\tt x += 1}'' and ``{\tt x ++}''.

\note{Video}{Talking Head}

Data flow testing criteria are specifically target values carried from
definitions to uses, known as {\bf definition-use} pairs. The criteria exercise
definition-use pairs in various ways.

\note{Video}{Next slide}

The first step of data flow analysis required for data flow testing criteria
involves annotating the CFG with the variables defined and used at each node of
the CFG. Formally, we refer to the set of variables defined at a node $n$ of a
CFG as $\mathit{defs}(n)$ and the variables used at a node $n$ as $\mathit{uses}(n)$.

Here's an example, using the \signmethod of the \signutilsclass.

As you can see, each node is labelled with the definitions and uses taking place
at that node, through the sets $\mathit{defs}$ and $\mathit{uses}$.

% TODO slide

\subsection{Definition-Use Paths}

\note{Video}{Next slide}

Here you can also see how the variable {\tt r} is defined at node 2, and then used
later at node 12 (also the end node $e$, of the CFG). This is called a {\it
definition-use} pair.

% Define a path as a sequence of nodes?? Should this be in the definition 
% of path coverage / CFGs from earlier

It's important to note when performing data flow analysis is that despite a
variable $v$ being defined at a node $n_i$, the value assigned to $v$ at $n_i$
may not be the value of $v$ accessed at a use appearing later in the program at
$n_j$.

The first, perhaps obvious reason for this is that there may be no path running
between $n_i$ and $n_j$ in the CFG. For example, suppose in the \signmethod,
that there was a statement {\tt return r} immediately following the definition
of {\tt r} at node 5. There would be no path from the definition of {\tt r} at
node 5 to the use of {\tt r} at node e (12), since the method would have already
exited.

\note{Video}{Next slide -- Killing definitions}

The second, perhaps less obvious reason, is that $v$'s value may be changed by
{\it another} definition that is executed {\it after} $n_i$ but {\it before}
$n_j$. Such a ``re-definition'' of the variable is referred to as a {\bf killing
definition}. For example, the definition of {\tt r} at node 2 is ``killed'', or
re-defined at node 5 or node 9. 

\note{Video}{Next slide -- Definition-clear path}

We are interested in paths between the definition and use of a variable $v$
where the path contains no killing definitions of $v$, otherwise known as a {\bf
definition-clear} path. The path from node 2 to node 12 is only definition-clear
with respect to {\tt r} if it avoids the re-definition of {\tt r} at nodes 5 and
9. (For this to happen, {\tt r} needs to be 0.)

Formally, a path from $n_i$ to $n_j$ is definition-clear with respect to a
variable $v$ if for each node $n_k$ on the path between $n_i$ and $n_j$, (i.e.,
$n_k \neq n_i \wedge n_k \neq n_j$), $v \notin \mathit{defs}(n_k)$. That is,
none of the nodes between $n_i$ and $n_j$ is a killing definition. 
%
If a definition-clear path exists from a definition of $v$ at $n_i$ to a
use of {\tt v} at $n_j$, the definition of $v$ is said to {\bf reach} the use
at $n_j$.

\note{Video}{Next slide -- Definition-use path}

We define a {\bf definition-use path} with respect to some variable $v$ as a
path between $n_i$ and $n_j$, where $v \in \mathit{defs}(n_i)$ and $v \in
\mathit{uses}(n_j)$, and that is definition-clear with respect to $v$.

We will refer to the complete set of definition-use paths present in a
program/CFG as the set $\mathit{DU}$. 

\note{Video}{Next slide -- table}

With \signmethod, the $DU$ (the last column of this table) is relatively small:

\begin{center}
\begin{tabular}{rrrrl}
    \toprule
    No. & Variable & Definition & Use & Definition-Use Path \\
    \midrule
    1 & {\tt r}  &          2 &   e & 2 $\rightarrow$ 4 $\rightarrow$ 8 $\rightarrow$ e \\
    2 & {\tt r}  &          5 &   e & 5 $\rightarrow$ 8 $\rightarrow$ e \\
    3 & {\tt r}  &          9 &   e & 9 $\rightarrow$ e \\
    4 & {\tt n}  &          s &   4 & s $\rightarrow$ 2 $\rightarrow$ 4 \\
    5 & {\tt n}  &          s &   8 & s $\rightarrow$ 2 $\rightarrow$ 4 $\rightarrow$ 8 \\
    6 & {\tt n}  &          s &   8 & s $\rightarrow$ 2 $\rightarrow$ 4 $\rightarrow$ 5 $\rightarrow$ 8 \\
    \bottomrule
\end{tabular}
\end{center}

There are six definition-use paths overall, three stemming from the definition of the
input parameter {\tt n} at the start node $s$, and the other three stemming from
the three different definitions of the variable {\tt r} at nodes 2, 5, and 9
respectively. Each definition of {\tt r} is eventually used at the return
statement, node 12, the end node $e$ of the CFG.

\note{Video}{Talking head}

\section{Data Flow Coverage Criteria}

The test requirements for data flow coverage criteria are to execute different
subsets of the complete set of definition-use paths present in a program, i.e.,
$\mathit{DU}$.

\note{Video}{All Defs coverage slide}

\subsection{All Defs Coverage}

{\bf All Defs Coverage} requires the test suite to execute at least one
definition-use path from each definition in the program. Or, to put it another
way, that {\it each definition reaches at least one use} in at least one test
case of the test suite.

More formally, All Defs Coverage involves constructing a series of subsets of
$\mathit{DU}$, one for each pair $(n, v)$, where $v \in \mathit{defs}(n)$. It
then requires one path from each subset to have been executed by the test suite.

\note{Video}{Advance slide for first three ticks, then again for 4}

Referencing the table then, All Defs Coverage for the \signmethod would include
test requirements 1, 2, 3 (each of the different definitions of {\tt r}) and one
of 4, 5 or 6 (different definition-use paths from the definition of {\tt n} at
the start node, s).

\note{Video}{All uses coverage slide}

\subsection{All Uses Coverage}

The {\bf All Uses} coverage criterion requires the test suite to execute at
least one definition-use path from each definition in the program to each use.
Or, to put it another way, it ensures that {\it each definition reaches each
possible use} in at least one test case of the final test suite. 

More formally, All Uses constructs a series of subsets of $DU$ for each unique
triple $(v, n_i, n_j)$ in the program/CFG, % is it program or CFG?
for a variable $v$, $v \in \mathit{defs}(n_i)$ and $v \in \mathit{uses}(n_j)$. It then
requires one path from each subset to have been executed by the test suite.

\note{Video}{Advance slide for first three ticks, then again for 4, then again for 5}

Referencing the table, All Uses Coverage for the \signmethod would include
test requirements 1, 2, 3 (each of the different definitions of {\tt r}), then 
test requirement 4, and one of test requirements 5 or 6. Test requirements 5 and
6 involve different paths from the same definition of {\tt n} at node s to its
use at node 8 (and therefore either may be included), whereas test requirement 4
involves the only definition-use path of {\tt n} from s to 4.

As such, All Uses subsumes All Defs.

\note{Video}{All uses coverage slide}

\subsection{All Def-Use Paths Coverage}

Finally, the {\bf All Def-Use Paths} coverage criterion requires every path in
$DU$ to be executed. This means including all the test requirements in the table.

\note{Video}{Advance slide for all ticks}

As such, All Def-Use Paths subsumes All Uses and All Defs.

One disadvantage of Data Flow Coverage Criteria is that not all of the
definition-clear paths are necessarily feasible. Sometimes, a choice of path can
mitigate this problem, otherwise, the test requirement cannot be satisfied and
must be dropped.


\note{Video}{Talking Head}

\section{When Should You Use Data Flow Testing?}

Data Flow Testing is particularly useful at catching bugs where program
variables are used to encapsulate the current {\it state}. Program state is
typically captured using boolean variables which act as flags denoting the
program is or is not in some particular state, or variables of an enumerated
type to capture the fact that a program is in one of several potential states.

Data Flow Coverage Criteria, can be used to detect if the state has been set
incorrectly in the program by tracking where a state variable is defined and
where it is later used. 

Variables representing state are particularly important in objects. Because
methods can be called in any sequence, an object cannot rely on the order of
program statements to encapsulate the current state of the program on its own.
Instead, the state must be stored and saved in an instance variable for methods
called later to then look up and reference.

\note{Video}{Class slide}

Take for example this class representing a vending machine. Whether a product
can be vended via the {\tt vend()} method depends on whether the {\tt allowVend}
boolean instance variable is set to {\tt true}. This happens after two coins
have been entered, via the {\tt addCoin()} method:

% \slide

The instance variable {\tt currentCoins} represents the number of coins that
have currently been inserted into the machine, via the {\tt addCoin()} method
has been invocated on the method. The {\tt returnCoins} method will return these
coins to the user (reseting {\tt currentCoins} to zero).

When the machine successfully vends a product, the total number of coins held by
the machine (represented by the {\tt totalCoins} instance variable) is increased
by the number of current coins ({\tt currentCoins}), {\tt currentCoins} is reset
to zero. Finally, {\tt allowVend} is set to false, since the machine should not
vend more products until further coins have been inserted into the machine via
{\tt addCoins}.

Since methods can be called at any time on an object, in any particular
sequence, we need a special form of CFG to represent this fact, if we are to
perform Data Flow Analysis. We will use a variant of the CFG for an object of a
class, referred to as the Class Control Flow Graph for the object, or CCFG for
short. 

\note{Video}{CCFG slide}

% \slide

A CCFG begins with the CFG for the object's constructor, and then the
possibility to call each method in a potentially continuous loop stemming from a
special node, c.

Consider one of the definition-pairs for {\tt currentCoins}, which is defined at
node 13 in the {\tt returnCoins} method and used at node 25 in the {\tt vend}
method.

Does a definition-clear path exist between nodes 13 and 25?

\note{Video}{Advance slide}

Yes, a path exists through the CCFG. It requires {\tt vend()} to be executed
after {\tt returnCoins()}. 

Is this definition-clear path feasible? Well, actually it is, as we'll see in a
moment. 

Should it be feasible? No!

In theory, the definition-use path from nodes 13 to 25 should not be feasible
--- the use of {\tt currentCoins} is in a section of code dedicated to vending a
product, but all of the coins have just been returned to the user in the
definition of {\tt currentCoins} in {\tt returnCoins}.

Yet, the path {\it is} feasible. Consider the method invocation sequence {\tt
VendingMachine()} $\rightarrow$ {\tt addCoin()} $\rightarrow$ {\tt addCoin()}.
Two coins are added, and {\tt allowVend} is set to true. We can now invocate
{\tt returnCoins()} (executing node 13) followed by {\tt vend()} (which
successfully executes node 25). 

There is a bug in the class! Where is the bug?

The {\tt returnCoins} method should also set {\tt allowVend} to false, but it
neglects to do so. This means that if two coins were previously entered, even if
they are returned to the use, the product can be vended. If {\tt allowVend} was
set to false after node 13, the path to node 25 would never be feasible.

The example therefore shows how state bugs in code can be uncovered using Data
Flow Analysis and testing. 

\end{document}